%&custom12pt


\begin{document}

\baselineskip 22pt

\centerline{\bf Abstract}

 
Recent research has shown that `rich' households save at much higher 
rates than others (see Carroll~\cite{carroll:richsave}; Dynan, 
Skinner, and Zeldes~\cite{dsz:richsave}; Gentry and 
Hubbard~\cite{gentry&hubbard:wealthysave}; 
Huggett~\cite{huggett:wealth}; 
Quadrini~\cite{quadrini:entrepreneurship}).  This paper documents 
another large difference between the rich and the rest of the 
population: portfolios of the rich are heavily skewed toward risky 
assets, particularly investments in their own privately held 
businesses.  The paper explores three possible explanations of these 
facts.  First, perhaps there is exogenous variation in risk tolerance, 
so that highly risk tolerant households engage in high-risk, 
high-return activities, and the risk-lovers who are lucky constitute 
the rich.  A second possibility is that capital market imperfections 
{\it a la} Gentry and Hubbard~\cite{gentry&hubbard:wealthysave} and 
Quadrini~\cite{quadrini:entrepreneurship} require entrepreneurial 
activities to be largely self-financed, and these same imperfections 
imply that entreprenurial investment will yield high average returns.  
The final possibility is that wealth enters households' utility 
functions directly as a luxury good as in 
Carroll~\cite{carroll:richsave} (one interpretation is that this 
reflects the utility of anticipated bequests), implying that risk 
aversion declines as wealth rises.  The paper concludes that the 
overall pattern of facts suggests both Carroll-style utility and 
Gentry/Hubbard-Quadrini style capital market imperfections are 
important.

\begin{table}[h]
Christopher D. Carroll \\
Department of Economics \\
The Johns Hopkins University \\
Baltimore, MD  21218-2685 \\
ccarroll@jhu.edu \\
(410) 516-7602 (office) \\
(303)-845-7533 (fax) \\
\end{table}

\clearpage\vfill\eject
\input Portable:latex:texhtml.tex

\bibliographystyle{Portable:latex:econometrica_fullnames}
\bibliography{Portable:latex:economics}

\end{document}
