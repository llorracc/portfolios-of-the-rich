\item[1983 SCF] The 1983 SCF did not ask about the investment strategy 
or risk characteristics of mutual funds or retirement accounts, so we 
had to make educated guesses based on other information.  Tax-free 
mutual funds were allotted to the 'fairly safe' category because such 
funds consist almost exclusively of state and local government bonds, 
direct holdings of which we put in this category.  Taxable mutual 
funds were allotted to the 'risky' category, because in the early 
1980s these funds typically contained a mix of stocks and bonds.  The 
calculation of risky, clearly safe, and fairly safe defined 
contribution pensions uses the institution that held the IRA/Keogh 
accounts as a proxy for investment direction.  If a real estate 
investment company held the accounts, then those defined contribution 
pensions were considered risky.  If a commercial bank, savings and 
loan, or credit union held the accounts, then those assets were 
considered fairly safe.  If a brokerage, insurance company, employer, 
school/college/university, investment management company, or the AARP 
held the accounts, the defined contribution pensions were split 50/50 
between the fairly safe and risky.  In the case that the household had 
no IRA/Keogh accounts, but had a thrift pension account, the assets 
were considered fairly safe.

\item[1989-1995 SCF] These surveys asked about the investment strategy 
for mutual funds and retirement accounts.  Funds and accounts that 
consisted exclusively of one category of asset (such as stock or bond 
mutual funds) we allocated in the same way that we allocated direct 
holdings of that asset type.  Mutual funds and accounts that contained 
a mix of stocks and safe or fairly safe bonds were allocated 
half-and-half to the 'fairly safe' and 'risky' categories.  Accounts 
invested in real estate, commodities or limited partnerships were put 
in the `risky' category.

