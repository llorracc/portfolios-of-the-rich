%custom12pt\begin{document}
% These numbers were generated by the program riskav.do
\begin{table}
\caption{Risk Aversion By Income and Net Worth, 1992 and 95 SCFs}
\begin{center}
\begin{tabular}{|c|c|c|c|c|}
\hline
\multicolumn{1}{|c|}{\textbf{Survey Year}} & \multicolumn{2}{|c|}{\textbf{%
1992}} & \multicolumn{2}{|c|}{\textbf{1995}} \\ \hline
\multicolumn{1}{|c|}{} & Mean & \% No Risk & Mean & \% No Risk \\ \hline
\multicolumn{1}{|c|}{\text{Income Percentiles}} &  &  &  &  \\ 
\multicolumn{1}{|c|}{\textit{99-100}} & 2.5 & 3.8 & 2.6 & 6.2 \\ 
\multicolumn{1}{|c|}{\textit{80-98.9}} & 2.8 & 16.9 & 2.8 & 16.1 \\ 
\textit{0-79.9} & 3.3 & 48.7 & 3.2 & 40.1 \\ \hline
\multicolumn{1}{|c|}{\text{Net Worth Percentiles}} &  &  &  &  \\ 
\multicolumn{1}{|c|}{\textit{99-100}} & 2.6 & 11.5 & 2.5 & 6.5 \\ 
\multicolumn{1}{|c|}{\textit{80-98.9}} & 2.9 & 21.6 & 2.8 & 17.6 \\ 
\multicolumn{1}{|c|}{\textit{0-79.9}} & 3.3 & 48.4 & 3.2 & 43.8 \\ \hline
\multicolumn{5}{p{.75\linewidth}}{\tiny Notes: The table summarizes 
answers to the following question: ``Which of the statements on this 
page comes closest to the amount of financial risk that you (and your 
spouse/partner) are willing to take when you save or make investments?  
1.  Take sustantial financial risks expecting to earn substantial 
returns; 2.  Take above average financial risks expecting to earn 
above average returns; 3.  Take average financial risks expecting to 
earn average returns; 4.  Not willing to take any financial risks} 
\end{tabular}
\label{table:riskaver}
\end{center}
\end{table}

%\end{document}
