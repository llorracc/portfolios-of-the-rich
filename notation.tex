%&LaTeX
\documentclass[12pt]{article}
\usepackage[]{Internal:latex:vmargin}

\setmarginsrb{1in}{1in}{1.0in}{1.25in}{0pt}{0pt}{0pt}{1.0in}
% {left}{top}{right}{bottom}



\begin{document}
\centerline{\LARGE \bf Proposed Notational Conventions}
\vspace{.25in}

This document proposes some notational rules drawn primarily from the 
recent finance text by Campbell, Lo, and MacKinlay, though with some 
important deviations.

\section{General Rules}
\begin{itemize}

\item $R$ with no subscripts will be taken to be the constant, 
riskless gross rate of return (`gross' meaning including return of 
principal) and $r$ will signify net rate of return, where gross and 
net returns are related by $R=1+r$.

\item The timing convention is that a variable is dated $t$ if it is 
known by the end of period $t$.  

\item If you are working with three assets or fewer (including a 
riskless asset), designate a single Roman letter as the `mnemonic' for 
each asset other than the riskless asset (which by convention will be 
the `default' asset and needs no mnemonic), and subscript variables 
that refer to that asset with the mnemonic letter.  For a risky asset 
that is supposed to reflect the characteristics of equity markets, use 
the mnemonic `e'; that is, the return on equities between period $t$ 
and $t+1$ would be designated $R_{e,t+1}$.

\item If you are working with more than three assets, use the variable 
$i$ as a subscript to indicate asset $i$ out of a list of up to $N$ 
assets.  Thus $R_{i,t}$ indicates the return on asset $i$ held from 
the end of period $t-1$ into period $t$.  

\item $w_{i,t}$ indicates the portfolio weight given to asset $i$ for 
the holding period from the end of period $t$ into period $t+1$.

\item Use boldface for vectors and matrices, and regular face for 
scalars.  Thus $w_{i,t}$ is a scalar and $\mathbf{w}_{t}$ is the vector of
the values $w_{i,t}$ for all $i$.

\item $\mathcal{N}(\mu,\sigma^{2})$ denotes the normal distribution 
with mean $\mu$ and standard deviation $\sigma$.

\item The mean expectation of a variable is designated by a bar over 
the variable.  Thus the statement that $r_{i} \sim 
\mathcal{N}(\overline{r}_{i},\sigma^{2}_{r_{i}})$ indicates that 
$r_{i}$ is distributed normally with mean value $\overline{r}$.

\item Pr(.) denotes the probability of an event

\item $E_{t}[X_{s}]$ is the expectation given all information known as 
of time $t$ of the value of variable $X$ as of time $s$.  The 
unconditional expectation of a variable whose value varies with time 
is given by $E [X_{t}]$.

\item A $\sim$ over a variable in an expectations formula indicates 
that the variable's value is stochastic viewed from the perspective of 
the period with respect to which the expectation is being taken.  Thus 
if $r_{i}$ is distributed lognormally as specified above, we can write 
$E_{t}[ \tilde{r}_{i,t+1}] = \overline{r}_{i}$.

\end{itemize}
\section{Parameter Definitions}

\begin{eqnarray*}
	\beta & - & \mbox{Time discount factor between periods}  \\
	\rho & - & \mbox{Coefficient of Relative Risk Aversion}  \\
	X_{t} & - & \mbox{Gross resources (`cash-on-hand') available for spending in period $t$}  \\
	S_{t} & - & \mbox{`Savings' left over at the end of period $t$ after consumption}  \\
	C_{t} & - & \mbox{Consumption in period $t$}  \\
	V_{t} & - & \mbox{Value function}  \\
	u_{t} & - & \mbox{Utility function}  \\
	B_{t} & - & \mbox{Utility from wealth/bequest function}  \\
	W_{t} & - & \mbox{Proportion of portfolio in risky assets}  \\
	R_{t} & - & \mbox{Total portfolio-weighted return between $t-1$ and $t$}  \\
	Y_{t} & - & \mbox{Noncapital income in period $t$}  \\
	P_{t} & - & \mbox{Permanent income in period $t$}  \\
	\epsilon_{t} & - & \mbox{Transitory shock to noncapital income in period $t$}  \\
	\eta_{t} & - & \mbox{Shock to permanent income in period $t$}  \\
	d_{t} & - & \mbox{Probability of dying at the end of period $t$}  \\
	p_{z} & - & \mbox{Probability that the variable $z$ is equal to zero}  \\
\end{eqnarray*}
\section{A Specific Problem} 
Using these conventions and a few others I will solve the notation 
problem recursively from the last period of life.

Consider the optimization problem in the last period of life.  The 
consumer obtains utility $u(C_{T})$ from consumption and $B(S_{T})$ 
from any wealth $S_{T}$ that is unconsumed and thus is left behind as 
a bequest.  The value function gives the utility obtained from 
choosing last-period consumption $C_{T}$ optimally:
\begin{eqnarray*}
	V_{T}(X_{T}) & = & \max_{\{C_{T}\}} u(C_{T}) + B(S_{T})  \\
	 & \mbox{such that} &   \\
	S_{T} & = & X_{T}-C_{T} \\
	S_{T} & \geq & 0 
\end{eqnarray*}
In the second-to-last period of life, the consumer must choose not 
only the level of consumption but also how to allocate any savings 
$S_{T-1}$ not consumed between the riskless asset and $N$ different 
potential risky asset categories.  Designate the proportion (or 
weight) of $S_{T-1}$ put into each of these possible assets 
$w_{i,T-1}$, and define $W_{T-1}=(1-\sum_{i=1}^{N} w_{i,T-1})$ as the 
total portfolio share in all risky assets implying that the portfolio 
share in the riskless asset will be (1-$W_{T-1}$).  Define the gross 
return on asset category $i$ between period $T-1$ and period $T$ as 
$R_{i,T}$ as per the guidelines, and define the portfolio-weighted 
average rate of return on savings between periods $T-1$ and $T$ as
\begin{equation}
R_{T} = R(1-W) + \sum_{i=1}^{N} w_{i,T-1} R_{i,T}.
\end{equation}
Designating the total noncapital income that the household receives
in the last period of life $Y_{T},$ total cash-on-hand in period $T$
will be given by
\begin{equation}
X_{T} = R_{T} S_{T-1} + Y_{T}.
\end{equation}
In addition, assume that there is some probability $d_{T}$ that the 
consumer will die after performing this period's consumption but 
before beginning next period.  Finally, assume that the consumer is 
not allowed to die in debt and therefore cannot borrow.  The 
consumer's problem under these circumstances is 
\begin{eqnarray*}
	V_{T-1}(X_{T-1}) & = & \max_{\{C_{T-1},\mathbf{w}_{T-1}\}} u(C_{T-1}) + (1-d_{T-1}) \beta E_{T-1} V_{T}(X_{T}) + d_{T-1} B(S_{T-1})  \\
	 & \mbox{such that} &   \\
	X_{T} & = & R_{T} S_{T-1} + Y_{T} \\
	S_{T-1}& = & X_{T-1}-C_{T-1} \\
	S_{T-1}& \geq & 0 
\end{eqnarray*}
or, substituting the constraints into the problem, we have:
\begin{eqnarray*}
	V_{T-1}(X_{T-1}) = &  &   \\
	  \max_{\{C_{T-1},\mathbf{w}_{T-1}\}} & & u(C_{T-1}) + (1-d_{T-1}) \beta E_{T-1} \left[V_{T}(\tilde{R}_{T}[X_{T-1}-C_{T-1}]+\tilde{Y}_{T}) \right]+ d_{T-1} B(X_{T-1}-C_{T-1})  \\
	 \mbox{such that} &  & C_{T-1} \leq X_{T-1}
\end{eqnarray*}
An analogous equation will hold for all earlier periods of life, so that
we can state the general maximization problem as
\begin{eqnarray*}
	V_{t}(X_{t}) = &  &   \\
	  \max_{\{C_{t},\mathbf{w}_{t}\}} & & u(C_{t}) + (1-d_{t}) \beta E_{t} \left[V_{t+1}(\tilde{R}_{t+1}[X_{t}-C_{t}]+\tilde{Y}_{t+1})\right] + d_{t} B(X_{t}-C_{t})  \\
	 \mbox{such that} &  & C_{t} \leq X_{t}
\end{eqnarray*}
In the process of solving maximization problems numerically, it
is often useful to define a function which yields the expected
value associated with pursuing every possible choice for the
control variables.  For this problem, the only way in which $C_{t}$ and 
$X_{t}$ affect $V_{t+1}$ is through their effects on $S_{t}$.  
We can now define a function 
\begin{eqnarray*}
	\Omega_{t}(S_{t},w_{1,t},\ldots,w_{N,t}) & = & (1-d_{t}) \beta E_{t}\left[ V_{t+1}(\tilde{R}_{t+1}S_{t}+\tilde{Y}_{t+1})\right] + d_{t} B(S_{t})  \\
\end{eqnarray*}
which returns the expected value from pursuing any possible choice of 
$S_{t}$ and portfolio share configuration, and the maximization 
problem can be rewritten somewhat more simply as:
\begin{eqnarray*}
	V_{t}(X_{t})  & = &  \max_{\{C_{t},\mathbf{w}_{t}\}} u(C_{t}) + \beta \Omega_{t}(S_{t},\mathbf{w}_{t})  \\
  	 & \mbox{such that} & \\
  	 S_{t} & = &    X_{t}-C_{t} \\
  	 S_{t} & \geq & 0.
\end{eqnarray*}

Written in this way, the problem is formidably difficult to solve 
because it involves simultaneous nonlinear maximization with respect 
to $N+1$ choice variables ($C_{t}$ and $N$ portfolio shares).  One way 
of solving problems of this kind numerically is to define a series of 
functions
\begin{eqnarray*}
	\Omega_{N,t}(S_{t},w_{2,t},\ldots,w_{N-1,t}) & = & \max_{\{w_{N,t}\}} \Omega_{t}(S_{t},w_{2,t},\ldots,w_{N,t}) \\
	\Omega_{N-1,t}(S_{t},w_{2,t},\ldots,w_{N-2,t}) & = & \max_{\{w_{N-1,t}\}} \Omega_{N,t}(S_{t},w_{2,t},\ldots,w_{N-1,t})
\end{eqnarray*}
which, given a fixed choice for $S_{t}$ and all of the portfolio 
shares up to a given share, finds the optimal value for that portfolio 
share given optimal choice of the remainder of portfolio shares.  The 
logic of this process is exactly equivalent to the logic behind the 
traditional recursive solution to dynamic optimization problems, and 
the process leads eventually to a function which yields the optimal 
value of any value of savings $S_{t}$ given optimal choice of 
portfolio shares, $\Omega_{*,t}(S_{t})$ where the * is used to 
indicate that optimal choice of portfolio shares is happening in the 
background.  The problem now can be written
\begin{eqnarray*}
	V_{t}(X_{t})  & = &  \max_{\{C_{t}\}} u(C_{t}) + \Omega_{*,t}(X_{t}-C_{t})  \\
  	 & \mbox{such that} & \\
  	 C_{t} & \leq & X_{t}
\end{eqnarray*}

Problems of this kind are usually solved using the first order 
conditions.  Define the derivative of a function $f(x,y,\ldots)$ with 
respect to its arguments as $f^{x}, f^{y}, \ldots$.  There will be one 
first order condition with respect to the portfolio share of each 
possible investment, possibly along with short sales constraints that 
require $0 \leq w_{i,t} \leq 1$ for all $i$.  Then there will be $N$ 
first order conditions with respect to the portfolio shares, one for 
each of the series of equations listed above, and each of these will 
take the form:
\begin{eqnarray*}
	0 & = & \beta E_{t} \left[(\tilde{R}_{i,t+1}-R)S_{t}(1-d_{t})V_{t+1}^{X}(\tilde{X}_{t+1}) \right]. 
\end{eqnarray*}

The first order condition for this problem with respect to consumption 
can be written:
\begin{eqnarray*}
	0 & = & u^{c}(C_{t+1})+(1-d_{t})\beta E_{t} \left[\tilde{R}_{t+1} V_{t+1}^{X}(\tilde{X}_{T+1})\right] + d_{t}B^{S}_{t}(X_{t}-C_{t}). 
\end{eqnarray*}
This equation will define the optimal level of consumption for any 
given choice of portfolio shares $\mathbf{w}_{t}$, unless the 
$C_{t}$ which satisfies this equation is greater than $X_{t}$, in 
which case the no-dying-in-debt constraint will bind and the consumer 
will spend $C_{t}=X_{t}$.

A prototypical specification for the income process is as follows.  Realized
income in period $t$ is given by permanent income in period $t$ times a 
multiplicative shock $\epsilon_{t}$:
\begin{equation}
Y_{t} = P_{t} \epsilon_{t}
\end{equation}
Often we assume that there is some positive probability that 
$\epsilon_{t} = 0$; write this probability as $p_{\epsilon} = 
$Pr($\epsilon_{t}=0$).  Often we assume that if the realization of 
$\epsilon_{t}$ is nonzero then $\epsilon_{t}$ is distributed 
lognormally with a mean such that $E_{t-1} [\tilde{\epsilon}_{t}] = 
1$.  This implies that $E_{t-1} [\tilde{\epsilon_{t}} | \epsilon_{t} > 
0] = 1/(1-p_{\epsilon}).$ Using the fact that for a lognormally 
distributed variable $z$
\begin{eqnarray*}
	\log E[z] & = & E[\log z] + \frac{1}{2} \mbox{var}[\log z]  
\end{eqnarray*}
we have that
\begin{eqnarray*}
\log E[\epsilon | \epsilon>0] & = & E[\log \epsilon | \epsilon>0] + \mbox{var}_{t}[\log \epsilon | \epsilon > 0]
\end{eqnarray*}
Using the approximation that $\log[1/(1-p_{\epsilon})] \approx 
p_{\epsilon}$ if $p_{\epsilon}$ is small, this gives us that
\begin{eqnarray*}
p_{\epsilon} & \approx & E[\log \epsilon | \epsilon>0] + \mbox{var}_{t}[\log \epsilon | \epsilon > 0] \\
E[\log \epsilon | \epsilon>0] & \approx & p_{\epsilon}-\sigma_{\epsilon}^{2}/2 \\
\end{eqnarray*}
and thus our assumption is that 
\begin{equation}
\log \epsilon_{t} \sim \mathcal{N}(p_{\epsilon}-\sigma_{\epsilon}^{2}/2,\sigma^{2}_{\epsilon})
\end{equation}
Similarly, but more simply, we often assume that permanent income is 
growing at some rate $G_{t}$ from period to period but is hit by a 
multiplicative shock $\eta_{t}$:
\begin{eqnarray*}
	P_{t+1} & = & G_{t+1}P_{t}\eta_{t+1}  \\
\end{eqnarray*}
where usually the assumption is that $\eta_{t}$ is lognormally 
distributed such that $E_{t-1}[\eta_{t}] = 1$ implying that the 
distribution for $\eta$ is $\eta \sim \mathcal{N}(-\sigma_{\eta}^{2}/2,\sigma_{\eta}^{2})$.



\end{document}
\ifthenelse{\boolean{ShowFirstStuff}}{

\subsubsection{Labor Income}


Each household is characterized by a level of permanent income $P$ 
that evolves over time according to the process
\begin{math}
	P_{t} = \eta_{t}P_{t-1}
\end{math}
where $\log \eta_{t} \sim N(-\sigma_{\eta}^{2}/2,\sigma_{\eta}^{2})$ 
which implies that $E_{t-1} \eta_{t} = 1$ and following 
Carroll~\cite{carroll:brookings} we assume $\sigma_{\eta} = .1$.

Realized noncapital income $Y_{t}$ is given by permanent income 
multiplied by a transitory shock $\zeta_{t}$,
\begin{eqnarray*}
	Y_{t} & = & G_{t} P_{t} \zeta_{t}
\end{eqnarray*}
where $G_{t}$ captures the age profile of earnings, $\zeta_{t}$ is 
equal to zero (representing a spell of unemployment) with probability 
$p^{y}=.01$ and otherwise is lognormally distributed with standard 
deviation $\sigma_{\zeta}=.1$ and with a mean such that $E_{t-1} 
\zeta_{t} = 1$.


\subsubsection{Capital Income and Entrepreneurial Income}

Some consumers can invest a positive share of their portfolio $0 < 
\omega_{t}^{k} \leq 1$ in an entrepreneurial project.  In order to 
capture in the simplest possible way the higher return on 
self-financed projects emphasized by 
Quadrini~\cite{quadrini:entrepreneurship} and Hubbard and 
Gentry~\cite{gentry&hubbard:wealthysave}, the gross return on the 
entreprenurial project $R_{t}^{k}$ is scaled by a monotonically 
increasing function $\Omega(\omega_{t}^{k})$ such that $\Omega(1)=1$; 
that is, one can only reap the highest possible return on an 
entreprenurial project by investing one's entire net worth in that 
project.  The simplest choice of functional form for $\Omega$ would be 
linear, but for technical reasons it is more convenient to assume that 
$\Omega$ is quadratic.\footnote{Because $\Omega'(\omega_{t}^{k})$ 
appears in the first order condition, assuming that $\Omega$ is 
quadratic helps to guarantee that there will be a uniquely optimal 
portfolio share choice for $\omega_{t}^{k}$.}

It is difficult to know how to parameterize the rate of return for the 
entrepreneurial project.  Because entrepreneurial investment is 
undoubtedly riskier than stock market investment, it is clear that we 
should assume some `entrepreneurial premium,' 
$E_{t}R^{k}_{t+1}>E_{t}R^{s}_{t+1}$.  Our baseline assumption is that 
$\overline{R}^{k} = E_{t}R_{t+1}^{k}=1.16$ or $0.10$ higher than the 
expected return on stocks, and that the stochastic distribution of 
rates of return around $\overline{R}^{k}$ mimics the distribution of 
the rate of return on stocks except that there is a much larger 
probability of entrepreneurial `failure' (in which $R^{k}_{t+1}=0$) 
than there is of a stock market crash (in which $R^{s}_{t+1}=.1$).  
Specifically, our baseline assumption is that the probability of 
`bankruptcy' of the entreprenurial project is $p^{k}=.1$ 
annually.\footnote{Quadrini~\cite{quadrini:entrepreneurship} provides 
evidence that the rate of failure for entrepreneurs who have been 
self-employed for three years or longer is $.1$ annually.  The failure 
rate in the first two years of entrepreneurial activity is much 
greater, but the additional modelling complication and solution time 
required to capture this phenomenon did not seem worth the payoff of 
greater realism.}

To complete the specification of our treatment of entrepreneurial 
investment, we assume that $\lim_{\omega^{k}\rightarrow 0} 
\Omega(\omega^{k}) = R/ E_{t}R_{t+1}^{k}$; that is, if the 
consumer were to invest an infinitesimal amount $\epsilon$ in the 
entrepreneurial project, the expected rate of return would be $R$, the 
riskfree return, but the project would still bear the large {\it ex 
ante} risks associated with entrepreneurial ventures.

Not all households are allowed to pursue an entrepreneurial project in 
every period.  Any household that pursued an entrepreneurial project 
in period $t-1$ is eligible to continue that entrepreneurial activity.  
But period $t-1$ nonentrepreneurs are permitted to invest in an 
entrepreneurial project in period $t$ only if they are lucky enough to 
obtain a new  `idea,' an event that occurs with probability 
$\overline{n}$ in each period.  (Formally, $\tilde{n} \sim u[0,1]$ and 
one obtains an idea if $\tilde{n} \leq \overline{n}$).  

If a consumer who is currently engaged in an entrepreneurial activity 
decides not to invest in the project in this period, then that 
consumer cannot resume the project next period; he joins the pool of 
`nonentrepreneurs' and must await the next random draw of an 
entrepreneurial `idea.'  The variable $Q_{t} \in \{0,1\}$ indicates 
whether the consumer is eligible to pursue an entrepreneurial idea.

For technical reasons the best way to specify the portfolio allocation 
problem is to have consumers first decide how much of their total 
wealth to invest in the entrepreneurial project, and then decide how 
to allocate the remaining non-entrepreneurial wealth between the 
riskless asset and stocks.  Mathematically, $\omega_{t}^{k}$ of the 
entire portfolio is invested in the entrepreneurial project, leaving 
$(1-\omega_{t}^{k})$ to allocate between riskless and risky 
investment, with $\omega_{t}^{s}$ designating the portion of this 
remainder allocated to stocks.

To summarize, the rate of return earned by a consumer who saves 
$W_{t}$ in period $t$ is given by $\hat{R}_{t+1}W_{t}$ where:
\begin{eqnarray*}
	\hat{R}_{t+1} & = & 
	R(1-\omega^{k}_{t})(1-\omega^{s}_{t}) + 
	R_{t+1}^{s}\omega_{t}^{s}(1-\omega_{t}^{k}) + R_{t+1}^{k}Q_{t+1} 
	\Omega(\omega_{t}^{k})\omega_{t}^{k} \nonumber 
\end{eqnarray*}

Finally, we assume that running an entrepreneurial project consumes a 
minimum amount of time $\tau>0$ and thus reduces the amount of wage 
income the entrepreneur earns to $Y_{t+1}(1-\tau)$; our baseline 
assumption is $\tau=.1$.  This assumption is necessary to prevent 
consumers who have ever received an entrepreneurial idea from keeping 
the entreprenurial project `alive' by investing an infinitesimal 
amount $\epsilon$ in the project in every period.  Such a strategy 
would preserve the option value inherent in the entrepreneurial 
project at an arbitrarily small cost; requiring some minimum fixed 
commitment of time imposes a lower bound on the cost of keeping the 
entrepreneurial option alive.


\subsubsection{Utility}

The consumer derives utility from two sources: the ongoing utility 
from consumption obtained from the CRRA utility function $u(c) = 
c^{1-\rho}/(1-\rho)$, and the utility derived from the contemplation 
of the future bequest to be left.  The bequest utility function is of 
the form
\begin{eqnarray*}
	B(W) & = & \frac{(W+\lambda)^{1-\alpha}}{1-\alpha} \\
\end{eqnarray*}
where the parameter restriction $\alpha<\rho$ is imposed because it 
implies that bequests are a luxury good, and $\lambda>0$ is imposed 
because it implies that there will be some level of permanent income 
below which the bequest motive will be inoperative (the desired 
bequest would be negative, which is ruled out).

\subsubsection{Bellman's Equation}


Bellman's equation for this problem is:

\begin{eqnarray}
	V_{t}(X_{t},P_{t},Q_{t}) & =& \max_{\{C_{t},\omega_{t}^{s},\omega_{t}^{k}\}} 
	u(C_{t}) + \beta E_{t }\left[(1-d_{t+1}) 
	V_{t+1}(X_{t+1},P_{t+1},Q_{t+1})+ d_{t+1}B(W_{t}) \right] \nonumber\\ 
	 & \mbox{such that} &   	\label{eq:bellmanraw}  \\
	W_{t}   & = & X_{t}-C_{t} \nonumber \\
	X_{t+1} & = & \hat{R}_{t+1}W_{t} + Y_{t+1}(1-\tau Q_{t}) \nonumber \\
	Y_{t+1} & = & P_{t+1}\zeta_{t+1} \nonumber \\
	P_{t+1} & = & G_{t} P_{t} \eta_{t+1} \nonumber \\
	\hat{R}_{t+1} & = & 
	R(1-\omega^{k}_{t})(1-\omega^{s}_{t}) + 
	R_{t+1}^{s}\omega_{t}^{s}(1-\omega_{t}^{k}) + R_{t+1}^{k}Q_{t+1} 
	\Omega(\omega_{t}^{k})\omega_{t}^{k} \nonumber 
\end{eqnarray}
and
\begin{equation}
Q_{t+1} = 
\begin{cases}
1 & \mbox{if $\omega_{t}^{k} > 0$ and $Q_{t} = 1$} \\
1 & \mbox{if $\omega_{t}^{k} = 0$ and $\tilde{n} < \overline{n}$}
0 & \mbox{if $\omega_{t}^{k} = 0$ and $\tilde{n} \geq \overline{n}$}, \\
\end{cases} \nonumber 
\end{equation}

\section{Solution Procedure}


Define the function

\begin{equation*}
\Omega_{t}(W_{t},P_{t},Q_{t},\omega_{t}^{k},\omega_{t}^{s}) = E_{t 
}\left[(1-d_{t+1}) V_{t+1}(\hat{R}_{t+1} 
W_{t}+Y_{t+1},P_{t+1},Q_{t+1})+ d_{t+1}B(W_{t})\right]
\end{equation*}

Now define the value function associated with the optimal 
stock investment share, given the share chosen for the entrepreneurial 
project:

\begin{equation}
{\overline{\Omega}_{t}}(W_{t},P_{t},Q_{t},\omega_{t}^{k}) = 
\max_{\{\omega^{s}_{t}\}} 
\Omega_{t}(W_{t},P_{t},Q_{t},\omega_{t}^{k},\omega_{t}^{s}) 
\label{eq:FOCwrtomegak}
\end{equation}

Assuming there is an optimum for $\omega_{t}^{s}$ in the range $0 < 
\omega_{t}^{s} < 1$, and denoting the derivative of $\Omega_{t}$ with 
respect to $\omega_{t}^{s}$ as $\Omega_{t}^{s}$, that optimum 
will be characterized by the FOC:

\begin{eqnarray*}
	0 & = & \overline{\Omega}_{t}^{s}(W_{t},P_{t},Q_{t},\omega_{t}^{k})  \\
	0 & = & \beta E_{t}\left[ \frac{\partial X_{t+1}}{\partial 
	\omega^{s}_{t}} \left( (1-d_{t+1})
	V_{t+1}^{x}(X_{t+1},P_{t+1},Q_{t+1})\right) \right]\\
	0 & = & \beta E_{t} \left[(R^{s}_{t+1} - R)(1-\omega_{t}^{k})(W_{t}) \left((1-d_{t+1}) 
	V_{t+1}^{x}(X_{t+1},P_{t+1},Q_{t+1})\right)\right]
\end{eqnarray*}

Call the value of $\omega_{t}^{s}$ which satisfies the FOC $\hat{\omega}_{t}^{s}$. 
Then define 
\begin{equation}
\check{\omega}_{t}^{s} = 
\begin{cases}
1 & \mbox{if $\hat{\omega}_{t}^{s} > 1$,} \\
0 & \mbox{if $\hat{\omega}_{t}^{s} < 0$,} \\
\hat{\omega}_{t}^{s} & \mbox{if $0 < \hat{\omega}_{t}^{s} < 1$} 
\end{cases}
\end{equation}

Finally, the optimal stock investment value function 
$\overline{\Omega}_{t}$ is defined as:

\begin{eqnarray*}
	\overline{\Omega}_{t}(W_{t},P_{t},Q_{t},\omega_{t}^{k}) & = & \max 
	[\Omega_{t}(W_{t},P_{t},Q_{t},\omega_{t}^{k},\check{\omega}_{t}^{s}),
	\Omega_{t}(W_{t},P_{t},Q_{t},\omega_{t}^{k},0),
	\Omega_{t}(W_{t},P_{t},Q_{t},\omega_{t}^{k},1)]
\end{eqnarray*}

Similarly, define the value associated with the optimal entrepreneurial share 
as
\begin{eqnarray*}
	\overline{\overline{\Omega}}_{t}(W_{t},P_{t},Q_{t}) & = & \max_{\{\omega_{t}^{k}\}} 
	\overline{\Omega}_{t}(W_{t},P_{t},Q_{t},\omega_{t}^{k})
\end{eqnarray*}

Assuming there is an optimum for $\omega_{t}^{k}$ in the range $0 < 
\omega_{t}^{k} < 1$, and denoting the derivative of 
$\overline{\Omega}_{t}$ with 
respect to $\omega_{t}^{k}$ as $\overline{\Omega}_{t}^{k}$, that optimum 
will be characterized by the FOC:

\begin{eqnarray*}
	0 & = & \overline{\Omega}_{t}^{k}(W_{t},P_{t},Q_{t},\omega_{t}^{k})  \\
      & = & \beta E_{t} \left[
	R^{k}_{t+1}Q_{t+1}[ 
	\Omega(\omega_{t}^{k})+\Omega^{'}(\omega^{k}_{t})\omega_{t}^{k}]
	-(R(1-\omega_{t}^{s}) + 
      R_{t+1}^{s}\omega_{t}^{s}))W_{t} \left((1-d_{t+1}) 
	V_{t+1}^{x}(X_{t+1},P_{t+1},Q_{t+1})\right)\right] \\
\end{eqnarray*}
where implicitly we are assuming that the optimal stock share of 
nonentreprenurial wealth $\omega_{t}^{s}$ is being chosen optimally in 
the background.

Call the value of $\omega_{t}^{k}$ which satisfies the FOC 
$\hat{\omega}_{t}^{k}$.  Then define 
\begin{equation}
\check{\omega}_{t}^{k} = 
\begin{cases}
1 & \mbox{if $\hat{\omega}_{t}^{k} > 1$,} \\
0 & \mbox{if $\hat{\omega}_{t}^{k} < 0$,} \\
\hat{\omega}_{t}^{k} & \mbox{if $0 < \hat{\omega}_{t}^{k} < 1$} 
\end{cases}
\end{equation}

Then the expected value function given optimal portfolio choices 
$\overline{\overline{\Omega}}_{t}$ is defined as:

\begin{eqnarray*}
	\overline{\overline{\Omega}}_{t}(W_{t},P_{t},Q_{t}) & = & \max 
	[\overline{\Omega}_{t}(W_{t},P_{t},Q_{t},\check{\omega}_{t}^{k}),
	\overline{\Omega}_{t}(W_{t},P_{t},Q_{t},0),
	\overline{\Omega}_{t}(W_{t},P_{t},Q_{t},1)]
\end{eqnarray*}

And we can rewrite the Bellman equation (\ref{eq:bellmanraw}) as 

\begin{eqnarray*}
	V_{t}(X_{t},P_{t},Q_{t}) & =& \max_{\{C_{t}\}} 
	u(C_{t}) + \beta \overline{\overline{\Omega}}_{t}(X_{t}-C_{t},P_{t},Q_{t})
\end{eqnarray*}

for which the FOC is:
\begin{eqnarray*}
    0 & = & u'(C_{t}) + \beta 
    \overline{\overline{\Omega}}_{t}^{W}(X_{t}-C_{t},P_{t+1},Q_{t+1}) \\
	0 & = & u'(C_{t}) + \beta E_{t}\left[ \frac{\partial X_{t+1}}{\partial 
	C_{t}} \left( (1-d_{t+1})
	V_{t+1}^{x}(X_{t+1},P_{t+1},Q_{t+1}) + d_{t+1}B'(W_{t}) \right) \right]\\
	u'(C_{t}) & = & \beta E_{t} \left[\hat{R}_{t+1} \left((1-d_{t+1}) 
	V_{t+1}^{x}(X_{t+1},P_{t+1},Q_{t+1}) + d_{t+1}B'(W_{t})\right)\right] \\
	C_{t}^{-\rho} & = & \beta E_{t} \left[\hat{R}_{t+1} \left((1-d_{t+1}) 
	V_{t+1}^{x}(X_{t+1},P_{t+1},Q_{t+1}) + d_{t+1}(W_{t}+\lambda)^{-\alpha}\right)\right] \\
	C_{t} & = & \left(\beta E_{t} \left[\hat{R}_{t+1} \left((1-d_{t+1}) 
	V_{t+1}^{x}(X_{t+1},P_{t+1},Q_{t+1}) + d_{t+1}
	(W_{t}+\lambda)^{-\alpha}\right)\right]\right)^{1/-\rho} \\
\end{eqnarray*}
where again the optimal shares are assumed to be chosen for each value 
of $C_{t}$ considered.

Note that, using the Envelope theorem, we have that, if none of the 
implicit constraints are binding, the derivative of the Bellman 
equation with respect to $X_{t}$ is:

\begin{eqnarray*}
	V_{t}^{x} & = & \beta E_{t}\overline{\overline{\Omega}}_{t}^{s}(X_{t}-C_{t},P_{t},Q_{t})\\
\end{eqnarray*}

\section{The Last Period}

In the last period of life, the consumer knows for certain that he 
will not be alive to consume in the next period.  The maximization 
problem then becomes:

\begin{eqnarray*}
	V_{T}(X_{T},P_{T},Q_{t}) & = & \max_{\{C_{t}\}}  u(C_{t}) + B(X_{t}-C_{t})  \\
\end{eqnarray*}

for which the FOC is

\begin{eqnarray*}
	u'(C_{T}) & = & B'(X_{T}-C_{T})  \\
	C_{T}^{-\rho} & = & (X_{T}-C_{T}+\lambda)^{-\alpha}  \\
	C_{T} & = & (X_{T}-C_{T}+\lambda)^{\alpha/\rho}
\end{eqnarray*}

The envelope theorem tells us that, for consumers who plan on leaving 
a positive bequest, 
\begin{eqnarray*}
	\frac{dV_{T}}{dX_{T}} & = & \frac{\partial V_{T}}{\partial X_{T}}
	+\frac{\partial V_{T}}{\partial C_{T}} \frac{\partial C_{T}}{\partial 
	X_{T}}  \\
	 & = & B'(X_{T})  \\
\end{eqnarray*}


\section{Scaling Issues}

\begin{eqnarray*}
	V(x) & = & \frac{x^{1-\rho}}{1-\rho}  \\
	W(x) & = & [(1-\rho)V(x)]^{1/(1-\rho)} \\
	W'(x) &= & \frac{[(1-\rho)V(x)]^{1/(1-\rho) - 1}}{(1-\rho)} V'(x) 
	(1-\rho)\\
	      &= & [(1-\rho)V(x)]^{1/(1-\rho) - (1-\rho)/(1-\rho)}V'(x) \\
	      &= & [(1-\rho)V(x)]^{\rho/(1-\rho)} V'(x) \\
	      &= & [x^{1-\rho}]^{\rho/(1-\rho)} x^{-\rho} \\
	      &= & x^{\rho}x^{-\rho} \\
	      & =& 1
\end{eqnarray*}

\begin{eqnarray*}
	V'(x) & = & x^{-\rho}  \\
    V''(x) & = & (-\rho)x^{-\rho-1}  \\
	U(x) & = & [V'(x)]^{1/-\rho} \\
	U'(x) &= & \frac{[V'(x)]^{1/-\rho - 1}}{-\rho} V''(x) \\
	      &= & \frac{[V'(x)]^{1/-\rho - (-\rho/-\rho)}}{-\rho} V''(x) \\
	      &= & \frac{[V'(x)]^{(1+\rho)/-\rho}}{-\rho} V''(x) \\
	      &= & \frac{[V'(x)]^{-(1+\rho)/\rho}}{-\rho} V''(x) \\
	      &= & \frac{[x^{-\rho}]^{-(1+\rho)/\rho}}{-\rho} (-\rho)x^{-\rho - 1} \\
	      &= & [x^{-\rho}]^{-(1+\rho)/\rho} x^{-(1+\rho)} \\
	      & =& [x^{-1}]^{-(1+\rho)}x^{-(1+\rho)} \\
	      & =& x^{(1+\rho)-(1+\rho)} \\
	      & =& 1
\end{eqnarray*}

\section{Parametric Assumptions}

\begin{tabular}{lcr}
$\beta$ & - & .95 
\end{tabular}
}{} % End ifthenelse{ShowFirstStuff}

\section{Useful Derivations}
\subsection{Value Function In Second-To-Last Period of Life If Utility Comes Only From Wealth}
Consider a consumer who obtains utility simply from holding wealth,
\begin{eqnarray}
	V_{t}(W_{t}) & = & \frac{W_{t}^{1-\alpha}}{1-\alpha} + \beta E_{t}
	V_{T}(W_{T}) \nonumber \\
     & \mbox{such that} &  \nonumber \\
	W_{T} & = & \hat{R}_{T} W_{t} \nonumber
\end{eqnarray}
where $\hat{R}_{T}$ is the (possibly stochastic) return on the 
wealth portfolio.  For problems with this structure the 
infinite-horizon value function takes the form $V(W) = \gamma 
\frac{W^{1-\alpha}}{1-\alpha}$.  We can use the method of undetermined 
coefficients to obtain the formula for $\gamma$:
\begin{eqnarray*}
	\gamma \frac{W_{T-1}^{1-\alpha}}{1-\alpha} & = & 
	\frac{W_{T-1}^{1-\alpha}}{1-\alpha} + \beta E_{T-1}  \frac{(\hat{R}_{T} 
	W_{T-1})^{1-\alpha}}{1-\alpha}  \\
	 & = & \frac{W_{T-1}^{1-\alpha}}{1-\alpha}\left[1 + \beta E_{T-1} 
	 [\hat{R}_{T}^{1-\alpha}] \right]  \\
	\gamma & = & 1+\beta E_{t} [\hat{R}_{t+1}^{1-\alpha}]  \\
\end{eqnarray*}

\subsection{Infinite-Horizon Value Function If Utility Comes Only From Wealth}
Consider a consumer who obtains utility simply from holding wealth,
\begin{eqnarray}
	V_{t}(W_{t}) & = & \frac{W_{t}^{1-\alpha}}{1-\alpha} + \beta E_{t}
	V_{t+1}(W_{t+1}) \label{eq:wealthutil} \\
     & \mbox{such that} &  \nonumber \\
	W_{t+1} & = & \hat{R}_{t+1} W_{t} \nonumber
\end{eqnarray}
where $\hat{R}_{t+1}$ is the (possibly stochastic) return on the 
wealth portfolio.  For problems with this structure the 
infinite-horizon value function takes the form $V(W) = \gamma 
\frac{W^{1-\alpha}}{1-\alpha}$.  We can use the method of undetermined 
coefficients to obtain the formula for $\gamma$:
\begin{eqnarray*}
	\gamma \frac{W_{t}^{1-\alpha}}{1-\alpha} & = & 
	\frac{W_{t}^{1-\alpha}}{1-\alpha} + \beta E_{t} \gamma \frac{(\hat{R}_{t+1} 
	W_{t})^{1-\alpha}}{1-\alpha}  \\
	 & = & \frac{W_{t}^{1-\alpha}}{1-\alpha}\left[1 + \beta \gamma E_{t} 
	 [\hat{R}_{t+1}^{1-\alpha}] \right]  \\
	\gamma & = & 1+\beta \gamma E_{t} [\hat{R}_{t+1}^{1-\alpha}]  \\
	\gamma(1-\beta E_{t} [\hat{R}_{t+1}^{1-\alpha}]) & = & 1  \\
	\gamma & = & \frac{1}{1-\beta E_{t} [\hat{R}_{t+1}^{1-\alpha}]}
\end{eqnarray*}
\subsection{Portfolio Choice Between a Safe and a Risky Asset}
Now let's give them a meaningful portfolio choice.  Suppose that the 
agent can choose to put a fraction $\omega$ of her portfolio in a 
risky asset which has stochastic gross return $\tilde{R}_{s,t+1}$.  
Suppose $\log \tilde{R}_{s,t+1} \sim 
N(r_s,\sigma_{r_s}^{2})$.  Recall that for 
lognormally distributed variables we know that
\begin{eqnarray*}
	\log E_{t} \tilde{R}_{s,t+1}& = & E_{t}[ \log \tilde{R}_{s,t+1}] + \frac{1}{2} 
	\mbox{var}_{t}[\log \tilde{R}_{s,t+1}] \\
	 & = & r_s + \frac{1}{2}\sigma_{r_s}^{2}  \\
\end{eqnarray*}
and so the overall return on the portfolio will be
\begin{eqnarray*}
	\hat{R}_{t+1} & = & (1-\omega) R + \omega \tilde{R}_{s,t+1}  \\
	\hat{r}_{t+1} & = & (1-\omega) r + \omega \tilde{r}_{s,t+1} \\
\end{eqnarray*}
which can be approximated by\footnote{To see this, note that $\log 
R^{1-\omega}\tilde{R}_{s,t+1}^{\omega} = (1-\omega)\log R + \omega \log 
\tilde{R}_{s,t+1}$ and recall that $\log R \approx r$, $\log 
\tilde{R}_{s,t+1} \approx \tilde{r}_{s,t+1}$, and $\log \hat{R}_{t+1} \approx 
\hat{r}_{t+1}.$}
\begin{eqnarray*}
	\hat{R}_{t+1} & \approx & R^{1-\omega}\tilde{R}^{\omega}_{t+1}  \\
\end{eqnarray*}

The value-function-maximizing choice of $\omega$ will be the choice 
that minimizes $\gamma$ (recall that $W^{1-\alpha}/(1-\alpha)<0$) 
which will therefore be the $\omega$ that minimizes $E_{t}[ 
\hat{R}_{t+1}^{1-\alpha}]$.
\begin{eqnarray*}
	E_{t}[ \hat{R}_{t+1}^{1-\alpha}] & = & 
	E_{t}[R^{(1-\omega)(1-\alpha)}\tilde{R}_{s,t+1}^{\omega(1-\alpha)}]  \\
	 & = & R^{(1-\omega)(1-\alpha)}E_{t}[\tilde{R}_{s,t+1}^{\omega(1-\alpha)}]  \\
\end{eqnarray*}

But because $\tilde{R}_{s,t+1}$ is distributed lognormally 
\begin{eqnarray*}
	\log E_{t}[\tilde{R}_{s,t+1}^{\omega(1-\alpha)}] & = & E_{t}[\log 
	\tilde{R}_{s,t+1}^{\omega(1-\alpha)}] + \frac{1}{2}\mbox{var}_{t}[\log 
	\tilde{R}_{s,t+1}^{\omega(1-\alpha)}]  \\
	 & = & \omega(1-\alpha) E_{t}[\log \tilde{R}_{s,t+1}] + 
	 \frac{[\omega(1-\alpha)]^{2}}{2}\mbox{var}_{t}[\log 
	\tilde{R}_{s,t+1}]  \\
	 & = & \omega(1-\alpha) r_s + 
	 \frac{[\omega(1-\alpha)]^{2}}{2}\sigma_{r_s}^{2}   \\
\end{eqnarray*}

Note that the choice of $\omega$ that minimizes 
$E_{t}\hat{R}_{t+1}^{1-\alpha}$ will be identical with the choice that 
minimizes its log.  Thus the problem is to find the $\omega$ that 
minimizes
\begin{eqnarray*}
	\log E_{t}\hat{R}_{t+1}^{1-\alpha}  & = & \log [R^{(1-\omega)(1-\alpha)} 
	E_{t}\tilde{R}_{s,t+1}^{\omega(1-\alpha)}] \\
    & = & (1-\omega)(1-\alpha) \log R + \omega(1-\alpha) r_s + 
	 \frac{[\omega(1-\alpha)]^{2}}{2}\sigma_{r_s}^{2}\\
	 & = & (1-\alpha)r + \omega(1-\alpha) 
	 (r_s-r) + \frac{[\omega(1-\alpha)]^{2}}{2}\sigma_{r_s}^{2}  \\
\end{eqnarray*}

The FOC is:
\begin{eqnarray*}
	0 & = & (1-\alpha)(r_s-r) + 
	[\omega(1-\alpha)](1-\alpha)\sigma_{r_s}^{2}  \\
	 & = & (r_s-r) + 
	[\omega(1-\alpha)]\sigma_{r_s}^{2}  \\
	\omega & = & 
	\frac{-(r_s-r)}{(1-\alpha)\sigma_{r_s}^{2}}  \\
	\omega & = & 
	\frac{(r_s-r)}{(\alpha-1)\sigma_{r_s}^{2}}  \\
\end{eqnarray*}

If $\alpha>1$ then this expression has the intuitive implications that 
the share of the portfolio devoted to risky assets increases with the 
equity premium and decreases as consumers get more risk averse 
($\alpha$ rises) or as the riskiness of the risky asset increases 
($\sigma_{r_s}^{2}$ rises).  For moderate values 
of $\alpha$ like 3 and for reasonable values of the variance of 
lognormally distributed returns $\sigma_{r_s}^{2}$ this 
equation implies a value of $\omega>1$.  That is, consumers borrow at 
the riskless rate in order to invest in the risky asset.  For example, 
suppose that the equity premium $r_s-r = .05$ and 
suppose that the standard deviation of annual stock returns is 
$\sigma_{r_s} = .15$ implying $\sigma_{r_s}^{2} = 
.0225$.  Then for $\alpha=3$ this equation yields $\omega = 
-.05/(-2*.0225) = 1.111\ldots$; that is, over 100 percent of the 
portfolio is placed in the risky asset.

\subsection{Portfolio Choice Between a Safe and Two Risky Assets}
\begin{eqnarray*}
	\hat{R}_{t+1} & \approx & R^{(1-\omega_{s}-\omega_{k})}\tilde{R}^{\omega_{s}}_{t+1}\tilde{R}^{\omega_{k}}_{t+1}  \\
\end{eqnarray*}

The value-function-maximizing choice of $(\omega_{s},\omega_{k})$ will 
be the choices that minimize $E_{t}[ \hat{R}_{t+1}^{1-\alpha}]$.
\begin{eqnarray*}
	E_{t}[ \hat{R}_{t+1}^{1-\alpha}] & = & 
	E_{t}[R^{(1-\omega_{s}-\omega_{k})(1-\alpha)}\tilde{R}_{s,t+1}^{\omega_{s}(1-\alpha)}\tilde{R}_{k,t+1}^{\omega_{k}(1-\alpha)}]  \\
	 & = & R^{(1-\omega_{s}-\omega_{k})(1-\alpha)}
     E_{t}[\tilde{R}_{s,t+1}^{\omega_{s}(1-\alpha)}\tilde{R}_{k,t+1}^{\omega_{k}(1-\alpha)}]  \\
\end{eqnarray*}

If we assume that $\tilde{R}_{s,t+1}$ and $\tilde{R}_{k,t+1}$ are 
distributed lognormally and independently of each other, we have

\begin{eqnarray*}
	\log E_{t}[\tilde{R}_{i,t+1}^{\omega_{i}(1-\alpha)}] & = & E_{t}[\log 
	\tilde{R}_{i,t+1}^{\omega_{i}(1-\alpha)}] + \frac{1}{2}\mbox{var}_{t}[\log 
	\tilde{R}_{i,t+1}^{\omega_{i}(1-\alpha)}]  \\
	 & = & \omega_{i}(1-\alpha) E_{t}[\log \tilde{R}_{i,t+1}] + 
	 \frac{[\omega_{i}(1-\alpha)]^{2}}{2}\mbox{var}_{t}[\log 
	\tilde{R}_{i,t+1}]  \\
	 & = & \omega_{i}(1-\alpha) r_s + 
	 \frac{[\omega_{i}(1-\alpha)]^{2}}{2}\sigma_{r_i}^{2}   \\
\end{eqnarray*}
for $i \in \{s,k\}$.

Note that the combination of $(\omega_{s},\omega_{k})$ that minimizes 
$E_{t}\hat{R}_{t+1}^{1-\alpha}$ will be identical with the choice that 
minimizes its log.  Thus the problem is to find the 
$(\omega_{s},\omega_{k})$ that minimize
\begin{eqnarray*}
	\log E_{t}\hat{R}_{t+1}^{1-\alpha} & = & \log [R^{(1-\omega_{s}-\omega_{k})(1-\alpha)} 
	E_{t}\tilde{R}_{s,t+1}^{\omega_{s}(1-\alpha)}\tilde{R}_{k,t+1}^{\omega_{k}(1-\alpha)}] \\
	 & = & (1-\alpha)r 
	 + \omega_{s}(1-\alpha) (r_s-r) + \frac{[\omega_{s}(1-\alpha)]^{2}}{2}\sigma_{r_s}^{2}  
	 + \omega_{k}(1-\alpha) (r_k-r) + \frac{[\omega_{k}(1-\alpha)]^{2}}{2}\sigma_{r_k}^{2}  \\
\end{eqnarray*}

The FOC with respect to $\omega_{i}$ for $i \in \{s,k\}$ is:
\begin{eqnarray*}
	0 & = & (1-\alpha)(r_i-r) + 
	[\omega_{i}(1-\alpha)](1-\alpha)\sigma_{r_i}^{2}  \\
	 & = & (r_i-r) + [\omega_{i}(1-\alpha)]\sigma_{r_i}^{2}  \\
	\omega_{i} & = & \frac{-(r_i-r)}{(1-\alpha)\sigma_{r_i}^{2}}  \\
	\omega_{i} & = & \frac{(r_i-r)}{(\alpha-1)\sigma_{r_i}^{2}}  \\
\end{eqnarray*}

This formula provides a useful way to parameterize the problem, 
because as $W \rightarrow \infty$ the consumers in our model in the 
limit derive all of their utility from wealth, so we should expect 
their behavior in the limit to resemble that of agents like those 
defined in equation ($\ref{eq:wealthutil}$).  If we observe values 
for $\omega_{r_{s}}$ and $\sigma_{r_{s}}^{2}$ we can calculate the 
value of $\alpha$ which would be consistent with those observations:
\begin{eqnarray*}
	\alpha & = & 1 + \frac{r_{s}-r}{\omega_{s}\sigma_{r_{s}}^{2}}
\end{eqnarray*}

Consider first the rich households who are not engaged in 
entrepreneurial activity.  Averaging across our four datasets, 
the observed value of $\omega_{s}$ for these households is roughly 
0.7.  Our baseline assumption about the process for the stock market
yields a value of $\sigma_{r_{s}}^{2} \approx .07$.  This yields an 
estimate of
\begin{eqnarray*}
	\alpha & = & 1 + \frac{.035}{0.7*.07} \\
	 & = & 1.7
\end{eqnarray*}


If take the richest 1/2 percent of the consumers in our datasets as 
reflecting a close approximation to $W = \infty$, the appropriate 
sample values to match are roughly $\omega_{k} = .5$, $\omega_{s} = 
.25$.  This implies that we must make assumptions such that 

\begin{eqnarray*}
	\frac{\omega_{s}}{\omega_{k}} & = & .5  \\
	 & = & \frac{(r_{s}-r)/\sigma_{r_{s}}^{2}}{(r_{k}-r)/\sigma_{r_{k}}^{2}} \\
	 & = & \left(\frac{r_{s}-r}{r_{k}-r}\right)\left(\frac{\sigma_{r_{k}}^{2}}{\sigma_{r_{s}}^{2}}\right)
\end{eqnarray*}


\end{document}



Furthermore, casual scrutiny of the {\it Forbes 400} list reveals that 
most of the wealthiest people in the world made their wealth primarily 
or exclusively from one enterprise or a set of closely connected 
enterprises, so it seems reasonable to impose.

Quadrini form:

In each period there is some probability that a given household will 
have the opportunity to enter the business sector.  However, operating 
a business requires the payment of a fixed cost $f$ every period.  

$s_{t}$  savings from last period outside of the business 
$h_{t}$  capital inside the business 


\begin{eqnarray*}
x_{t+1} & = & \hat{R}_{t+1} s_{t} + (\alpha_{t+1} h_{t} - f) + y_{t+1} \\
h_{t+1} & = & h_{t}+(x_{t}-s_{t}-c_{t})
\end{eqnarray*}

\begin{eqnarray*}
s_{t} & \geq & 0 \\
h_{t} & \geq & 0 
\end{eqnarray*}


\begin{eqnarray*}
	V_{t}(X_{t}) & = & \max_{\{C_{t},K_{t}\}} u(C_{t}) + \beta E_{t} 
	V_{t+1}(X_{t+1})  \\
	 & s.t. &   \\
	X_{t+1} & = & R_{t+1}^{s}W_{t} + R_{t+1}^{k}Q_{t+1} \Omega[\frac{K_{t}}{W_{t}+K_{t}}]K_{t}  \\
\end{eqnarray*}

\begin{eqnarray*}
	V_{t}(X_{t}) & = & \max_{\{C_{t},K_{t}\}} u(C_{t}) + \beta E_{t} 
	V_{t+1}\left(R_{t+1}^{s}(X_{t}-C_{t}-K_{t}) + 
	R_{t+1}^{k}Q_{t+1} \Omega[\frac{X_{t}-C_{t}-K_{t}}{K_{t}}]K_{t}\right)  \\
\end{eqnarray*}


All period-$t$ dated uncertainty is resolved at the beginning of the 
period, before the consumer makes that period's consumption and 
investment choices.  Thus the consumer learns the value of 
$\zeta_{t}$, the transitory shock to labor income, $\eta_{t}$ the 
permanent shock to labor income, and the values for $R_{t}^{s}$, the 
gross return on stocks purchased in the previous period, and 
$R_{t}^{k}$, the gross return on the entrepreneurial project (if any) 
in which the household invested in the previous period.  We assume 
that the net return on the entrepreneurial project is given by 
$R_{t}^{k}Q_{t+1} \Omega(\omega_{t-1}^{k})$ where $\Omega$ is a function that 
scales down the return on the entrepreneurial project to an extent 
determined by the fraction of the household's portfolio that is 
invested in the project.  If the entire portfolio is invested in the 
project ($\omega_{t-1}^{k}=1$) then $\Omega(1)=1$ and the household 
reaps the full return.  If a smaller fraction of the portfolio is 
invested, the return is proportionately lower.

In order to start an entrepreneurial venture, a household who is not 
currently engaged in an entrepreneurial project must have an `idea,' 
an event that occurs randomly with probability $p=.05$ at the 
beginning of each year.  The household then must choose whether to 
implement the idea by investing a positive fraction of its portfolio 
in the entrepreneurial project, or to remain a worker.  We 
the presence or absence of 


The expected value of wealth 
that is not consumed in the present period is given by the discounted 
value if the consumer is alive in the next period (and event with 
probability $(1-d_{t+1})$ plus the value they attach to the bequest they 
will leave behind in the event of death.  Denote total household 
resources at the beginning of period $t$ by $X_{t}$, household 
noncapital income by $Y_{t}$, the rate of return on stocks as 
$R_{t}^{s}$ and the share of the household portfolio in stocks is 
designated $\omega_{t}^{s}$.  The return on the entrepreneurial 
project depends on the proportion of the household portfolio invested 
in the project.  The raw return is denoted $R_{t}^{k}$, but this is 
multiplied by a function $\Omega()$




get utility from current consumption and from the 
expectation of future consumption (if they live 

At the beginning of the period, there is a small probability $p=.05$ 
that the household will have an entrepreneurial `idea.'  
Implementing the `idea' 

Non-entrepreneurial households solve the problem:


amount of the entrepreneurial project to the value associated with 
choosing $\omega_{t}^{k}=0$.


\subsubsection{First Order Conditions}

The FOC wrt $C_{t}$ is:

\begin{eqnarray*}
	0 & = & u'(C_{t}) + \beta E_{t}\left[ \frac{\partial X_{t+1}}{\partial 
	C_{t}} \left( (1-d_{t+1})
	V_{t+1}^{x}(X_{t+1},P_{t+1},Q_{t+1}) + d_{t+1}B'(W_{t}) \right) \right]\\
	u'(C_{t}) & = & \beta E_{t} \left[\hat{R}_{t+1} \left((1-d_{t+1}) 
	V_{t+1}^{x}(X_{t+1},P_{t+1},Q_{t+1}) + d_{t+1}B'(W_{t})\right)\right] \\
	C_{t}^{-\rho} & = & \beta E_{t} \left[\hat{R}_{t+1} \left((1-d_{t+1}) 
	V_{t+1}^{x}(X_{t+1},P_{t+1},Q_{t+1}) + d_{t+1}(X_{t+1}+\lambda)^{-\alpha}\right)\right] \\
	C_{t} & = & \left(\beta E_{t} \left[\hat{R}_{t+1} \left((1-d_{t+1}) 
	V_{t+1}^{x}(X_{t+1},P_{t+1},Q_{t+1}) + d_{t+1}
	(X_{t+1}+\lambda)^{-\alpha}\right)\right]\right)^{1/-\rho} \\
\end{eqnarray*}

The FOC wrt $\omega_{t}^{s}$ is:

\begin{eqnarray*}
	0 & = & \beta E_{t}\left[ \frac{\partial X_{t+1}}{\partial 
	\omega^{s}_{t}} \left( (1-d_{t+1})
	V_{t+1}^{x}(X_{t+1},P_{t+1},Q_{t+1}) + d_{t+1}B'(W_{t}) \right) \right]\\
	0 & = & \beta E_{t} \left[(R - R^{s}_{t+1})(W_{t}) \left((1-d_{t+1}) 
	V_{t+1}^{x}(X_{t+1},P_{t+1},Q_{t+1}) + d_{t+1}B'(W_{t})\right)\right]
\end{eqnarray*}

If $Q_{t}=0$ there is no FOC wrt $\omega_{t}^{k}$.  If $Q_{t} > 0$ then 
the FOC wrt $\omega_{t}^{k}$ is:

\begin{eqnarray*}
	0 & = & \beta E_{t}\left[ \frac{\partial X_{t+1}}{\partial 
	\omega^{k}_{t}} \left( (1-d_{t+1})
	V_{t+1}^{x}(X_{t+1},P_{t+1},Q_{t+1}) + d_{t+1}B'(W_{t}) \right) \right]\\
      & = & \beta E_{t} \left[(R - 
	R^{k}_{t+1}(Q_{t+1} \Omega(\omega_{t}^{k})+Q_{t+1} \Omega^{'}(\omega^{k}_{t})\omega_{t}^{k}))(W_{t}) \left((1-d_{t+1}) 
	V_{t+1}^{x}(X_{t+1},P_{t+1},Q_{t+1}) + d_{t+1}B'(W_{t})\right)\right]
\end{eqnarray*}

The consumer decides whether or not to engage in the project by 
comparing the value associated with engaging in the optimal positive 


	0 & = & \Omega_{t}^{s}(W_{t},P_{t},Q_{t},\omega_{t}^{k},\omega_{t}^{s})  \\
      & = & \beta E_{t} \left[(R(1-\omega_{t}^{k}) - 
	R^{k}_{t+1}Q_{t+1}[ 
	\Omega(\omega_{t}^{s})+\Omega^{'}(\omega^{k}_{t})\omega_{t}^{s}])W_{t} \left((1-d_{t+1}) 
	V_{t+1}^{x}(X_{t+1},P_{t+1},Q_{t+1}) + d_{t+1}B'(W_{t})\right)\right]



The quality of the `idea' is indicated by the indicator variable 
$Q_{t} \in [0,1,2]$.  $Q_{t} = 0$ indicates a household without an 
entrepreneurial idea, $Q_{t} = 1$ indicates a household with a `new' 
entrepreneurial idea, and $Q_{t} = 2$ indicates a household with a 
`mature' entrepreneurial idea.  

Quadrini~\cite{quadrini:entrepreuneurship} shows that the exit 
rate from entrepreneurship declines dramatically as tenure of the 
entrepreneur in the business increases.  Roughly 40 percent of 
entrepreneurial projects fail in the first year, but the failure rate 
of projects which have survived for three years or longer is only 
about 10 percent.  We capture this phenomenon by making the stochastic 
rate of return on entreprenurial projects $R_{t+1}^{k}$ a function of 
$Q_{t}$.  For `new' projects there is a 40 percent chance that 
$R_{t+1}^{k} = 0$ while for `mature' projects the probability of 
failure is only 10 percent.  

