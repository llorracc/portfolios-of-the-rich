%&LaTeX
%&custom12pt


\documentclass[12pt]{article}\usepackage{Portable:latex:mathfig:geompsfi,epsfig,amsmath,latexsym,ifthen,Portable:latex:vmargin,afterpage,Portable:latex:harvard_mod,Endnotes}

\input Portable:latex:boxedeps.tex
\SetOzTeXEPSFSpecial 
\HideDisplacementBoxes%\ShowDisplacementBoxes %%alternatives

\setpapersize{USletter}

%\input setmargins.tex


\newboolean{Figures}
\setboolean{Figures}{true}

\newboolean{Tables}
\setboolean{Tables}{true}

\newboolean{ShowTrueEqns}        % Solution procedure differs from text notation
\setboolean{ShowTrueEqns}{false} % If true, this prints the formulas associated w. true solution method

\newboolean{ShowFirstStuff}
\setboolean{ShowFirstStuff}{true}

\newboolean{TilburgVersion}
\setboolean{TilburgVersion}{true}

\newboolean{NBERWPVersion}
\setboolean{NBERWPVersion}{false}

\newboolean{ModelsVersion}
\setboolean{ModelsVersion}{false}

\newboolean{BookVersion}
\setboolean{BookVersion}{true}

\newboolean{PrintVersion}
\setboolean{PrintVersion}{true}

\ifthenelse{\boolean{PrintVersion}}{
\setmarginsrb{1.2in}{1.4in}{1.2in}{1.0in}{0pt}{0pt}{0pt}{.5in}
% {left}{top}{right}{bottom}
}{}

\ifthenelse{\boolean{BookVersion}}{
\setmarginsrb{1.5in}{1.5in}{1.5in}{1.5in}{0pt}{0pt}{0pt}{.5in}
% {left}{top}{right}{bottom}
}{}

\newcounter{chapter}
\setcounter{chapter}{10}


\renewcommand{\thefigure}{\arabic{chapter}.\arabic{figure}} 
\renewcommand{\thetable}{\arabic{chapter}.\arabic{table}} 



%\pagestyle{empty}

\begin{document}




%\baselineskip 18pt

%\ifthenelse{\boolean{ShowFirstStuff}}{


\begin{titlepage}


\ifthenelse{\boolean{NBERWPVersion}}{\hfill \tiny NBER WP Version}{}

\ifthenelse{\boolean{BookVersion}}{\hfill \tiny Book Version}{}

\vspace{1.5in}


{\centerline {\LARGE Portfolios of the Rich}}
\medskip\medskip

\vspace{.5in}

\normalsize

\centerline{\large Christopher D. Carroll$^\dagger$}
\centerline{ccarroll@jhu.edu}
\medskip

\vspace{.2in}

\centerline{\today}

\vspace{1.5in}

\noindent {\bf Keywords:} portfolios, risk aversion, entrepreneurship, 
capital market imperfections, bequests

\medskip\medskip
\noindent {\bf JEL Codes:} D10, D31, D91, E21, G11

\vspace{.2in}

\medskip\medskip

\small $^\dagger$ NBER and The Johns Hopkins University.  
Correspondence to Christopher Carroll, Department of Economics, Johns 
Hopkins University, Baltimore, MD 21218-2685 or ccarroll@jhu.edu.  

\medskip\medskip 

\small 

\ifthenelse{\boolean{BookVersion}}{}{The original version of this 
paper was prepared for the conference ``Household Portfolios'' held at 
the European University Institute on December 11-12 1999, and this 
paper will be included in the corresponding conference volume with the 
same title, edited by Luigi Guiso, Michael Haliassos, and Tullio 
Jappelli and to be published by MIT Press.  }I am grateful to Kevin 
Moore for excellent research assistance, and to Marco Pagano and other 
participants in the conference for numerous useful suggestions and 
ideas.  An archive of all of the programs and data used to produce the 
tables, with extensive instructions, can be found at my website, 
http://www.econ.jhu.edu/people/ccarroll/carroll.html.


\end{titlepage}

\ifthenelse{\boolean{BookVersion}}{

% Double-space the footnotes/endnotes
\addtolength{\footnotesep}{5pt}

% Double-space the paper

\renewcommand{\baselinestretch}{2.0} \normalsize

%\usepackage{Portable:latex:Endnotes_doublespace,Portable:latex:aertt}
%\usepackage{Portable:latex:aer_cdc} % Custom style file that conforms to AER requirements



% The AER wants endnotes rather than footnotes, so
% translate footnote commands to endnote commands
\let\footnote=\endnote

% Double-space the footnotes/endnotes
\addtolength{\footnotesep}{5pt}

% Double-space the paper
%\baselineskip 32pt
}




\centerline{\bf Abstract}\medskip\medskip
%}{} % End ifthenelse{\boolean{ShowFirstStuff}}
Recent research has shown that `rich' households save at much higher 
rates than others (see Carroll~\cite{carroll:richsave}; Dynan, 
Skinner, and Zeldes~\cite{dsz:richsave}; Gentry and 
Hubbard~\cite{gentry&hubbard:wealthysave}; 
Huggett~\cite{huggett:wealth}; 
Quadrini~\cite{quadrini:entrepreneurship}).  This paper documents 
another large difference between the rich and the rest of the 
population: portfolios of the rich are heavily skewed toward risky 
assets, particularly investments in their own privately held 
businesses.  The paper explores three possible explanations of these 
facts.  First, perhaps there is exogenous variation in risk tolerance, 
so that highly risk tolerant households engage in high-risk, 
high-return activities, and the risk-lovers who are lucky constitute 
the rich.  A second possibility is that capital market imperfections 
{\it a la} Gentry and Hubbard~\cite{gentry&hubbard:wealthysave} and 
Quadrini~\cite{quadrini:entrepreneurship} require entrepreneurial 
activities to be largely self-financed, and these same imperfections 
imply that entreprenurial investment will yield high average returns.  
The final possibility is that wealth enters households' utility 
functions directly as a luxury good as in 
Carroll~\cite{carroll:richsave} (one interpretation is that this 
reflects the utility of anticipated bequests), implying that risk 
aversion declines as wealth rises.  The paper concludes that the 
overall pattern of facts suggests both Carroll-style utility and 
Gentry/Hubbard-Quadrini style capital market imperfections are 
important.

%\ifthenelse{\boolean{NBERWPVersion}}{\baselineskip 22pt}{}

\ifthenelse{\boolean{TilburgVersion}}{ 

}{
This paper presents data on the portfolio structure of rich households 
and of typical households, and compares the results to the predictions 
of a dynamic stochastic optimization model of household portfolio 
choice which nests the two hypotheses about the reasons for the 
higher saving rate of the rich.  The results lead to three main 
conclusions: 1) the ability of some `entrepreneurial' households to 
earn a higher rate of return on their investments is not sufficient by 
itself to explain the patterns in the data, because that model implies 
that even households who can earn a high rate of return will try to 
consume all of their wealth before they die, and there is no evidence 
of decumulation at old ages by the very rich; 2) the heterogeneity in 
saving rates at different levels of permanent income induced by 
assuming that bequests are luxury goods is also insufficient, taken 
alone, to generate the degree of wealth concentration observed at the 
top of the distribution, because the rates of return on publicly 
traded assets are not high enough to generate the extraordinary 
concentrations of wealth observed at the very top of the wealth 
distribution; 3) a model which combines the two assumptions is 
consistent with the data, because the high rate of return on 
entrepreneurial investments allows the accumulation of massive amounts 
of wealth while the `luxury' nature of bequests implies there is no 
reason to decumulate that wealth before death.
}
\vfill\eject
\clearpage

\clearpage



\section{Introduction}

Ever since the pathbreaking work of Pareto more than a century ago, 
economists have known that wealth is extremely unevenly distributed.  
More recently, survey data have revealed that portfolio structures are 
also very different for households with different levels of wealth.  
While the portfolios of the rich are complex, the portfolio of 
financial and real assets of the median household (at least in the 
U.S.) is rather simple: a checking/savings account plus a home and 
mortgage, and not much else.\footnote{Bertaut and 
Starr-McCluer~\cite{b&s-m:usportfolios} find that the only kind of 
financial asset held by more than half of US households is a 
checking/transactions account.} Overwhelmingly, the data tell us that 
if we wish to understand aggregate portfolio behavior, it is critical 
to understand the behavior of the richest few percent of households, 
both because they control the bulk of aggregate wealth and because 
their portfolio behavior is much more complex than that of the typical 
household.

Though the foregoing arguments may seem to provide a compelling 
rationale for studying the portfolios of the rich, there has been 
little recent academic work in this area.  The goal of this paper is 
to provide a summary of the basic facts about portfolios of wealthy 
households in the U.S. (and how the facts have changed over time) in a 
form which allows comparison of their behavior both with the 
rest of the population in the U.S. and with portfolio behavior among 
other groups and other countries\ifthenelse{\boolean{BookVersion}}{ 
surveyed in this volume}{surveyed in the recent volume {\it Household 
Portfolios} edited by Guiso, Haliassos, and 
Jappelli~\cite{ghj:portfolios}}, and to make a preliminary attempt to 
understand the characteristics that will be required of any model 
which hopes to be consistent with the observed behavior.

The principal conclusion will be that the most important way in which 
the portfolios of the rich differ from those of the rest is that the 
rich hold a much higher proportion of their portfolios in risky 
investments, with a particularly large concentration of net worth in 
their own entrepreneurial ventures.

After the empirical conclusions are presented, the paper informally 
considers how these results relate to theoretical models of portfolio 
behavior.  The starting point will be a standard stochastic version of 
the Life Cycle/Permanent Income Hypothesis model.  That model will 
prove inadequate, however, because it implies that the rich should 
look like scaled-up versions of everybody else.  They should have 
neither the extreme wealth-to-income ratios observed in the data, nor 
the unusual portfolio structures.  The goal of the theoretical 
discussion will be to consider whether any of three potential 
modifications to the standard model might explain the observed 
combination of facts.

The first idea is that perhaps there is exogenous, immutable {\it ex 
ante} variation in risk aversion across households.\footnote{By {\it 
ex ante} we mean a preference difference which exists prior to any 
saving or portfolio choice decision the household makes, and which is 
unaffected by the outcomes of such choices.} In that case more 
risk-tolerant households would take greater risks and on average would 
earn higher returns.  If owning a private business is the form of 
economic activity that offers the highest risk and highest return, one 
might expect that the most risk tolerant households would gravitate 
toward entrepreneurship, and on average would end up richer (though 
the failures might end up poorer).\footnote{Surprisingly, it is not 
clear that classical theory supports the proposition that less risk 
averse individuals will invest a higher proportion of their risky 
investments in the most-risky activities.  See 
Gollier~\cite{gollier:classical} for a discussion of the `mutual fund 
separation theorem' which implies that the composition of risky assets 
should be similar whatever the level of risky asset holdings.  We will 
assume that this reflects a limitation of classical theory, rather 
than a plausible description of behavior.} The paper will argue that 
this story has several defects, ranging from the fact that the 
empirical evidence fails to find a correlation between wealth growth 
and initial (expressed) risk aversion to the fact that, taken alone, 
the story provides neither an explanation for the lack of 
diversification of entrepreneurial investments nor for the tendency of 
wealthy households to hold much of their net worth in {\it their own} 
entrepreneurial ventures.

These points lead to the second possibility: that the observed 
patterns are entirely a consequence of capital market imperfections, 
as suggested recently by Gentry and 
Hubbard~\cite{gentry&hubbard:wealthysave} and 
Quadrini~\cite{quadrini:entrepreneurship}.  Those authors argue that 
adverse selection and moral hazard problems require entrepreneurial 
enterprises to be largely self-financed.  They further assume that 
there is a minimum efficient scale for private enterprises and that 
this minimum scale is large relative to the wealth of the typical 
household.  The combination of these two assumptions can explain why 
households with low or moderate wealth or income are less likely to 
become entrepreneurs.  Furthermore, this story requires no differences 
in tastes among members of the population, and in principle can 
explain both the high saving rates of the rich and the high portfolio 
shares in their own entrepreneurial ventures.  However, this story too 
has problems.  The first is that, in the absence of differences in 
preferences between the rich and the rest, the standard model implies 
that those households who have invested heavily in their own 
entrepreneurial ventures should try to balance the riskiness of these 
investments by holding all other assets in very safe forms.  Instead, 
the non-entrepreneurial investments of rich entrepreneurs are 
much riskier than the portfolios of nonrich nonentrepreneurs.  A 
second problem with this story is that even the model with imperfect 
capital markets implies that as the rich get old, they eventually 
begin running down their wealth.  In contrast, empirical data reveal 
no evidence that wealthy elderly households ever begin to run 
down their wealth.

The final possibility is that the model's assumption about the 
household utility function needs to be changed in a manner similar to 
that proposed by Carroll~\cite{carroll:richsave}, who simply assumes 
that wealth enters the utility function as a luxury good in a modified 
Stone-Geary form.  Because Max Weber~\cite{weber:capitalism} argued 
that a love of wealth for its own sake is the spirit of capitalism, 
Bakshi and Chen~\cite{bakshi&chen:spirit} and Zou~\cite{zou:spirit} 
have dubbed such models `capitalist spirit' models.  
Carroll~\cite{carroll:richsave} proposed this modification to the 
standard model as a way to explain the high lifetime saving rates of 
the rich, and argued that many different kinds of behavior, ranging 
from philanthropic bequest motives to pure greed, would result in a 
formulation of saving behavior that would be well captured by the 
modified model.  An unanticipated consequence of the model is that it 
implies that rich households have lower relative risk aversion than 
the nonrich, which in turn could explain why the rich hold riskier 
portfolios than the rest, and why high-wealth or high-income young 
households are more likely to begin entrepreneurial ventures.

The one feature of the data that the `capitalist spirit' model taken 
alone cannot explain is the tendency of entrepreneurs to invest 
largely in {\it their own} entrepeneurial ventures, which appears to 
require some form of capital market imperfection.  The paper thus 
concludes that the main features of the data can probably be explained 
in a model which combines capital market imperfections of the kind 
emphasized by Gentry and Hubbard~\cite{gentry&hubbard:wealthysave} and 
Quadrini~\cite{quadrini:entrepreneurship} with a utility function like 
that postulated in Carroll~\cite{carroll:richsave}.

\section{The Data}


\subsection{Portfolios of the Rich}

U.S. survey data on the portfolios of the rich are the best in the 
world.  The 1962-63 {\it Survey of Financial Characteristics of 
Consumers} (henceforth SFCC) was the first wealth survey to heavily 
oversample the richest households.  The next comprehensive wealth 
survey was the 1983 {\it Survey of Consumer Finances}, which was 
followed by a 1989 SCF which consisted of a subsample of reinterviewed 
households from the 1983 survey along with a fresh batch of new 
households.  Since 1989 the SCF has been performed triennially (though 
with no further panel elements), with the latest survey having been 
completed in 1998.

The availability of data spanning such a long time period opens up the 
possibility of studying how portfolios change in response to changes in 
the economic enviornment.  Before examining the data on portfolio 
structure, therefore, we first present a summary of the taxation and 
legal changes that we might expect to have had a substantial impact on 
portfolio structure of wealthy households.

\subsubsection{The Tax Environment}
Table \ref{table:laws} summarizes the changes over time in the three 
aspects of US taxes that are particularly important for the rich.  
(For information on broader changes in the US tax code see the paper 
by 
Poterba~\cite{poterba:taxportfolios}\ifthenelse{\boolean{BookVersion}}{ 
in this volume}{}).  The first two columns show the statutory top 
marginal federal tax rate, which declined from 91 percent in 1963 to 
39.6 percent in 1993 and thereafter.  The second column shows the 
actual taxes paid as a proportion of their incomes by the richest one 
percent of households.  In spite of the dramatic decline in top 
marginal rates, the proportion of income paid in taxes has been fairly 
steady, varying between around 20 and 25 percent over the entire 
period.  This reflects the fact that during the era of high top 
marginal rates, the tax code was riddled with tax shelters and 
loopholes that made it possible for almost all rich people to avoid 
paying the confiscatory top marginal rates on the statute books.

The estate tax is also highly relevant for the rich.  The structure of 
the estate tax is rather complex, but that structure remained largely 
the same over the period in question.  The first \$x of an estate is 
free from estate taxation altogether, where \$x is indicated by the 
column of the table labelled `exemption.'  Above \$x, taxes begin at a 
marginal rate of $y$ percent and peak at a top marginal rate of $z$ 
percent, where $y$ and $z$ are the first and second numbers in the 
column labelled `tax range.'  

The exclusion for closely held businesses is a mechanism that reduces 
the reported amount of the value of a closely-held business that is 
taxable, under the condition that the heir plans to `actively manage' 
the business rather than sell it.  The marital deduction indicates how 
much of the estate is taxed when one spouse dies and the estate falls 
into the hands of the widow or widower.  The 100 percent deduction 
since 1985 means that estates are taxed only when both members of a 
married couple have died.

The final kind of tax that is relevant to the rich is the gift tax 
exclusion amount \$g, whose value is reported in the last column of 
the table.  This is the amount that each member of the household 
(husband and wife) can give to any individual (son, daughter, 
son-in-law, daughter-in-law, grandchildren, etc.)  annually without 
incurring any additional taxes for the recipient or donor.

The table shows that there have been two big changes in the taxes 
specifically relevant for the rich over the period in question: the 
large increase in exemption levels for the estate tax in the early 
1980s, and the more gradual, but cumulatively very large, decline in 
top marginal rates.  The most important change not captured in the 
table is probably the abrupt termination of a variety of tax shelters 
in the 1986 tax reform.

A final feature of the tax code that is relevant for the rich is the 
`step-up in basis at death.'  The capital gains tax `basis' for an 
asset is normally defined as the nominal price at which the asset was 
bought.  However, if the asset has been inherited, then the basis is 
the nominal valuation of the asset at the time it was inherited.  The 
step up in basis at death provides an incentive for individuals who 
anticipate leaving a bequest whose value is less than the exemption 
amount to hold their assets in forms which yield returns 
disproportionately in the form of capital gains, since capital gains 
that happen before death are untaxed.  (Incentives for the very rich 
to hold their assets in forms which yield mainly capital gains are 
smaller because the capital gains do contribute to the valuation of 
the estate for tax purposes and thus are marginally taxed at the 
marginal estate tax rate for those who will leave bequests in excess 
of the exemption amount).

Implications of the tax system for the portfolio structure of the rich 
are not always easy to determine by examining statutory provisions.  
For example, the incentive provided by the `step up in basis at death' 
to hold assets in forms that yield capital gains depends importantly 
on the effective marginal rate of taxation on other forms of capital 
income, which (as discussed above) is not very well proxied by the 
statutory top marginal rate.  The exclusion for closely held 
businesses does provide an incentive to hold at least a limited 
absolute amount of the portfolio in the form of closely-held 
businesses {\it if} the individual expects his or her heirs to 
continue to run the business.  However, no marginal incentive to 
further business ownership is provided once the total amount of wealth 
held in this form exceeds the exclusion amount.  For a more detailed 
historical analysis of tax policies relevant for the rich in the 
postwar period, see Brownlee~\cite{brownlee:historical}.


\subsubsection{Detailed Portfolio Structure}
Our statistical summary of the portfolio structure of the rich begins 
with Table \ref{table:pctown}, which provides data on the proportion 
of the rich (defined here and henceforth as the top one percent of 
households by net worth) who own any amount of various kinds of 
assets.

Perhaps the most dramatic change over time in the table is the sharp 
increase in the proportion of households with defined contribution 
pension plans.  In the 1962-63 SFCC, only 10.1 percent of the rich had 
any such account, but by 1983 the fraction had already jumped to 65.6, 
while by 1995 the fraction had reached 78.6 percent.  The low 
percentage in 1962-63 reflects the fact that there was 
little tax advantage to such plans until the early 1980s, when 
Individual Retirement Accounts (IRAs) suddenly became available in 
principle to the whole population, and eligibility for company-based 
401(k) pension plans was greatly expanded.  What is interesting is the 
speed with which rich households availed themselves of these new 
options.  In contrast, Bertaut and 
Starr-McCluer~\cite{b&s-m:usportfolios}\ifthenelse{\boolean{BookVersion}}{ 
in this volume}{} show (Table 3) that only 31 percent of all 
households had acquired such accounts by the time of the 1983 survey.

Another notable change is that the proportion holding individual stock 
shares directly has fallen from 84.0 percent in 1962 to 65.0 percent 
in 1995, while the proportion holding mutual funds has risen from 
about 24 percent to about 45 percent.  This reflects a broad pattern 
in which households have increasingly decided to hold shares in the 
form of mutual funds rather than individual stocks.  This pattern has 
not been much studied by economists, although it is interesting 
because it reflects a convergence of actual behavior toward portfolio 
theory's recommendation for diversification.

Among the other categories of assets, the largest changes are seen in 
the holdings of `other bonds' (primarily corporate bonds), which 
declined very sharply between 1962 and 1983 and fluctuated 
substantially between 1983 and 1995.  Because nominal interest income 
is taxable annually while capital gains are taxable only upon 
realization, the sharp increase in nominal interest rates caused by 
the acceleration of inflation in the 1960s and 1970s could explain a 
shift out of interest-bearing assets between the 1963 and 1983.  
However, there is no obvious tax reason for the fluctuations between 
1983 and 1995.

The proportion of the richest households who have equity in a 
privately held business has fluctuated substantially over the years, 
from a low of 69.0 percent in 1962-63 to a high of 88.0 percent in 
1983.  To some extent, fluctuations in this variable may reflect
stock market valuations, because after a large increase in stock
prices a higher proportion of the wealthy will be rich because of
their stock holdings compared with the proportion who are rich
because of their holdings of other kinds of assets.  (The 1983 SCF
was conducted before the bull markets of the 80s and 90s had
boosted stock valuations.)

With respect to debt holdings, the proportion of rich households with 
any debt jumped sharply between the 1962-63 SFCC, when it was 50.2
percent, and the 1983 SCF, when it was 77.9 percent, but exhibited no 
clear trend thereafter.  Among debt categories, the most striking 
change is the increase in the proportion of households with mortgage 
debt, from 30.7 percent in 1962 to 52.5 percent in 1995.  This likely 
reflects the fact that mortgage interest remained tax deductible after 
the 1986 tax reform while other forms of debt lost their deductible 
status.

On the whole, the striking feature of this table is that the 
proportion of rich households owning various categories of assets has 
not changed greatly for most categories of assets - particularly 
considering that small sample sizes mean that there is inevitably some 
measurement error in the statistics for any particular year.\footnote{ 
One exception is `other financial assets,' which had an 89.3 percent 
owernship rate in the 1962-63 SFCC but much lower rates in the later 
surveys.  This is almost certainly because holdings of cash were 
included in this grab-bag category in the SFCC but not in the SCF's.  
In any case, the next table shows that `other finanical assets' 
constitute a trivial proportion of net worth in all surveys.}

Another useful comparison is of the rich to the rest of the 
population.  Average values of ownership shares for the nonrich over 
the five survey years are presented in the last column of the table.  
The broadest observation to make here is that rich households are more 
likely to own virtually every kind of asset.  Particularly striking is 
the discrepancy in the proportion owning equity in a privately held 
business, which averages about 75 percent for the rich but only 13 
percent for the rest of the population.  The contrast in ownership of 
shares in publicly traded companies is only slightly less dramatic: 74 
percent versus 16 percent.

Table \ref{table:compnw} examines the relative weight of various kinds 
of assets in the net worth of the richest households.  The table shows 
that the shift in value from stocks to mutual funds was substantial, 
but even at the end of the sample in 1995, total net worth in 
individual shares still remained substantially greater than that in 
mutual funds.  One of the largest shifts over time is in the role of 
investment real estate, which jumps from 7.4 percent of net worth in 
1962-63 to over 20 percent in 1983.  Investment real estate continues 
to constitute more than 20 percent of the portfolio until 1995, when 
its share drops to 13.1 percent.  The jump in investment real estate 
between the early 1960s and the early 1980s may reflect the prominent 
role of real estate in tax shelters until the tax reform act of 1986.  
One would have expected a decline in the value of investment real 
estate following the repeal of many of these tax shelters in the 1986 
tax act, so it is surprising that no decline is manifest until 1995.

Another interesting observation from the table is the small amount of 
mortgage debt (only 1.1 percent of net worth on average) despite the 
fact that more than half of the rich have positive amounts of such 
debt.

Comparing the rich to the rest of the population, again perhaps the
most important difference is the importance of business equity for the
rich.  Such wealth accounts for about 40 percent of total net worth 
of the rich in 1983 and thereafter, vastly more than its share in 
the net worth of the typical household.  Other differences include
the lower total indebtedness of the rich and the much smaller 
proportion of total wealth tied up in home equity.

\subsubsection{Portfolio Structure And Portfolio Theory}

The usual theoretical analysis of portfolio allocation considers the 
optimal proportion of net worth to invest in `risky' versus `safe' 
assets.  This stylized theoretical treatment is conceptually useful 
but difficult to bring to data, because it is hard to allocate every 
asset to one of these two categories.  Table~\ref{table:riskysafe} 
reflects an effort to find a compromise between the complexity of 
actual portfolios and the simplicity of theory.

Among financial assets, there are some that are clearly safe (like 
checking, saving, and money-market accounts) and some that are clearly 
risky (like stock shares).  But other assets are harder to allocate, 
either because the item itself has an ambiguous status (like long-term 
government bonds, which are subject to inflation risk but not 
repayment risk (we hope!)) or because the asset is a composite with 
unknown proportions of risky and safe assets (like mutual funds which 
hold both stocks and government bonds).  
\ifthenelse{\boolean{BookVersion}}{ Consistent with the other papers 
in this volume, w}{W}e have allocated all financial assets to one of 
three categories: Clearly safe, fairly safe, and risky, which can of 
course be further aggregated into broad measures of safe and risky 
assets.  We have divided nonfinancial assets into the primary 
residence, investment real estate, business equity, vehicles, and 
`other.'

With these definitions, we can construct three definitions of risky 
assets: A `narrow' definition, which includes only risky financial 
assets; a `broad' definition, which includes clearly and fairly risky 
financial assets, business equity, and investment real estate; and a 
`broadest' definition which adds even the `fairly safe' assets.

It is apparent from the table that the portfolios of the rich are 
dramatically more risky than those of the rest of the 
population.\footnote{One might wonder whether the differences in risky 
shares partly reflect age differences between the rich and the rest.  
However, when the age range for the rich and the rest is restricted to 
households aged 35-54, the divergence between the rich and the rest 
is, if anything, even greater.  For example, the portfolio share of 
private business for the age 35-54 rich is 47.6 versus 17.9 for the 
age 35-54 nonrich - a greater discrepancy than the 37.7 versus 14.8 
figures in Table~\ref{table:riskysafe}.} Across the five surveys the 
proportion of their portfolios that consisted of broadly risky assets 
was about 80 percent, compared with an average percentage of only 40 
percent for the nonrich households.  Examining the data in more detail 
reveals two key differences between the rich and the rest: the rich 
hold a much smaller proportion of their wealth in home 
equity\footnote{Home equity is calculated as the value of primary 
residence minus mortgage debt.} (7.4 percent versus 49.6 percent) and a 
much larger proportion in business equity and investment real estate 
(the sum of these two categories is 52.1 percent for the rich versus 
26.2 percent for the rest).

\subsubsection{Portfolio Diversification and Age Structure}

Another perspective on the portfolios of the rich is presented in 
Table \ref{table:divers}, which provides a census of the portfolio 
structure of the rich along the three dimensions corresponding to 
ownership or non-ownership of clearly safe, fairly safe, and risky 
assets, a total of $2^{3}=8$ different possibilities.  In all five 
survey years, a majority or nearly a majority of the rich held some 
assets in each of these three categories.  This is a sharp contrast to 
the behavior of the rest of the population, which is much more evenly 
distributed among the 8 categories but is most heavily concentrated in 
the region with only safe assets.  (See Bertaut and 
Starr-McCluer~\cite{b&s-m:usportfolios} for the data on the rest of 
the population.)

Finally, Table \ref{table:riskybyage} presents data on ownership rates 
for risky assets by age of the household head for each of the survey 
years.\footnote{Portfolio shares are for the whole population of the 
rich, not just for those who own risky assets, i.e.  the numbers are 
not conditional on participation.} Interestingly, the patterns for 
ownership rates and for portfolio shares are different: The 
probability of owning at least some amount of risky assets is 
monotonically increasing in age, but the {\it proportion} of the 
portfolio composed of `broad risky' assets rises through the first 
three age categories (up to age 49) but exhibits no clear pattern 
across the older age groups.\footnote{It is important to recall that 
these figures may reflect the effects of both cohort and time effects 
as well as age effects, so the true age effects may differ from the 
reported numbers.} Ownership rates of `risky' assets show a similar 
monotonic increase (at least until age 70+), while the portfolio share 
shows some tendency to decline with age.  As King and 
Leape~\cite{king&leape:ageinfo} argue, the monotonic increase in 
ownership rates may reflect the accumulation of experience with 
different assets as the household ages.  The reduction in the `risky' 
share of the portfolio for the 50+ age groups is interesting because 
it corresponds roughly to the common financial advice to shift assets 
away from risky forms as retirement approaches (though admittedly no 
such pattern is evident for the `broad risky' portfolio share).  Note, 
however, that there is some debate about whether this advice is 
theoretically sound; furthermore, 
\ifthenelse{\boolean{BookVersion}}{as shown by several of the country 
chapters in this volume, }{as shown by the comparative analysis of age 
profiles of risky investment in several countries in Guiso, Haliassos, 
and Jappelli~\cite{ghj:portfolios},} there does not seem to be a 
consistent pattern to age profiles of the risky portfolio share across 
countries.

\subsubsection{International Evidence on Portfolios of the Rich}

Evidence about portfolios of the rich in other countries is presented 
in Table~\ref{table:rich_intl}.\ifthenelse{\boolean{BookVersion}}{This 
table reflects data provided by the respective country experts for 
each of the country chapters examined in this volume.}{The data in 
this table were provided by the respective country experts who 
contributed country chapters to the {\it Household Portfolios} 
conference volume referenced in the bibliography.} Before describing 
the results, it is important to emphasize the problems associated with 
such international comparisons.  Probably the greatest problem is that 
surveys in other countries generally have not made such an intense 
effort as the SCF does to get a large and representative sample of the 
very richest households; furthermore, little is known about exactly 
how participation rates for the wealthy vary across countries.  As a 
result, a table merely presenting data from the top 1 percent of 
surveyed households across countries might well reflect differences in 
survey success and methodology more than actual differences in 
behavior across countries.  Our response to this problem is twofold.  
First, rather than focusing on the top 1 percent, where the variation 
in participation rates is likely to be very large across countries, we 
report information about the top 5 percent of households.  Second, we 
strongly discourage direct comparison of portfolio statistics for the 
`rich' across countries.  Instead, it seems likely to be more reliable 
simply to examine how the differences between the rich and the rest 
vary across countries.

Other survey differences also hamper international comparisons.  From 
the standpoint of comparing the results to the predictions of 
portfolio theory, we would like to be able to divide all assets 
between safe and risky categories.  Unfortunately, the problems in 
making such allocations are even greater in most other surveys than 
they are in the SCF. In particular, most surveys collect little or no 
information about the investment strategies of mutual funds or defined 
contribution pensions, or about the risk characteristics of other 
financial assets.  Given these problems, we concluded that the most 
informative feasible exercise was to allow individual country experts 
to determine, for each asset category, whether there was sufficient 
information about that category to allocate the asset unambiguously to 
one of the four levels of riskiness.  If not, the analyst was asked to 
include the asset in the category `risk characteristics unknown.'  An 
example in the SCF would be a mutual fund which the respondent 
indicated invested in both stocks and bonds.  Because the SCF does not 
collect any information about the proportion of the fund's value 
invested in each of these two categories, we included all such mutual 
fund assets in the `risk characteristics unknown' 
category.\footnote{This contrasts with our strategy in 
Table~\ref{table:riskysafe}, where we divided such investments 50-50 
between the `fairly safe' and `fairly risky' categories.} Under this 
strategy, at least the reader can be confident that the assets 
included in, say, the `clearly risky' category are indeed all risky.

A final problem is in normalization.  Portfolio theory yields 
predictions about the proportion of the portfolio that should be held 
in various kinds of assets.  Accordingly, table~\ref{table:rich_intl} 
reports the ratio of various kinds of nonfinancial assets and debts to 
total net worth.  It is very important to remember, however, that all 
of the measurement problems that affect the components of net worth 
also affect the total.  For example, the net value of private business 
is not measured in the German survey data, and consequently is not 
included in net worth.  Furthermore, the German survey does not 
provide separate data for the value of the respondent's home and the 
value of all other real estate owned by that respondent, so the number 
reported in the table for `private residence' actually reflects all 
real estate.  Since private business wealth constitutes at least 30 
percent of total net worth of the rich in the three countries for 
which survey data on these components of wealth do exist, and 
investment real estate is around another 15 percent of net worth, the 
apparently surprising finding that the gross value of `private 
residence' constitutes 88 percent of net worth for the `rich' German 
households should not be taken at face value.

Keeping all of these problems in mind, a few conclusions still seem 
warranted.

The most important is probably that in every country the top 5 percent 
hold a substantially larger proportion of their financial assets in 
risky forms than do the rest.  The difference is smallest in the UK, 
which may reflect the residual effects of the large-scale 
privatization of the Thatcher years and more recently the 
demutualization of many formerly cooperative financial enterprises.  
\footnote{Shares were distributed to depositors, and thus many 
lower-income households who otherwise owned no shares became 
shareowners.  Research has shown that many lower-wealth households 
have simply held onto the shares they obtained through 
demutualizations.}

Another result common to all countries is that the ratio of debt to 
net worth is substantially smaller for the rich than for the rest, 
although the disparity is enormous in some countries (the US) and 
rather small in others (Italy).

A striking difference across countries is in the breakdown of wealth 
between financial and nonfinancial forms.  The two extremes are the US 
and Italy.  The ratio of nonfinancial to financial wealth for the top 
5 percent in the US is about 1.5, while that ratio in Italy is 
approximately 7.  Similar, though less extreme, results hold for the 
bottom 95 percent of households (where measurement problems are 
probably somewhat smaller).  The Italian country authors indicate that 
part of the discrepancy probably reflects systematic severe 
underestimation of financial assets in Italy.  Nonetheless, while the 
magnitude of the difference may be mismeasured, qualitatively the 
observation that nonfinancial assets are much more important in Italy 
than the US is probably true.

A final observation is that there are large differences in the levels 
of debt held by the bottom 95 percent across countries, ranging from a 
high of \$36,000 in the US to a low of only 4290 euro in Italy.  This 
observation reinforces existing research which has found that more 
highly developed financial markets in the US have allowed much higher 
levels of borrowing.\footnote{Italy is a particularly interesting 
case.  Until recently, the minimum down payment on a home mortgage in 
Italy was on the order of 50 percent, while 5 percent down payment 
mortgages have been common in the U.S. for at least a decade.  
Furthermore, the legal system in Italy makes reposession of property 
extremely difficult and time consuming.  Thus, many Italians cannot 
afford to buy a house, and those who do buy end up borrowing much 
less.  The high value of nonfinancial assets (mainly housing wealth) 
relative to financial is probably largely attributable to these 
features of the Italian financial system.}


\section{Analysis}

It is now time to begin trying to understand the underlying behavioral 
patterns which give rise to the data reported above.  We start by 
presenting a baseline formal model of saving over the life cycle, to 
which we will add a portfolio choice decision.

\subsection{The Basic Stochastic Life Cycle Model}

The following model is what I will henceforth characterize as the 
basic stochastic life cycle model.  The consumer's goal is to 
\begin{eqnarray}
	\max &  & \sum_{s=t}^{T} \beta^{s-t} \mathcal{D}_{t,s} u(C_{t})  \label{eq:maxutil} 
\end{eqnarray}
where $u(C)$ is a constant relative risk aversion utility function 
$u(C)=c^{1-\rho}/(1-\rho)$, $\beta$ is the (constant) geometric 
discount factor, and $\mathcal{D}_{t,s} = \prod_{h=t}^{s-1}(1-d_{h})$ 
is the probability that the consumer will not die between periods $t$ 
and $s$ ($\mathcal{D}_{t,t}$ is defined to be 1; $d_{t}$ is the 
probability of death between period $t$ and $t+1$).

The maximization is of course subject to constraints.  In particular, 
if, following Deaton~\cite{deatonLiqConstrs}, we define $X_{t}$ as 
`cash-on-hand' at time $t$, the sum of wealth and current income, then 
the consumer faces a budget constraint of the form
\begin{eqnarray*}
	X_{t+1} & = & R_{t+1}S_{t} + Y_{t+1}
\end{eqnarray*}
where $S_{t} = X_{t}-C_{t}$ is the portion of last period's resources 
the consumer did not spend, $R_{t+1}$ is the gross rate of return 
earned between $t$ and $t+1$, and $Y_{t+1}$ is the noncapital income 
the consumer earns in period $t+1$.

Assume that the consumer's noncapital income in each period is given 
by their permanent income $P_{t}$ multiplied by a mean-one transitory 
shock, $E_{t}[\tilde{\epsilon}_{t+1}] = 1$, and assume that permanent 
income grows at rate $G_{t}$ between periods, but is also buffeted by 
a mean-one shock, $P_{t+1} = G_{t+1}P_{t}\eta_{t+1}$ such that $E_{t} 
[\tilde{\eta}_{t+1}]=1$, where our notational convention is that a 
variable inside an expectations operator whose value is unknown as of 
the time at which the expectation is taken has a $\sim$ over it.

Given these assumptions, the consumer's choices are influenced by
only two state variables at a given point in time: the level of 
the consumer's assets $X_{t}$ and the level of permanent income, $P_{t}$.
As usual, the problem can be rewritten in recursive form with a value
function $V_{t}(X_{t},P_{t})$.  Written out fully in this form,
the consumer's problem is

\begin{eqnarray}
	V_{t}(X_{t},P_{t}) & =& \max_{\{C_{t}\}} 
	u(C_{t}) + \beta \mathcal{D}_{t,t+1} E_{t}\left[V_{t+1}(\tilde{X}_{t+1},\tilde{P}_{t+1}) \right] \\ 
	 & \mbox{such that} &   	\label{eq:bellmanstd} \nonumber \\
	S_{t}   & = & X_{t}-C_{t} \nonumber \\
	X_{t+1} & = & R_{t+1}S_{t} + Y_{t+1} \nonumber \\
	Y_{t+1} & = & P_{t+1}\epsilon_{t+1} \nonumber \\
	P_{t+1} & = & G_{t} P_{t} \eta_{t+1} \nonumber 
\end{eqnarray}

\subsection{The Saving Behavior of the Rich}


Within the last decade, advances in computer speed and numerical 
methods have finally allowed economists to solve life cycle 
consumption/saving problems like that presented above with serious 
uncertainty and realistic utility (see, in particular, Hubbard, 
Skinner, and Zeldes~\cite{hsz:importance}; 
Huggett~\cite{huggett:wealth}; Carroll~\cite{carroll:bslcpih}; and the 
references therein).  I have argued elsewhere 
(Carroll~\cite{carroll:bslcpih}) that the implications of these models 
fit the available evidence on the consumption/saving behavior of the 
typical household reasonably well, certainly much better than the old 
Certainty Equivalent (CEQ) models did.

However, another finding from this line of research has been that the
model is unable to account for the very high concentrations of wealth
at the top of the distribution.  

\subsubsection{How Rich Are They?}

Figure~\ref{fig:Top1pctWProfilePatientvsSCF} shows the ratio of wealth 
to permanent income\footnote{SCF respondents are asked whether their 
total income this year was above normal, about normal, or below 
normal.  Following Friedman~\cite{friedmanATheory}, I define 
permanent income as the level of income the household would normally 
receive.} by age for the population as a whole and for the households 
in the richest one percent by age category from the 1992 and 1995 
SCFs.  Also plotted for comparison is the level of the wealth to 
income ratio at the top 1 percent implied by a standard life cycle 
model of saving similar to that in Carroll~\cite{carroll:bslcpih} or 
Hubbard, Skinner, and Zeldes~\cite{hsz:importance}.  (Specifically, it 
is the Carroll model with HSZ `baseline' parameter values).  The 
richest one percent are much richer than implied by the life cycle 
model.  In addition, the figure plots the age profile of the 99th 
percentile that would be implied by the HSZ model if it were assumed 
that households do not discount future utility at all.  The figure 
shows that even with such patient households, the model remains far 
short of predicting the observed wealth to income ratios at the 99th 
percentile.\footnote{This figure is reproduced from 
Carroll~\cite{carroll:richsave}.}  

This finding is reconfirmed in a recent paper by Engen, Gale, and 
Uccello~\cite{egu:adequacy}, who do a very careful job of modelling 
pension arrangements, tax issues, and other institutional details 
neglected in Carroll~\cite{carroll:richsave} and also find that the 
wealth-to-income ratios at the top part of the income distribution are 
much greater than predicted by a life cycle dynamic stochastic 
optimization model, even with a time preference rate of zero.

\subsubsection{How Do They Spend It All?}

They don't.  

In the 1989, 1992, and 1995 SCFs, households were asked whether their 
spending usually exceeds their income, and whether their spending 
exceeded their income in the previous year.  In order to run down 
their wealth, households obviously must eventually spend more than 
their income.  Yet only five percent of the rich elderly households in 
the SCF answered that their spending usually exceeded their income.

More evidence is presented in Figure~\ref{fig:OldRichDontDissave}, 
which shows the levels of wealth by age for the elderly in the 1992 
and 1995 SCFs.  There is no evidence in this figure that wealth is 
declining for this population; indeed, if anything it seems to be 
increasing,\footnote{This is in effect a smoothed profile of wealth by 
age adjusted for cohort effects; see Carroll~\cite{carroll:richsave} 
for methodological details.} consistent with the answers that the rich 
elderly give to the questions about whether they are spending more 
than their incomes.  The implication is that most of the wealth which 
we observe them holding will still be around at death.  This is 
clearly a problem for any model in which the only purpose in saving is 
to provide for one's own future consumption.

This crude evidence is backed up by a study by Auten and 
Joulfaian~\cite{auten&joulfaian:charitable} which finds that the 
elasticity of bequests with respect to lifetime resources is well in 
excess of one (their point estimate is 1.3).  See 
Carroll~\cite{carroll:richsave} for a summary of further evidence 
that, far from spending their wealth down, the rich elderly continue 
to save.

\subsection{Adding Portfolio Choice}

Recently, a wave of papers (Bertaut and 
Halaissos~\cite{bertaut&haliassos:portfolio}; 
Fratantoni~\cite{fratantoni:equitypremium};Gakidis~\cite{gakidis:stocksforold}; 
Cocco, Gomes, and Maenhout~\cite{cgm:lcportfolio}; and 
Hochgurtel~\cite{hochgurtel:bufferportfolio}) has examined the 
predictions of stochastic life cycle models of the kind considered 
above when households facing labor income risk are allowed to choose 
freely between investing in a low-return safe asset and investing in 
risky assets parameterized to resemble the returns yielded by equity 
investments in the past.

The only modification to the formal optimization problem presented 
above necessary to allow portfolio choice is to designate $R_{t+1}$ as 
the portfolio-weighted return, which will depend on the proportion of 
the portfolio that is allocated to the safe and the risky assets, and 
on the rate of return on the risky asset between $t$ and $t+1$.  Call 
the proportion of the portfolio invested in the risky asset (`stocks') 
$w_{s,t}$ (where $w$ is mnemonic for the portfolio `weight'), and 
$(1-w_{s,t})$ is the portion invested in the safe asset.  If the 
return on stocks between $t$ and $t+1$ is $R_{s,t+1}$, the 
portfolio-weighted return on the consumer's savings will be 
$R(1-w_{s,t})+R_{s,t}w_{s,t}$.

However, even without solving a model of this type formally, it is 
clear that such models will not be able to explain the empirical 
differences between the portfolio behavior of the rich and the 
behavior of the rest of the population, because when the utility 
function is in the CRRA class, problems of this type are homothetic.  
That is, there is no systematic difference in the behavior of 
households at different levels of lifetime permanent income.  Hence, 
such models provide no means to explain the very large differences 
between the rich and the rest in saving and portfolio behavior 
documented above.

\subsection{Three Possible Modifications}

There are at least three ways one might consider modifying the model 
in hopes of explaining the apparent nonhomotheticity of saving and 
portfolio behavior.


\subsubsection{Heterogeneity in Risk Tolerance}

The first is simply to allow for exogenous, immutable {\it ex ante} 
heterogeneity in risk tolerance across members of the population.  
Formally, rather than assuming that all households have the same value 
of $\rho$, we can assume that each household has an idiosyncratic, 
specific $\rho_{i}$.

The effect of this would be to allow households with low values of 
$\rho$ (high risk tolerance) to choose highly risky but 
high-expected-return portfolios.  On average, the risk-tolerant 
households would be rewarded with higher returns and would therefore 
end up richer than the rest of the population.  Thus, the rich would 
be disproportionately risk-lovers, and would therefore have riskier 
portfolios than the rest.  As shorthand, I will call this the 
`preference heterogeneity' story henceforth.

\subsubsection{Capital Market Imperfections}

A second possibility is to follow Gentry and 
Hubbard~\cite{gentry&hubbard:wealthysave} and 
Quadrini~\cite{quadrini:entrepreneurship} in assuming that there are 
important imperfections in capital markets which 1) require 
entrepreneurial investment to be largely self-financed; 2) imply that 
entrepreneurial investment has a higher return than investments made 
on open capital markets; and 3) require a large minimum scale of 
investment.  As those authors show, the combination of these three 
assumptions can yield an implication that portfolios of higher wealth 
or higher income households will be much more heavily weighted toward 
entrepreneurial investments, and that rich households with business 
equity have higher than average saving rates (under the further 
assumption that the intertemporal elasticity of substitution is high 
which means that they take advantage of the high returns that are 
available to them by saving more).  I will refer to this theory as the 
`capital market imperfections' story.

\subsubsection{Bequests as a Luxury Good}

A final possibility is to change the assumption about the lifetime 
utility function.  Carroll~\cite{carroll:richsave} proposes adding a 
`joy of giving' bequest motive of the form $B(S)$ in a modified 
Stone-Geary form,\footnote{It might seem that a `joy of giving' 
bequest motive and a `dynastic bequest motive' of the type considered 
by Barro~\cite{barro:bondsnetworth} would be virtually 
indistinguishable, but it turns out that there are several important 
differences.  For example, the dynastic bequest model collapses to a 
standard life cycle model for households with no offspring, yet 
empirical evidence suggests that the rich childless elderly continue 
to save.  See Carroll~\cite{carroll:richsave} for more arguments that 
the `joy of giving' bequest motive fits the data better.}
\begin{eqnarray*}
B(S) &= &\frac{(S + \gamma)^{1-\alpha}}{1-\alpha}.
\end{eqnarray*}

Carroll~\cite{carroll:richsave} shows that if one assumes that 
$\alpha<\rho$ then wealth will be a `luxury good' in the sense that as 
lifetime resources rise, a larger proportion of those resources is 
devoted to $S_{T}$.  In the limit as lifetime resources approach 
infinity, the proportion of resources devoted to the bequest 
approaches 1.  The other salient feature of the model is that if 
$\gamma > 0$ there will be a `cutoff' level of lifetime resources 
such that households poorer than the cutoff will leave no bequest at 
all.  Thus the model is capable of matching the crude stylized fact 
that low-income people tend to leave no bequests, and also 
captures the fact (from Auten and 
Joulfaian~\cite{auten&joulfaian:charitable}) that among those who 
leave bequests, the elasticity of lifetime bequests with respect to 
lifetime income is greater than one.

In this paper the assumption is that one receives utility from the 
contemplation of the potential bequest in proportion to the 
probability that death (and the bequest) will occur.  Thus Bellman's 
equation is modified to
\begin{eqnarray}
	V_{t}(X_{t},P_{t}) & =& \max_{\{C_{t},w_{s,t}\}} 
	u(C_{t}) + \beta (1-d_{t}) E_{t}\left[V_{t+1}(\tilde{X}_{t+1},\tilde{P}_{t+1}) \right] + d_{t} B(S_{t}), \nonumber
\end{eqnarray}
and the transition equations for the state variables are unchanged.

While it is obvious how this model might help to explain the high 
saving rates of the rich, it is not so obvious why it might help 
explain the high degree of riskiniess of their portfolios.  It turns 
out, however, that {\it precisely the same assumption which implies 
that bequests are a luxury good also implies that households are less 
risk-averse with respect to gambles over bequests than with respect to 
gambles over consumption.}\footnote{The presence of the $\gamma$ term 
in $B(S)$ implies increasing relative risk aversion as bequest gambles 
get larger.  However, this does not alter the fact that risk aversion 
with respect to gambles over bequests is always less than risk 
aversion with respect to gambles over consumption.} That assumption is 
that the exponent on the utility-from-bequests function $\alpha$ must 
be less than the exponent on the utility from consumption $\rho$.  
This implies that the marginal utility from bequests declines more 
slowly than the marginal utility from consumption and thus as wealth 
rises more and more of it is devoted to bequests rather than 
consumption.  However, the traditional interpretation of exponents 
like $\rho$ and $\alpha$ in utility functions of this class is as 
coefficients of relative risk aversion, so the assumption that 
bequests are a luxury good has the immediate implication of less risk 
aversion with respect to bequest gambles than consumption gambles!

Following Max Weber as recently interpreted by Zou~\cite{zou:spirit} 
and Bakshi and Chen~\cite{bakshi&chen:spirit}, I will henceforth call 
this the ``Capitalist Spirit'' model.

\subsection{Distinguishing the Three Models}

All of these theories can in principle explain the basic facts that 
the portfolios of the rich are disproportionately risky and that 
investments in closely-held businesses are a disproportionate share of 
the portfolios of the rich.  This section attempts to distinguish 
between the three theories on the basis of other kinds of evidence.

We begin with some direct evidence that there are substantial
differences in the risk preferences of the rich compared with the
rest of the population.  Table~\ref{table:riskaver} reports 
the results of a direct question SCF respondents are asked about
their risk tolerance.  Specifically, the respondents are asked
\small
\begin{quote}
Which of the statements on this page comes closest to the amount of 
financial risk that you (and your [husband/wife/partner]) are willing 
to take when you save or make investments?

\begin{enumerate}
\item TAKE SUBSTANTIAL FINANCIAL RISKS EXPECTING TO EARN SUBSTANTIAL RETURNS.
\item TAKE ABOVE AVERAGE FINANCIAL RISKS EXPECTING TO EARN ABOVE AVERAGE RETURNS.
\item TAKE AVERAGE FINANCIAL RISKS EXPECTING TO EARN AVERAGE RETURNS.   
\item NOT WILLING TO TAKE ANY FINANCIAL RISKS.
\end{enumerate}
\end{quote}
\normalsize

For 1992 and 1995, the table reports the mean values of the response 
and the percent of households reporting that they are not willing to 
take any financial risks, by permanent income and net worth 
percentile.\footnote{Our method of identifying permanent income is 
simple: we restrict the sample to households who reported that their 
income in the survey year was `about normal.'  Thus we are employing 
Friedman's original definition of permanent income, rather than modern 
definitions as the annuity value of human and nonhuman wealth.} The 
table shows that occupants of the highest permanent income and net 
worth brackets are notably more likely to express a willingness to 
accept above-average risk in exchange for above-average returns.  Even 
more dramatic is the difference between the proportion of the rich and 
of the rest who express themselves as `not willing to take any 
financial risks.'  Among the richest 1 percent by wealth, less than 
ten percent express such extreme risk aversion; among the bottom 80 
percent, nearly half express this sentiment.

Although economists have traditionally dismissed answers to survey 
questions of this type as meaningless, a recent literature (with 
contributions by Kahneman, Wakker and Sarin~\cite{kws:bentham}, 
Oswald~\cite{oswald:happiness}, Barsky, Juster, Kimball, and 
Shapiro~\cite{bjks:askforcrra}, and Ng~\cite{ng:acase}) has argued 
forcefully that answers to questions about preferences can provide 
reliable and useful information.  Thus, these answers should be taken as serious 
evidence that the rich are more risk tolerant than the rest.

However, the table does not answer the question of the direction of 
causality between risk preference and wealth.  It is possible, as the 
preference heterogeneity story would have it, that high risk tolerance 
leads to wealth, but it is equally possible that there is causality 
from wealth to risk tolerance.

One piece of existing evidence that is suggestive of causality from 
wealth to risk tolerance is the finding by Holtz-Eakin, Rosen, and 
Joulfaian~\cite{hrj:entrep} that the receipt of an inheritance 
substantially increases the probability that the recipient will start 
an entrepreneurial venture.  Their interpretation is that because the 
inheritors presumably knew that they would eventually inherit, their 
failure to start the entrepreneurial venture in advance of the receipt 
of the inheritance demonstrates the presence of liquidity constraints.  
An alternative interpretation is that the increase in disposable 
wealth increases the household's risk tolerance enough for them to 
become willing to take the risk of starting an entrepreneurial 
venture.\footnote{Note that capital market imperfections or 
uncertainty about the timing and/or size of the bequest are still 
required; without either uncertainty or imperfections the household's 
effective wealth would not change when a perfectly anticipated bequest 
was received.}

The ideal experiment to answer the causality question would be to 
exogenously dump a large amount of wealth on a random sample of 
households and examine the effect both on their expressed risk 
preferences and on their risk-taking behavior.  The closest 
approximation to this ideal experiment in an available dataset is the 
receipt of unexpected inheritances between the 1983 and 1989 panels of 
the SCF.

Table~\ref{table:drho} presents the results of a simple regression 
analysis of the change in risk aversion between 1983 and 1989 on the 
size of inheritances received between the two surveys, using the 
numerical answer to the survey question about risk attitudes as the 
measure of risk aversion.  That is, defining RISKAV83 as the 1983 
answer to the risk aversion question and RISKAV89 as the 1989 answer, 
we define DRISKAV = RISKAV89-RISKAV83 and regress DRISKAV on a measure 
of the size of inheritances received and a set of control 
variables.\footnote{The sample is restricted to households whose 
composition did not change between the two survey years, in order to 
ensure that changes in risk aversion really reflect changes in the 
attitudes of the same individual(s).} Specifically, LINH is the log of 
the value of inheritances, and the control variables in the weighted 
regression are the same as the variables used by Gentry and 
Hubbard~\cite{gentry&hubbard:wealthysave} in their extensive 
investigation of entrepreneurship using these data.

The coefficient on LINH is overwhelmingly statistically signficant and 
negative, indicating that larger inheritances produce a greater 
decline in risk aversion.  Recall that the simple preference 
heterogeneity story was one in which individuals enter the workforce 
with a built-in level of risk aversion which was unchanging through 
the lifetime.  If we interpret the RISKAV83 and RISKAV89 variables as 
measures of this risk aversion, the results in Table~\ref{table:drho} 
constitute a direct rejection of this story.  Indeed, almost half of 
households whose composition is unchanged report a different value of 
RISKAV89 than RISKAV83, and given that a large proportion of the 
change can be explained {\it ex post} via regressions like that 
reported in Table~\ref{table:drho}, it is clear that these changes do 
not merely reflect measurement error.

One potential problem with this experiment is that inheritances may be 
anticipated.  If so, the recipient might take the prospective 
inheritance into account in formulating risk attitudes even before 
actual receipt.  However, if this were true, a regression of the 
change in risk preferences on the size of inheritances received would 
find a coefficient of zero, and so the fact that we found a highly 
significantly negative coefficient despite this bias only strengthens 
the case that changes in wealth affect risk aversion.  Indeed, when we 
restrict the sample (in column 2) to those households who said in 1983 
that they did not expect ever to receive a substantial inheritance 
(and who presumably were surprised when they did), the coefficient 
estimate is a bit larger (though the difference is not statistically 
significant).

Unfortunately, there is a more serious problem with the experiment: 
The survey question cannot necessarily be interpreted as revealing the 
respondent's underlying coefficient of relative risk aversion.  
Instead, the question is about the respondent's willingness to bear 
financial risk, and economic theory informs us that willingness to 
bear financial risk should depend upon a great many factors in 
addition to an agent's raw coefficient of relative risk aversion.  In 
particular, what should matter is the expected coefficient of relative 
risk aversion for the future period's value function, which may 
depend, for example, on whether the consumer anticipates possibly 
being liquidity constrained in that future period.  However, one 
conclusion from the recent work on portfolio theory cited above is 
that the proportion of the portfolio invested in the risky asset 
should {\it decline} in the level of current-period cash-on-hand.  The 
reason for this counterintuitive result is that when there is little 
financial wealth, virtually all of future consumption will be financed 
by labor income, and so adding a small financial risk has very little 
effect on overall consumption risk, so the agent is willing to invest 
a large proportion of her modest portfolio in the risky financial 
asset (this argument relies on an implicit assumption that the 
correlation between financial risk and labor income risk is low, as 
Cocco et.  al.~\cite{cgm:lcportfolio} show it is).  As wealth grows 
large, however, the proportion of future consumption to be financed 
out of that wealth also grows large, and thus willingness to bear 
additional financial risk declines (see Cocco, Gomes, and 
Maenhout~\cite{cgm:lcportfolio} for a fuller discussion of these 
issues).  Thus, appropriately calibrated portfolio theory implies that 
we would expect to see a declining willingness to bear financial risk 
as wealth increased, rather than the reverse as indicated in the 
table.

Many economists remain uncomfortable with using survey measures like 
the SCF risk attitudes question.  However, even if the results of 
Table~\ref{table:drho} are set aside, there are several other problems 
with the preference heterogeneity story as a complete explanation for 
the observed pattern of facts.

In principle, the preference heterogeneity story can indeed explain 
the large share of business equity in the portfolios of the richest 
households, under the assumption that private business investments 
bear the highest risk and the highest return among the categories of 
assets available.  This assumption is plausible and therefore not 
problematic.  However, several other features of the entrepreneurial 
investments of the rich {\it are} problematic for this theory.

First, entrepreneurial investments of the rich are highly 
undiversified.  If the rich are even slightly risk-averse, elementary 
portfolio theory under perfect capital markets implies that the 
optimal strategy is to invest a tiny amount in each of a large number 
of entrepreneurial ventures in order to diversify the idiosyncratic 
risk.  Table~\ref{table:buswnotdiversified} shows that instead, among 
the rich households with any private business equity, over 80 percent 
of that equity is in a single entreprenurial venture, while the three 
largest entreprenurial investments account for 94 percent of 
entrepreneurial wealth (a similar pattern holds for nonrich 
entrepreneurs).  Furthermore, for rich entrepreneurial households, 
almost half of all income comes directly from business enterprises in 
which the household has an ownership stake.  Failure of the business 
would wipe out not only the asset value of the business, but also the 
business-derived income, and thus the total riskiniess of business 
ownership is even greater than appears from the share of business 
equity in total net worth.  This means the incentive for 
diversification is even stronger.

The next problem for the preference heterogeneity theory is that it 
provides no explanation for the fact that the great majority of 
entrepreneurial wealth is in enterprises in which a member of the 
household has an active management role.  
Table~\ref{table:buswnotdiversified} shows that 85 percent of all 
entrepreneurial wealth is held in such `actively managed' businesses.  
Again, with perfect capital markets, management should be completely 
detached from ownership to diversify idiosyncratic risk.

A final problem is that the preference heterogeneity story provides no 
explanation for the failure of the elderly rich to spend down their 
assets.  Indeed, because risk tolerance is positively correlated with 
the intertemporal elasticity of substitution in models with 
time-separable preferences, we should actually expect the rich to be 
running down their wealth {\it faster} than the non-rich if the only 
difference in preferences between the rich and non-rich is in their 
degree of risk tolerance.

Given that several of the preceding arguments imply that the 
preference heterogeneity story also requires some form of capital 
market imperfections in order to explain the data, it is interesting 
to examine whether capital market imperfections by themselves might do 
the trick.

The central requirement of any story based purely on capital market 
imperfections is that business ownership must yield higher-than-market 
rates of return.  Unfortunately, the economics and business 
literatures do not appear to contain credible estimates of the average 
rate of return on closely-held business ventures.  Suppose for the 
moment that we accept on faith the proposition that closely-held 
business ventures earn a higher rate of return (in exchange for higher 
risk) than is available on open capital markets, and that such 
ventures must be substantially self-financed for moral hazard or 
adverse selection reasons.  By themselves, and in the absence of 
preference heterogeneity, these assumptions cannot explain the 
positive correlation between the level of initial labor income or 
initial wealth and the propensity to start businesses documented by 
Quadrini~\cite{quadrini:entrepreneurship} and Gentry and 
Hubbard~\cite{gentry&hubbard:wealthysave}.  The problem is that these 
arguments should apply with just as much force to very small business 
ventures (which can be financed without large initial wealth or 
income) as to larger ones: As anyone who has read the novel {\it A 
Confederacy of Dunces} knows, there are principal/agent and moral 
hazard problems even for a hot dog vendor.

Gentry and Hubbard~\cite{gentry&hubbard:wealthysave} address this 
problem by simply assuming that there is a minimum efficient scale for 
business enterprises which is large relative to the resources of the 
median household, but this approach is insufficient to explain the 
data because the richest households would have wealth vastly greater 
than any fixed minimum efficient scale and therefore would have no 
need to tie up more than a trivial fraction of their total net worth 
in any single business enterprise.  
Quadrini~\cite{quadrini:entrepreneurship} deals with this problem by 
postulating a `ladder' of business opportunities at ever-rising 
minimum efficient scales, so that no matter how rich the household 
becomes there is always an opportunity to jump up to an even-higher 
rung on the ladder.

Even if we were to accept the story that there is a complicated ladder 
of minimum efficient scales of business operation {\it a la} Quadrini, 
the capital market imperfections story still faces three problems.  
First, it provides no explanation for the failure of the rich elderly 
eventually to begin running down their wealth.  Second, if the risk 
preferences of the rich were similar to those of the rest, the extra 
risk associated with their entrepreneurial wealth should induce them 
to try to minimize the riskiness of the remainder of their portfolio.  
However, Table~\ref{table:busownriskyfin} shows that the financial 
asset holdings of rich households who have a substantial fraction of 
their net worth tied up in business equity are actually considerably 
{\it riskier} than the financial asset holdings of the rest of the 
population (although less risky than the financial investments of the 
rich nonentrepreneurs).  Finally, the results in 
Table~\ref{table:riskaver} strongly suggest that the rich, whether 
entrepreneurs or nonentrepreneurs, are much more risk tolerant than 
the rest of the population, and capital market imperfections alone can 
explain neither this nor the finding in Table~\ref{table:drho} that 
increases in wealth produce reductions in reported risk aversion.

It is now time to consider whether the `capitalist spirit' model can 
explain the overall pattern of facts.  Recall that this model assumed 
that bequests are a luxury good, with the corollary implication that 
households are less risk averse with respect to risks to their bequests 
than with respect to risks to their consumption.  Since the luxury 
good aspect of bequests implies that as a household becomes richer, it 
plans to devote more and more of its resources to the bequest, the 
model also implies that risk aversion declines in the level of wealth.  
Thus the capitalist spirit model is consistent with the results on 
self-reported risk attitudes, as well as with the risky portfolio 
structure for the rich, and with the evidence that receipt of 
inheritances reduces risk aversion (at least if we assume that those 
inheritances were not perfectly anticipated).  It also can explain why 
higher income or higher net worth households are more likely to invest 
in risky entrepreneurial ventures.  Finally, it can explain the 
failure of the elderly rich to run down their assets before death 
(indeed, this is the empirical fact that the model was developed to 
explain; its ability to explain the other empirical patterns 
documented here was not anticipated in the original statement of the 
model).

However, if capital markets were perfect, rich households would still 
have every incentive to diversify the idiosyncratic component of their 
entrepreneurial investments by holding small shares of many 
entrepreneurial ventures.  Their failure to do so is presumably 
explained by capital market imperfections.\footnote{In discussing this 
paper, Marco Pagano suggested that entrepreneurs may obtain utility 
directly from the ownership and consequent control over their 
entrepreneurial ventures.  This would lead to a preferences-based 
theory of the nondiversification of entrepreneurial investment.  
However, in order to explain the overall pattern of facts, it would 
still be necessary to modify the utility function to put wealth in the 
utility function in some form, since not all of the rich are 
entrepreneurs.  Because there is substantial independent evidence of 
capital market imperfections, it seems preferable to stick with a 
story which explains the facts via an assumption of capital market 
imperfections plus a single change in the utility function rather than 
with perfect capital markets plus two changes in the utility 
function.} Note, however, that one attractive feature of a model which 
combines capital market imperfections with the `capitalist spirit' 
utility function is that it is possible to dispense with the awkward 
assumption of a `ladder' of minimum efficient scales which was 
necessary in the basic model of capital market imperfections in order 
to explain the data.  This makes the analysis of such models 
considerably more tractable, transparent, and plausible.

\section{Conclusions}

The standard model of household behavior implies that the rich are 
just like scaled-up versions of everybody else, including in their 
portfolio allocation patterns.  The data summarized in this paper 
contradict that assertion both for the US since 1963 and for the 
other countries included in this survey.

The most important differences between the portfolios of the rich and 
the rest are the much higher proportion of their assets that the rich 
hold in risky forms, and their much higher propensity to be involved 
in entrepreneurial activities and to hold much of their net worth in 
the form of their own entrepreneurial ventures.  Several different 
features of the data point to a conclusion that relative risk aversion 
is a decreasing function of wealth.  Other features, particularly the 
concentration of the wealth of the rich in their own entrepreneurial 
ventures, suggest that capital market imperfections are also 
important.  But it appears that most of the features of the data can 
be explained by assuming both that bequests are a luxury good and that 
capital market imperfections require entrepreneurial enterprises to be 
largely self-financed and self-managed.

This is not to say that there could not also be exogenous differences 
in risk aversion across households; just that the case does not appear 
to be strong that such differences are necessary to explain the 
differences between the portfolios of the rich and the rest.


\vfill\eject
\input Portable:latex:texhtml.tex

\bibliographystyle{Portable:latex:econometrica_fullnames}
\bibliography{Portable:latex:economics}


\ifthenelse{\boolean{Tables}}{

\renewcommand{\baselinestretch}{1.0} \normalsize

\baselineskip 12pt

\begin{table}
\ifthenelse{\boolean{BookVersion}}{\caption{}}{\caption{Major Features of the Tax Code Relevant for the Rich}}
\label{table:laws}\vspace{.2in}
\centerline{\BoxedEPSF{:Tables:laws.eps}}
{\small

\vspace{.25in}

Sources:

For marginal and effective rates prior to 1980, see 
Brownlee~\cite{brownlee:historical}.  For marginal rates from 
1980-1998, see Booth~\cite{booth:taxrevs}.  For effective rates from 
1980-93, see Slemrod~\cite{slemrod:progressivity}.  For effective 
rates for 1995, see Kasten, Sammartino, and Weiner~\cite{ksw:liabs}.  
For estate and gift tax information, see Johnson and 
Eller~\cite{johnson&eller:inhtax} and 
Joulfaian~\cite{joulfaian:estatetax}.
}
\end{table}

\clearpage

\begin{table}
\ifthenelse{\boolean{BookVersion}}{\caption{}}{\caption{Ownership Rates of Assets and Liabilities}}
\label{table:pctown}
\centerline{\BoxedEPSF{:Tables:pctown.eps}}
\end{table}


\clearpage

\begin{table}
\ifthenelse{\boolean{BookVersion}}{\caption{}}{\caption{Composition of Net Worth}}
\label{table:compnw}
\centerline{\BoxedEPSF{:Tables:compnw.eps}}
\end{table}

\clearpage

\begin{table}
\ifthenelse{\boolean{BookVersion}}{\caption{}}{\caption{Composition of Net Worth by Risk Category}}
\label{table:riskysafe}%\vspace{.3in}
\centerline{\BoxedEPSF{:Tables:riskysafe.eps}}
\end{table}

\clearpage 

Calculations of fairly risky and fairly safe mutual funds 
and defined contribution pensions in Table~\ref{table:riskysafe} are 
as follows.

\begin{itemize}

\item[1962 SFCC] Due to the lack of information on mutual fund 
investment strategies, all mutual funds are classified as risky and 
all defined contribution pensions are classified as safe.

\item[1983 SCF] The 1983 SCF did not ask about the investment strategy 
or risk characteristics of mutual funds or retirement accounts, so we 
had to make educated guesses based on other information.  Tax-free 
mutual funds were allotted to the 'fairly safe' category because such 
funds consist almost exclusively of state and local government bonds, 
direct holdings of which we put in this category.  Taxable mutual 
funds were allotted to the 'risky' category, because in the early 
1980s these funds typically contained a mix of stocks and bonds.  The 
calculation of risky and fairly safe, and clearly safe defined 
contribution pensions uses the institution that held the IRA/Keogh 
accounts as a proxy for investment direction.  If a real estate 
investment company held the accounts, then those defined contribution 
pensions were considered risky.  If a commercial bank, savings and 
loan, or credit union held the accounts, then those assets were 
considered fairly safe.  If a brokerage, insurance company, employer, 
school/college/university, investment management company, or the AARP 
held the accounts, the defined contribution pensions were split 50/50 
between the fairly safe and risky.  In the case that the household had 
no IRA/Keogh accounts, but had a thrift pension account, the assets 
were considered fairly safe.

\item[1989-1995 SCF] These surveys asked about the investment strategy 
for mutual funds and retirement accounts.  Funds and accounts that 
consisted exclusively of one category of asset (such as stock or bond 
mutual funds) we allocated in the same way that we allocated direct 
holdings of that asset type.  Mutual funds and accounts that contained 
a mix of stocks and bonds were allocated half-and-half to the 'fairly 
safe' and 'risky' categories.  Accounts invested in real estate, 
commodities or limited partnerships were put in the `risky' category.

\end{itemize}

\clearpage\vfill\eject
\begin{table}
\ifthenelse{\boolean{BookVersion}}{\caption{}}{\caption{Degree of Diversification of Portfolio Structure}}
\label{table:divers}
\centerline{\BoxedEPSF{:Tables:divers.eps}}
\vspace{.15in}\footnotesize
\noindent Note: A description of the asset classifications appears in 
the notes at the end of Table \ref{table:riskysafe}.

\noindent Source: Author's calculations using the Survey of Financial 
Characteristics of Consumers and Surveys of Consumer Finances
\end{table}

\clearpage

\begin{table}
\ifthenelse{\boolean{BookVersion}}{\caption{}}{\caption{Risk Bearing By Age}}
\label{table:riskybyage}
\centerline{\BoxedEPSF{:Tables:riskybyage.eps}}
\vspace{.05in}
\noindent Notes: The definition of risky financial assets corresponds 
to the sum of clearly risky and fairly risky assets defined in Table 
\ref{table:riskysafe}.  The definition of broad risky assets 
corresponds to the `risky assets - broad' classification in Table 
\ref{table:riskysafe}.
\medskip

\noindent Source: Survey of Financial Characteristics of Consumers and 
Surveys of Consumer Finances

\end{table}

\clearpage

\begin{table}
\ifthenelse{\boolean{BookVersion}}{\caption{}}{\caption{International Comparison of Portfolio Structure}}
\label{table:rich_intl}
%\centerline{\BoxedEPSF{:Tables:rich_intl_p1.eps}}
\end{table}


\clearpage

. % Make blank pages which will contain the rich_intl table data

\clearpage\vfill\eject

. % Make blank pages which will contain the rich_intl table data


\clearpage\vfill\eject

\input :Tables:riskaver

\clearpage

\begin{table}
\ifthenelse{\boolean{BookVersion}}{\caption{}}{\caption{Effect of Inheritances on the Change in Risk Aversion}}\medskip
\label{table:drho}
\centerline{\BoxedEPSF{:Tables:drho.eps}}
\noindent 
\end{table}
\medskip\footnotesize
Notes: 					
Dependent variable DRISKAV is the change in attitude toward financial risk for the household between 1983 and 1989, as described in the text.
A negative change implies a reduction in risk aversion.
Standard errors in parentheses.					
$^{*}$denotes significance at the 90 percent level;
$^{**}$denotes significance at the 95 percent level;
$^{***}$denotes significance at the 99 percent level.
The first column ("All Recipients") reports results including all 
households who received an inheritance between 1983 and 1989.  The 
second column ("Surprised") includes all households who received an 
inheritance between 1983 and 1989 but reported in 1983 that they did 
not expect ever to receive a substantial inheritance.
The regression specification follows Gentry and Hubbard's baseline
specification.  Variable definitions: 

\begin{tabular}{ll}
LINH &  Log of total value of inheritances received between 1983 and 1989					
\\ MARRIED &  Dummy variable for hh head married in 1989					
\\ KIDS &  Number of kids under 18 in the household in 1983					
\\ A2 &  Age dummy, hh head between 35 and 54 in 1983					
\\ A3 &  Age dummy, hh head at least 55 in 1983					
\\ UNEMP83 &  Dummy variable for hh head unemployed in 1983 and employed in 1989
\\ UNEMP89 &  Dummy variable for hh head employed in 1983 and unemployed in 1989
\\ OWNHOME &  Dummy variable for hh being a homeowner in 1983
\\ EDUC &  HH head years of education in 1989
\end{tabular}
\normalsize
\clearpage


\begin{table}
\ifthenelse{\boolean{BookVersion}}{\caption{}}{\caption{Lack of Diversification of Business Wealth}}\medskip
\label{table:buswnotdiversified}
\centerline{\BoxedEPSF{:Tables:buswnotdiversified.eps}}
\end{table}

\clearpage

\begin{table}
\ifthenelse{\boolean{BookVersion}}{\caption{}}{\caption{Riskiness of Financial Assets in Entrepreneurs' Portfolios}}
\label{table:busownriskyfin}
\centerline{\BoxedEPSF{:Tables:busownriskyfin.eps}}
\vspace{.05in} \noindent \medskip 

Notes: 

Only households with $\geq $ \$1000 in net worth are included 
in this table.

BUS is defined as the ratio of noncorporate business equity to total 
net worth.  Definitions of financial assets, safe, and risky assets 
are as in previous tables.

Source: Calculations by the author using the 1995 Survey of Consumer Finances	
\end{table}


}

\clearpage

\begin{figure}
	\centerline{\BoxedEPSF{:Figures:Top1pctWProfilePatientvsSCF.eps}}
	\medskip\medskip\medskip\medskip
	\ifthenelse{\boolean{BookVersion}}{\caption{}}{\caption{Wealth Profiles for Baseline and More Patient Households}}
	\label{fig:Top1pctWProfilePatientvsSCF}
	Source: Reproduced from Carroll~\cite{carroll:richsave}
\end{figure}

\begin{figure}
	\centerline{\BoxedEPSF{:Figures:OldRichDontDissave.eps}}
	\medskip\medskip\medskip\medskip
	\ifthenelse{\boolean{BookVersion}}{\caption{}}{\caption{Age Profile of Log Wealth for the 99th Percentile, SCF Data}}
	\label{fig:OldRichDontDissave}
	Source: Reproduced from Carroll~\cite{carroll:richsave}
\end{figure}



\ifthenelse{\boolean{BookVersion}}{

\vfill\eject\clearpage

\setcounter{table}{0}
\setcounter{figure}{0}

%\centerline{\bf Captions}


\begin{table}[p]
{\caption{Major Features of the Tax Code Relevant for the Rich}}
\end{table}

\begin{table}[p]
{\caption{Ownership Rates of Assets and Liabilities}}
\end{table}

\begin{table}[p]
{\caption{Composition of Net Worth}}
\end{table}

\begin{table}[p]
{\caption{Composition of Net Worth by Risk Category}}
\end{table}

\begin{table}[p]
{\caption{Degree of Diversification of Portfolio Structure}}
\end{table}

\begin{table}[p]
{\caption{Risk Bearing By Age}}
\end{table}

\begin{table}[p]
{\caption{International Comparison of Portfolio Structure}}
\end{table}

\begin{table}[p]
{\caption{Risk Aversion By Income and Net Worth, 1992 and 95 SCFs}}
\end{table}

\begin{table}[p]
{\caption{Effect of Inheritances on the Change in Risk Aversion}}
\end{table}

\begin{table}[p]
{\caption{Lack of Diversification of Business Wealth}}
\end{table}

\begin{table}[p]
{\caption{Riskiness of Financial Assets in Entrepreneurs' Portfolios}}
\end{table}

\begin{figure}[p]
{\caption{Wealth Profiles for Baseline and More Patient Households}}
\end{figure}

\begin{figure}[p]
{\caption{Age Profile of Log Wealth for the 99th Percentile, SCF Data}}
\end{figure}

\vfill\eject\clearpage

\renewcommand{\baselinestretch}{2.0} \normalsize

\def\enotesize{\normalsize}

\theendnotes

}
{
\clearpage

\begin{figure}
	\centerline{\BoxedEPSF{:Figures:Top1pctWProfilePatientvsSCF.eps}}
	\medskip\medskip\medskip\medskip
	\ifthenelse{\boolean{BookVersion}}{\caption{}}{\caption{Wealth Profiles for Baseline and More Patient Households}}
	\label{fig:Top1pctWProfilePatientvsSCF}
	Source: Reproduced from Carroll~\cite{carroll:richsave}
\end{figure}

\begin{figure}
	\centerline{\BoxedEPSF{:Figures:OldRichDontDissave.eps}}
	\medskip\medskip\medskip\medskip
	\ifthenelse{\boolean{BookVersion}}{\caption{}}{\caption{Age Profile of Log Wealth for the 99th Percentile, SCF Data}}
	\label{fig:OldRichDontDissave}
	Source: Reproduced from Carroll~\cite{carroll:richsave}
\end{figure}
}


\end{document}

