%&custom12pt
%\documentclass[12pt]{article}
%\usepackage{amsmath,latexsym,ifthen,Portable:latex:vmargin,afterpage,Portable:latex:harvard_mod,Endnotes}

\setmarginsrb{1.2in}{1.0in}{1.2in}{1.0in}{0pt}{0pt}{0pt}{.5in}
% {left}{top}{right}{bottom}

\setpapersize{USletter}

%\input setmargins.tex




\newboolean{Figures}
\setboolean{Figures}{false}
\newboolean{Tables}
\setboolean{Tables}{true}

\newboolean{ShowTrueEqns}        % Solution procedure differs from text notation
\setboolean{ShowTrueEqns}{false} % If true, this prints the formulas associated w. true solution method

\newboolean{ShowFirstStuff}
\setboolean{ShowFirstStuff}{true}

\newboolean{TilburgVersion}
\setboolean{TilburgVersion}{true}

\newboolean{FlorenceVersion}
\setboolean{FlorenceVersion}{true}

\newboolean{ModelsVersion}
\setboolean{ModelsVersion}{false}

%\pagestyle{empty}
\begin{document}

\baselineskip 18pt

\ifthenelse{\boolean{ShowFirstStuff}}{


\centerline{\Large Portfolios of the Rich}


\medskip
\centerline{Christopher D. Carroll}

\medskip
\vspace{1in}

\centerline{Abstract}

Recent research has shown that `rich' households save at much higher 
rates than others (see Carroll~\cite{carroll:richsave}; Dynan, 
Skinner, and Zeldes~\cite{dsz:richsave}; Gentry and 
Hubbard~\cite{gentry&hubbard:wealthysave}; 
Huggett~\cite{huggett:wealth}; 
Quadrini~\cite{quadrini:entrepreneurship}).  This paper documents 
another large difference between the rich and the rest of the 
population: portfolios of the rich are heavily skewed toward 
investments in their own privately held businesses.  The paper 
explores three possible explanations.  First, perhaps there is 
exogenous variation in risk tolerance which leads highly risk tolerant 
households to engage in high-risk, high-return activities, and the 
risk-lovers who are lucky end up rich.  A second possibility is that 
capital market imperfections {\it a la} Gentry and 
Hubbard~\cite{gentry&hubbard:wealthysave} and 
Quadrini~\cite{quadrini:entrepreneurship} require entrepreneurial 
activities to be largely self-financed, and these same imperfections 
imply that entreprenurial investiment will yield high average returns.  
The final possibility is that wealth enters households' utility 
functions directly as a luxury good as in 
Carroll~\cite{carroll:richsave} (one interpretation is that this 
reflects the utility of anticipated bequests).  A consequence of that 
model which was not anticipated in the original paper is its 
implication that household preferences exhibit decreasing relative 
risk aversion, which could explain why rich households are much more 
likely to engage in high-risk high-return entrepreneurial activities 
even without capital market imperfections or taste differences.  The 
paper concludes that the overall pattern of facts suggests both 
Carroll-style utility and Gentry/Hubbard-Quadrini style capital market 
imperfections are important.

\ifthenelse{\boolean{TilburgVersion}}{ 

}{
This paper presents data on the portfolio structure of rich households 
and of typical households, and compares the results to the predictions 
of a dynamic stochastic optimization model of household portfolio 
choice which nests the two hypotheses about the reasons for the 
higher saving rate of the rich.  The results lead to three main 
conclusions: 1) the ability of some `entrepreneurial' households to 
earn a higher rate of return on their investments is not sufficient by 
itself to explain the patterns in the data, because that model implies 
that even consumers who can earn a high rate of return will try to 
consume all of their wealth before they die, and there is no evidence 
of decumulation at old ages by the very rich; 2) the heterogeneity in 
saving rates at different levels of permanent income induced by 
assuming that bequests are luxury goods is also insufficient, taken 
alone, to generate the degree of wealth concentration observed at the 
top of the distribution, because the rates of return on publicly 
traded assets are not high enough to generate the extraordinary 
concentrations of wealth observed at the very top of the wealth 
distribution; 3) a model which combines the two assumptions is 
consistent with the data, because the high rate of return on 
entrepreneurial investments allows the accumulation of massive amounts 
of wealth while the `luxury' nature of bequests implies there is no 
reason to decumulate that wealth before death.
}
\vfill\eject
\clearpage

\clearpage


\noindent {\sc Reporter, to bank robber Willie Sutton:} \newline
\indent	``Mr. Sutton, why do you rob banks?''

\noindent {\sc Willie Sutton:} \newline
\indent ``That's where the money is!''


\section{Introduction}

If Willie Sutton had been an economist interested in portfolio 
behavior, he would not have been very interested in the median 
household.  Ever since the pathbreaking work of Pareto more than a 
century ago, economists have known that wealth is extremely unevenly 
distributed.  And more recently, survey data have revealed that the 
portfolio of financial and real assets of the median household (at 
least in the U.S.) is rather simple: a checking/savings account plus a 
home and mortgage, and not much else.\footnote{Bertaut and 
Starr-McCluer~\cite{b&s-m:usportfolios} find that the only kind of 
financial asset held by more than half of US households is a checking 
account.} Overwhelmingly, the data tell us that if we wish to 
understand aggregate portfolio behavior, it is critical to understand 
the behavior of the richest few percent of households, both because 
they control the bulk of aggregate wealth and because their portfolio 
behavior is much more complex than that of the typical household.

Despite the strength of these arguments, there seems to have been 
little recent academic work directed at understanding the portfolio 
behavior of wealthy households.  The goal of this paper is to provide 
a summary of the basic facts (and how the facts have changed over 
time) in a form which allows comparison of the behavior of the rich 
both with the behavior of the rest of the population in the U.S. and 
with portfolio behavior among other groups and other countries 
surveyed in this volume, and to make a preliminary attempt to discern 
the characteristics that will be required of any model which hopes to 
be consistent with the full range of observed behavior.

The principal conclusion from the empirical work will be that 
the most important way in which the portfolios of the rich differ
from those of the rest is that they are heavily weighted toward
investments in privately held entrepreneurial ventures.  This 
is particularly true of the rich who became rich without having
inherited a substantial amount of wealth.

After the empirical conclusions are presented, the paper informally 
considers how these results related to the academic literature on 
theoretical models of portfolio behavior.  The starting point for the 
theoretical analysis will be a standard stochastic version of the Life 
Cycle/Permanent Income Hypothesis model.  That model will prove 
inadequate, however, because it implies that the rich should look like 
scaled-up versions of everybody else.  They should neither have 
disproportionate wealth-to-income ratios, nor have particularly 
unusual portfolio structures.

The gist of the theoretical discussion will be to consider whether any 
of three potential modifications to the standard model might explain 
the observed combination of facts.  The first idea is that perhaps 
there is exogenous variation in risk aversion across households.  In 
that case more risk-tolerant households would take greater risks and 
on average would earn higher returns.  If owning a private business is 
the form of economic activity that offers the highest risk and highest 
return, the most risk tolerant households would gravitate toward 
entrepreneurship, and on average would end up richer.  The paper will 
argue that a defect of this story is that, taken alone, it provides no 
explanation for the strong positive empirical relationship between 
{\it ex ante} wealth and the propensity to begin entrepreneurial 
ventures.  Nor, in the absence of capital market imperfections, does 
it explain why the risk-lovers would feel the need to invest in {\it 
their own} entrepreneurial ventures.

The latter point leads to the second possibility: that the observed 
patterns are simply a consequence of capital market imperfections, as 
argued recently by Gentry and 
Hubbard~\cite{gentry&hubbard:wealthysave} and 
Quadrini~\cite{quadrini:entrepreneurship}.  Those authors argue that 
adverse selection and moral hazard problems require entrepreneurial 
enterprises to be largely self-financed.  They further assume that 
there is a minimum efficient scale for private enterprises and that 
this minimum scale is large relative to the wealth of the typical 
household.  The combination of these two assumptions can explain why 
households with low or moderate wealth do not often become 
entrepreneurs.  Furthermore, this story requires no differences in 
tastes among members of the population, and in principle can explain 
both the high saving rates of the rich and the high portfolio shares 
they hold in their own entrepreneurial ventures.  However, a problem 
with this story is that, if the standard model's assumption about 
household preferences is not changed, even the model with imperfect 
capital markets implies that as the rich get old, they eventually 
begin running down their wealth.  In contrast, empirical data reveal 
no evidence that wealthy elderly households eventually begin to run 
down their wealth.

The final possibility considered is that the model's assumption about 
the household utility function needs to be changed in a manner similar 
to that proposed by Carroll~\cite{carroll:richsave}, who simply 
assumes that wealth enters the utility function as a luxury good in a 
modified Stone-Geary form.  Since Max Weber~\cite{weber:capitalism} 
argued that a love of wealth for itself is the spirit of capitalism, 
Bakshi and Chen~\cite{bakshi&chen:spirit} and Zou~\cite{zou:spirit} 
have dubbed models of this class `capitalist spirit' models.  
Carroll~\cite{carroll:richsave} proposed this modification to the 
standard model as a way to explain the high lifetime saving rates of 
the rich, and one interpretation of the utility of wealth was that it 
represented a `joy of giving' bequest motive.  It turns out, however, 
that an unanticipated consequence of the model is that it implies 
decreasing relative risk aversion, which in turn can explain why the 
rich hold riskier portfolios than the rest of the population.  

The paper concludes that the main features of the data can probably be 
explained in a model which combines capital market imperfections of 
the kind emphasized by Gentry and Hubbard and Quadrini with a utility 
function like that postulated in Carroll~\cite{carroll:richsave}.


\section{The Data}


\subsection{Portfolios of the Rich}

U.S. survey data on the portfolios of the rich are the best in the 
world.  The 1962-63 {\it Survey of Financial Characteristics of 
Consumers} (henceforth SFCC) was the first wealth survey to heavily 
oversample the richest households.  The next comprehensive wealth 
survey was the 1983 {\it Survey of Consumer Finances}, which was 
followed by a 1989 SCF which consisted of a subsample of reinterviewed 
households from the 1983 survey along with a fresh batch of new 
households.  Since 1989 the SCF has been performed triennially (though 
with no further panel elements), with the latest survey having been 
completed in 1998 (we hope to have data from the 1998 survey in time 
for the final draft of this paper).

The availability of data spanning such a long time period opens up the 
possibility of studing how portfolios change in response to changes in 
the economic enviornment.  Before presenting the data on portfolio 
structure, therefore, we first present a summary of the taxation and 
legal changes that we might expect to have had a substantial impact on 
portfolio structure of the rich.

\subsubsection{The Tax Environment}
Table \ref{table:laws} summarizes the changes over time in the three 
aspects of taxes that are particularly important to the rich.  (For 
information on broader changes in the tax code see the paper in this 
volume by Poterba).  The first two columns show the statutory top 
marginal federal tax rate, which declined from 91 percent in 1963 to 
39.6 percent in 1993 and thereafter.  The second column shows the 
actual proportion of their incomes that were paid in taxes by the 
richest one percent of households.  In contrast to the dramatic 
decline in top marginal rates, the proportion of their incomes that 
the rich actually paid in taxes has been remarkably steady, varying 
between around 20 and 25 percent over the entire period.  This 
reflects the fact that during the era of high top marginal rates, the 
tax code was riddled with tax shelters and loopholes that made it 
possible for almost all rich people to avoid paying the confiscatory 
top marginal rates that were on the statute books.

The estate tax is also highly relevant for the rich.  The structure of 
the estate tax is rather complex, but that structure remained largely 
the same over the period in question.  The structure is as follows.  
The first \$x of an estate is free from estate taxation altogether, 
where \$x is indicated by the column of the table labelled 
`exemption.'  Above \$x, the tax rate begins at a marginal rate of $y$ 
percent and peaks at a top marginal rate of $z$ percent, where $y$ and 
$z$ are the first and second numbers in the column labelled `tax 
range.'  The exclusion for closely held businesses is essentially a 
mechanism that reduces the amount of the value of a closely-held 
business that is taxable, under the condition that the heir plans to 
`actively mange' the business rather than sell it.  The marital 
deduction indicates how much of the estate is taxed when it is 
transferred to a spouse.  The 100 percent deduction since 1985 
essentially means that estates are taxed only when both members of a 
married couple have died.

The final kind of tax that is relevant to the rich is the gift tax 
exclusion amount \$g, whose value is reported in the last column of 
the table.  This is the amount that each member of the household 
(husband and wife) can give to any individual (son, daughter, 
son-in-law, daughter-in-law, grandchildren, etc) annually without 
incurring any additional taxes for the recipient or donor.

The table shows that there have been two big changes in the taxes 
specifically relevant for the rich over the period in question: the 
large increase exemption levels for the estate tax in the early 1980s, 
and the more gradual, but cumulatively very large, decline in top 
marginal rates.

\subsubsection{Detailed Portfolio Structure}
Our statistical summary of the portfolio structure of the rich begins 
with Table \ref{table:pctown}, which provides data on the proportion 
of the rich (defined here and henceforth as the top one percent of 
households by net worth) who own any amount of various kinds of 
assets.

Perhaps the most dramatic change over time in the table is the sharp 
increase in the proportion of households with defined contribution 
pension plans.  In the 1962-63 SFCC, only 12.5 percent of the rich had 
any such account, but by 1983 the fraction had already jumped to 65.6, 
while by 1995 the fraction had reached 78.6 percent.  The low 
percentage in 1962-63 is attributable to the fact that there was 
little tax advantage to such plans until the early 1980s, when 401(k)s 
and IRAs suddenly were made available in principle to the whole 
population.  What is interesting is the speed with which rich 
households availed themselves of these new options.  Bertaut and 
Starr-McCluer~\cite{b&s-m:usportfolios} show (Table 3) that only 31 
percent of all households had acquired such accounts by the time of 
the 1983 survey.

Another notable change is that the proportion holding individual stock 
shares directly has fallen from 83.2 percent in 1962 to 65.0 percent 
in 1995, while the proportion holding mutual funds has risen from 
about 25 percent to about 45 percent.  This likely reflects a broad 
pattern in which households have increasingly decided to hold shares 
in the form of mutual funds rather than individual stocks.  This 
pattern has not been much studied by economists, although it is 
interesting because it reflects a convergence of actual behavior 
toward the recommendation of portfolio theory for diversification.

Among the other categories of assets, the largest changes are seen in 
the holdings of `other bonds' (primarily corporate bonds), which 
declined very sharply between 1962 and 1983 and fluctuated 
substantially between 1983 and 1995.  There is no obvious tax reason 
for these fluctuations.

The proportion of the richest households who have equity in a 
privately held business has fluctuated substantially over the years, 
from a low of 66.5 percent in 1962-63 to a high of 88.0 percent in 
1983.

With respect to debt holdings, the proportion of rich households with 
any debt jumped sharply between the SFCC, when it was 52.4 percent, 
and the 1983 SCF, when it was 77.9 percent, but exhibited no clear 
trend thereafter.  Among debt categories, the most striking change is 
the increase in the proportion of households with mortgage debt, from 
31.9 percent in 1962 to 52.5 percent in 1995.  This likely reflects 
the fact that mortgage interest remained tax deductible while other 
forms of debt lost their deductible status in the 1986 tax reform.

On the whole, the striking feature of this table is that the 
proportion of rich households owning various categories of assets has 
not changed greatly for most categories of assets - particularly 
considering that small sample sizes mean that there is inevitably some 
measurement error in the statistics for any particular year.\footnote{ 
One exception is `other financial assets,' which had an 88.8 percent 
owernship rate in the 1962-63 SFCC but much lower rates in the later 
surveys.  This is almost certainly because holdings of cash were 
included in this grab-bag category in the SFCC but not in the SCF's.  
In any case, the next table shows that `other finanical assets' 
constitute a trivial proportion of net worth in all surveys.}

Another useful comparison is of the rich to the rest of the 
population.  Average values for ownership shares for the nonrich over 
the five survey years are presented in the last column of the table.  
The broadest observation that can be made here is that rich households 
are more likely to own virtually every kind of asset in every time 
period.  Particularly striking is the discrepancy in the proportion 
owning equity in a privately held business, which averages about 75 
percent for the rich but only 13 percent for the rest of the 
population.  The contrast in ownership of shares in publicly
traded companies is only slightly less dramatic: 74 percent
versus 16 percent.

Table \ref{table:compnw} examines the relative weight of various kinds 
of assets in the net worth of the richest households.  The table shows 
that the shift from stocks to mutual funds was substantial, but even 
at the end of the sample in 1995, total net worth in individual shares 
remained substantially greater than that in mutual funds.  One of the 
largest shifts over time is in the role of investment real estate, 
which jumps from 7.4 percent of net worth in 1962-63 to over 20 
percent in 1983.  Investment real estate continues to constitute more 
than 20 percent of the portfolio until 1995, when its share drops to 
13.1 percent.  The jump in investment real estate between the early 
1960s and the early 1980s may reflect the prominent role of real 
estate in tax shelters until the tax reform act of 1986.  One would 
have expected a decline in investment real estate following the repeal 
of many of these tax shelters in the 1986 tax act, so it is surprising 
that no decline is manifest until 1995.

Another interesting observations from the table is the small size of 
mortgage debt (only 1.1 percent of net worth on average) despite the 
fact that more than half of the rich have positive amounts of mortgage 
debt.

Comparing the rich to the rest of the population, again perhaps the
most important difference is the importance of business equity for the
rich.  Such wealth accounts for about 40 percent of total net worth 
of the rich in 1983 and thereafter, vastly more than its share in 
the net worth of the typical household.  Other differences include
the lower total indebtedness of the rich and the much smaller 
proportion of total wealth tied up in home equity.

\subsubsection{Portfolio Structure And Portfolio Theory}

The usual theoretical analysis of portfolio allocation considers the 
optimal proportion of net worth to invest in `risky' versus `safe' 
assets.  This stylized theoretical treatment is conceptually useful 
but difficult to bring to data, because it is hard to allocate every 
asset to one of these two categories.  Table~\ref{table:riskysafe} 
reflects an effort to find a compromise between the complexity of 
actual portfolios and the simplicity of theory.

Among financial assets, there are some that are clearly safe (like 
checking, saving, and money-market accounts) and some that are clearly 
risky (like stock shares).  But other assets are harder to allocate, 
either because the item itself has an ambiguous status (like long-term 
government bonds, which are subject to inflation risk but not 
repayment risk (we hope!)) or because the asset is a composite with 
unknown proportions of risky and safe assets (like mutual funds which 
hold both stocks and government bonds).  We have therefore allocated 
all financial assets to one of four categories: Clearly safe, probably 
safe, clearly risky, and probably risky, which can of course be 
further aggregated into broad measures of safe and risky assets.  
Finally, we report unsecured debt separately because theory does not 
give much guidance about whether to subtract it from the `safe' or 
`risky' category.

We have divided nonfinancial assets into the primary residence, 
investment real estate, vehicles, and `other.'  Because equity in 
privately held business is substantially different from other kinds of 
investments, it merits its own category.

With these definitions, we can construct three definitions of risky 
assets: A `narrow' definition, which includes only financial assets 
which are rated as clearly risky; a `broad' definition, which includes 
clearly and probably risky financial assets, business equity, and 
investment real estate; and a `broadest' definition which adds even 
the `probably safe' assets.

It is apparent from the table that the rich bear dramatically more 
risk than the rest of the population.  In all five surveys the 
proportion of their portfolios that consisted of broadly risky assets 
was about 75 percent or greater, compared with an average percentage 
of only 40 for the nonrich households.  Examining the data in more 
detail reveals two key differences between the rich and the rest: the 
rich hold a much smaller proportion of their wealth in home equity (6 
percent versus 34 percent) and a much larger proportion in business 
equity and investment real estate (the sum of these two categories
is 52.1 percent for the rich versus 26.2 percent for the rest).

Another perspective on the portfolios of the rich is presented in 
Table \ref{table:diversification}, which provides a census of the 
portfolio structure of the rich along the four dimensions 
corresponding to ownership or non-ownership of clearly safe, probably 
safe, clearly risky, and probably risky assets, a total of $4^{2}=16$ 
different possibilities.  In all five survey years, a majority or 
nearly a majority of the rich held some assets in each of these four 
categories.  This is a sharp contrast to the behavior of the rest of 
the population, which is much more evenly distributed among the 16 
categories but is most heavily concentrated in the region with only 
safe assets.  (See Bertaut and Starr-McCluer for the data on the
rest of the population.)

Finally, Table \ref{table:riskbyage} presents data on ownership rates 
for risky assets by age of the household head for each of the survey 
years.\footnote{The high ownership rates for risky assets in the 1962 
SFCC are an error; we accidentally included `other financial assets' 
in the risky category, which is inappropriate because we believe that 
most of the content of `other financial assets' is holdings of cash.  
This will be fixed in a future version of the paper.} Interestingly, 
the patterns for ownership rates and for portfolio shares are 
different: The probability of owning at least some amount of risky 
assets is monotonically increasing in age, but the {\it proportion} of 
the portfolio composed of `broad risky' assets rises through the first 
three age categories (up to age 49) but falls for the 50-59 age group.  
Ownership rates of `risky' assets show a similar monotonic increase 
(at least until age 70+), while the portfolio share shows no clear 
rise with age but again does display a decline in the 50-59 age group.  
As King and Leape~\cite{king&leape:ageinfo} argue, the monotonic 
increase in ownership rates may reflect the accumulation of experience 
with different assets as the household ages.  The reduction in the 
risky share of the portfolio is interesting because it corresponds 
roughly to the common financial advice to shift assets away from risky 
forms as retirement approaches.\footnote{It is important to recall, 
however, that these figures may reflect the effects of both cohort and 
time effects as well as age effects, so the true age effects may 
differ.}


 
\section{Analysis}

Thus far we have presented data with little effort to understand the 
underlying behavior which gives rise to these data.  It is now time to 
begin trying to understand these underlying behavioral patterns.

We start by presenting a baseline formal model of saving over the life 
cycle.

\subsection{The Basic Stochastic Life Cycle Model}

The following model is what I will characterize as the basic 
stochastic life cycle model.  The consumer's goal is to solve the 
problem
\begin{eqnarray*}
	\max &  & \sum_{s=t}^{T} \beta^{s-t} \mathcal{D}_{t,s} u(C_{t})  \label{eq:maxutil} \\
\end{eqnarray*}
where $u(C)$ is a constant relative risk aversion utility function 
$u(C)=c^{1-\rho}/(1-\rho)$, $\beta$ is the (constant) geometric 
discount factor, and $\mathcal{D}_{t,s} = \prod_{h=t}^{s-1}(1-d_{h})$ 
is the probability that the consumer will not die between periods $t$ 
and $s$ ($\mathcal{D}_{t,t}$ is defined to be 1; $d_{t}$ is the 
probability of death between period $t$ and $t+1$).

The maximization is of course subject to constraints.  In particular, 
if, following Deaton~\cite{deatonLiqConstrs}, we define $X_{t}$ as 
`cash-on-hand' at time $t$, the sum of wealth and current income, then 
the consumer faces a budget constraint of the form
\begin{eqnarray*}
	X_{t+1} & = & R_{t+1}S_{t} + Y_{t+1}
\end{eqnarray*}
where $S_{t} = X_{t}-C_{t}$ is the portion of last period's resources 
the consumer did not spend, $R_{t+1}$ is the gross rate of return 
earned between $t$ and $t+1$, and $Y_{t+1}$ is the noncapital income 
the consumer earns in period $t+1$.

Assume that the consumer's noncapital income in each period is given 
by their permanent income $P_{t}$ mutiplied by a mean-one transitory 
shock, $E_{t}[\epsilon_{t+1}] = 1$, and assume that permanent income 
grows at rate $G_{t}$ between periods, but is also buffeted by a 
mean-one shock, $P_{t+1} = G_{t+1}P_{t}\eta_{t+1}$ such that $E_{t} 
[\eta_{t+1}]=1$.

Given these assumptions, the consumer's choices are influenced by
only two state variables at a given point in time: the level of 
the consumer's assets $X_{t}$ and the level of permanent income, $P_{t}$.
As usual, the problem can be rewritten in recursive form with a value
function $V_{t}(X_{t},P_{t})$.  Written out fully in this form,
the consumer's problem is

\begin{eqnarray}
	V_{t}(X_{t},P_{t}) & =& \max_{\{C_{t},w_{s,t}\}} 
	u(C_{t}) + \beta \mathcal{D}_{t,t+1} E_{t}\left[V_{t+1}(X_{t+1},P_{t+1}) \right] \nonumber\\ 
	 & \mbox{such that} &   	\label{eq:bellmanstd}  \\
	S_{t}   & = & X_{t}-C_{t} \nonumber \\
	X_{t+1} & = & R_{t+1}S_{t} + Y_{t+1} \nonumber \\
	Y_{t+1} & = & P_{t+1}\epsilon_{t+1} \nonumber \\
	P_{t+1} & = & G_{t} P_{t} \eta_{t+1} \nonumber \\
\end{eqnarray}

\subsection{The Saving Behavior of the Rich}


Within the last decade, advances in computer speed and numerical 
methods have finally allowed economists to solve life cycle 
consumption/saving problems like that presented above with serious 
uncertainty and realistic utility (see, in particular, Hubbard, 
Skinner, and Zeldes~\cite{hsz:importance}; 
Huggett~\cite{huggett:wealth}; Carroll~\cite{carroll:bslcpih}; and the 
references therein).  I have argued elsewhere 
(Carroll~\cite{carroll:bslcpih}) that the implications of these models 
fit the available evidence on the consumption/saving behavior of the 
typical household reasonably well, certainly much better than the old 
Certainty Equivalent (CEQ) models did.

However, another finding from this line of research has been that the
model is unable to account for the very high concentrations of wealth
at the top of the distribution.  

\subsubsection{How Rich Are They?}

Very.

Figure~\ref{fig:Top1pctWProfilePatientvsSCF} shows the ratio of wealth 
to `permanent' income\footnote{SCF respondents are asked whether their 
total income this year was above normal, about normal, or below 
normal.  Following Friedman~\cite{friedmanATheory}, I define 
permanent income as the level of income the household would normally 
receive.} by age for the population as a whole and for the households 
in the richest one percent by age category from the 1992 and 1995 
SCFs.  Also plotted for comparison is the level of the wealth to 
income ratio at the top 1 percent implied by a standard life cycle 
model of saving similar to that in Carroll~\cite{carroll:bslcpih} or 
Hubbard, Skinner, and Zeldes~\cite{hsz:importance}.  (Specifically, it 
is the Carroll model with HSZ `baseline' parameter values).  The 
richest one percent are much richer than implied by the life cycle 
model.  In addition, the figure plots the age profile of the 99th 
percentile that would be implied by the HSZ model if it were assumed 
that consumers do not discount future utility at all.  The figure 
shows that even with such patient consumers, the model remains far 
short of predicting the observed wealth to income ratios at the 99th 
percentile.\footnote{This figure is reproduced from 
Carroll~\cite{carroll:richsave}.}  

This finding is reconfirmed in a recent paper by Engen, Gale, and 
Uccello~\cite{egu:adequacy}, who do a very careful job of modelling 
pension arrangements, tax issues, and other institutional details 
neglected in the exercise above and also find that the 
wealth-to-income ratios at the top part of the income distribution are 
much greater than predicted by a life cycle dynamic stochastic 
optimization model, even with a time preference rate of zero.

\subsubsection{How Do They Spend It All?}

They don't.  

In the 1989, 1992, and 1995 SCFs, households were asked whether their 
spending usually exceeds their income, and whether their spending 
exceeded their income in the previous year.  In order to run down 
their wealth, obviously households obviously must spend more than 
their income.  Yet only five percent of the rich elderly households in 
the SCF answered that their spending usually exceeded their income.

More evidence is presented in Figure~\ref{fig:OldRichDontDissave}, 
which shows the levels of wealth by age for the elderly in the 1992 
and 1995 SCFs.  There is no evidence in this figure that wealth is 
declining for this population; indeed, if anything it seems to be 
increasing.\footnote{This is in effect a smoothed profile of wealth by 
age adjusted for cohort effects; see Carroll~\cite{carroll:richsave} 
for methodological details.} This is consistent with the answers that 
the rich elderly give to such questions as whether they have spent 
more than their income in the previous year or whether they usually 
spend more than their incomes.  Virtually none of the rich report 
spending more than their incomes, either in the previous year or on a 
regular basis.

The implication is that most of the wealth which we observe them 
holding will still be around at death.  This is obviously a
problem for any model in which the only purpose in saving is 
to provide for one's own future consumption.  

This crude evidence is backed up by a study by Auten and 
Joulfaian~\cite{auten&joulfaian:charitable} which finds that the 
elasticity of bequests with respect to lifetime resources is well in 
excess of one (their point estimate is 1.3).  See 
Carroll~\cite{carroll:richsave} for a summary of further evidence 
that, far from spending their wealth down, the rich elderly continue 
to save.


\subsection{Adding Portfolio Choice}

Recently, a wave of papers (Haliassos and 
Bertaut~\cite{haliassos&bertaut:fewholdstocks}, 
Fratantoni~\cite{fratantoni:equitypremium},Gakidis~\cite{gakidis:stocksforold}; 
Cocco, Gomes, and Maenhout~\cite{cgm:lcportfolio}; and 
Hochgurtel~\cite{hochgurtel:bufferportfolio}) has examined the 
predictions of these kinds of models when consumers facing labor 
income risk are allowed to choose freely between investing in a 
low-return safe asset like the one considered in the earlier models 
and investing in risky assets parameterized to resemble the returns 
yielded by equity investments in the past.

The only modification to the formal optimization problem presented 
above necessary to allow portfolio choice is to designate $R_{t+1}$ as 
the portfolio-weighted return, which will depend on the proportion of 
the portfolio that is allocated to the safe and the risky assets, and 
on the rate of return on the risky asset between $t$ and $t+1$.  Call 
the proportion of the portfolio invested in the risky asset (`stocks') 
$w_{s,t}$ (where $w$ is mnemonic for the portfolio `weight'), and 
$(1-w_{s,t})$ is the portion invested in the safe asset.  If the 
return on stocks between $t$ and $t+1$ is $R_{s,t+1}$, the 
portfolio-weighted return on the consumer's savings will be 
$R(1-w_{s,t})+R_{s,t}w_{s,t}$.

However, even without solving a model of this type formally, it is 
clear that such models will not be able to explain the empirical 
differences between the portfolio behavior of the rich and the 
behavior of the rest of the population, because when the utility
function is in the CRRA class problems of this type are homothetic.
That is, there is no systematic difference in the behavior of 
consumers at different levels of lifetime permanent income.  
Hence, such models provide no means to explain the very large
differences in saving and portfolio behavior documented above.

\subsection{Three Possible Modifications}

There are at least three ways one might consider modifying the model 
in hopes that the modified model might be able to explain the 
nonhomotheticity of behavior.

\subsubsection{Heterogeneity in Risk Tolerance}

The first is simply to allow for heterogeneity in risk tolerance
across members of the population.  Formally, rather than assuming
that all consumers have the same value of $\rho$, the coefficient
of relative risk aversion, we can assume that each consumer has 
an idiosyncratic, specific $\rho_{i}$.  

The effect of this would be to allow consumers with low values of 
$\rho$ (high risk tolerance) to choose highly risky but high-return 
portfolios.  On average, the risk-tolerant consumers would be rewarded 
with higher returns and would therefore end up richer than the rest of 
the population.  Thus, the rich would be disproportionately 
rich-lovers, and would therefore have riskier portfolios than the 
rest.  As shorthand, I will call this the `preference heterogeneity'
story henceforth.

\subsubsection{Capital Market Imperfections}

A second possibility is to follow Gentry and 
Hubbard~\cite{gentry&hubbard:wealthysave} and 
Quadrini~\cite{quadrini:entrepreneurship} in assuming that there are 
important imperfections in capital markets which 1) require 
entrepreneurial investment to be largely self-financed; 2) imply that 
entrepreneurial investment has a higher return than investments made 
on open capital markets; and 3) require a large minimum scale of 
investment.  As those authors show, the combination of these three 
assumptions can yield an implication that portfolios of higher 
wealth or higher income consumers will be much more heavily weighted 
toward entrepreneurial investments, and that rich consumers with
business equity have higher than average saving rates (because 
of the high returns that are available to them).  I will refer
to this theory as the `capital market imperfections' story.

\subsubsection{Bequests as a Luxury Good}

A final possibility is to change the assumption about the lifetime 
utility function.  Carroll~\cite{carroll:richsave} proposes adding a 
bequest motive of the form $B(S)$ takes a modified Stone-Geary form,
\begin{eqnarray*}
B(S) &= &\frac{(S + \gamma)^{1-\alpha}}{1-\alpha}.
\end{eqnarray*}

Carroll~\cite{carroll:richsave} shows that if one assumes that 
$\alpha<\rho$ then wealth will be a `luxury good' in the sense that as 
lifetime resources rise, a larger proportion of those resources is 
devoted to $S_{T}$.  In the limit as lifetime resources approach 
infinity, the proportion of resources devoted to the bequest 
approaches 1.  The other salient feature of the model is that if 
$\lambda > 0$ there will be a `cutoff' level of lifetime resources 
such that consumers poorer than the cutoff will leave no bequest at 
all.  Thus the model is capable of matching the crude stylized fact 
that low-income people tend to leave little or no bequests, and also 
captures the fact (from Auten and 
Joulfaian~\cite{auten&joulfaian:charitable}) that among those who 
leave bequests, the elasticity of lifetime bequests with respect to 
lifetime income is greater than one.

In this paper the assumption is that one receives utility from the 
contemplation of the potential bequest in proportion to the 
probability that death (and the bequest) will occur.  Thus Bellman's 
equation would be modified to:
\begin{eqnarray}
	V_{t}(X_{t},P_{t}) & =& \max_{\{C_{t},w_{s,t}\}} 
	u(C_{t}) + \beta (1-d_{t}) E_{t}\left[V_{t+1}(X_{t+1},P_{t+1}) \right] + d_{t} B(S_{t})\nonumber\\ 
\end{eqnarray}
and the transition equations for the state variables are unchanged.

While it is obvious how this model might help to explain the high 
saving rates of the rich, it is not so obvious why it might help 
explain the high degree of riskiniess of their portfolios.  It turns 
out, however, that precisely the same assumption which implies that 
bequests are a luxury good {\it also implies that people are less 
risk-averse with respect to gambles over bequests than with respect to 
gambles over consumption.}  That assumption is that the exponent
on the utility from bequests function $\alpha$ must be less than
the exponent on the utility from consumption $\rho$.  This implies
that the marginal utility from bequests declines more slowly than
the marginal utility from consumption and thus as wealth rises more
and more of it is devoted to bequests rather than consumption.
However, the traditional interpretation of exponents like $\rho$
and $\alpha$ in utility functions of this class is as coefficients
of relative risk aversion, so the implication of less risk aversion
with respect to bequest gambles than consumption gambles is 
an immediate implication of the assumption that bequests are 
a luxury good!

Max Weber~\cite{weber:capitalism} argued that the `spirit of 
capitalism' was the pursuit of wealth for its own sake.  Following Max 
Weber~\cite{weber:capitalism} as interpreted by Zou~\cite{zou:spirit} 
and Bakshi and Chen~\cite{bakshi&chen:spirit}, I call this the 
``Capitalist Spirit'' model.

\subsection{Distinguishing the Three Models}

The princial unfinished empirical work of this paper is to provide
empirical evidence which can distinguish between the three theories
outlined above, all of which can explain the basic facts that the
portfolios of the rich are riskier than those of the rest and that
investment in closely-held businesses are a disproportionate 
share of the portfolios of the rich.

Beginning with the heterogeneous preferences theory, it is useful to 
flesh out some of its implicit assumptions.  In order to explain the 
disproportionate share of business equity in the portfolios of the 
richest households, it is necessary to assume that business 
investments bear the highest risk and the highest return among the 
categories of assets summarized in the tables.  This assumption is 
plausible and therefore not problematic.  What {\it is} problematic 
for a theory based purely on preference heterogeneity is the tendency 
of the rich to invest mainly in {\it their own} entrepreneurial 
ventures.  We do not have a table demonstrating this point yet (there 
will be one in the next version), but from examining the raw data it 
is clear that the great majority of the wealth in closely-held 
businesses is in businesses in which the household has an `active 
management role' to use the terms of the survey.

This may not be surprising intuitively either, but upon reflection it 
is clear that it implies some form of capital market imperfection.  If 
there were no such imperfections, the optimal strategy would be to 
hold only a tiny share in one's own entrepreneurial venture and 
similarly small shares in everyone else's entrepreneurial ventures in 
order to diversify the idiosyncratic risk.  Thus the preference 
heterogeneity story alone cannot explain the pattern of facts.

A further problem with the preference heterogeneity story is that it 
cannot explain the positive relationship between the level of labor 
income and the propensity to start entrepreneurial ventures.  Table 
\ref{table:8389transitions} shows that entry into entrepreneurship is 
strongly correlated with the initial level of labor income.  This is 
also true for young consumers who have not accumulated much net worth 
(not shown).  Unless there is some compelling reason to assume that 
people who are more risk tolerant also earn higher labor income, the 
pure preference heterogeneity story cannot explain this correlation.  
(In fact, one might suppose that {\it ceteris paribus}, people who are 
more risk averse would have higher labor income because the level and 
the log standard deviation of permanent labor income are negatively 
related (Carroll and Samwick~\cite{carroll&samwick:nature}).  In 
particular, people with higher levels of education have higher and 
less variable incomes, so people who are highly risk averse should 
choose to get more education, which would induce a negative, not a 
positive, correlation between the level of labor income and risk 
tolerance.

A final problem with the preference heterogeneity story is that it 
provides no explanation for the failure of the elderly rich to spend 
down their assets.  Indeed, because risk tolerance is positively 
correlated with the intertemporal elasticity of substitution in models 
with time-separable preferences, we should actually expect the rich to 
be running down their wealth {\it faster} than the non-rich if the 
only differences between the rich and non-rich is in their degree of 
risk tolerance.

Given that the preference heterogeneity story implicitly also requires 
some form of capital market imperfections in order to explain the 
data, it is interesting to examine whether capital market 
imperfections alone might work.

The central requirement of any story based on capital market 
imperfections is that business ownership must yield higher-than-market 
rates of return.  We have done considerable research in the economic 
and business literatures to attempt to find a credible estimate of the 
rate of return on closely-held business ventures, but have found 
nothing usable for our purposes.  We intend in the next draft of the 
paper to examine this by checking whether wealth accumulation is 
greater on average over the 1983-89 panel for households who are 
business owners in 1983.

Suppose for the moment that we accept on faith the proposition that 
closely-held business ventures earn higher rates of return (in 
exchange for higher risk) than is available on open capital markets, 
and that such ventures must be self-financed for moral hazard or 
adverse selection reasons.  By themselves, and in the absence of 
preference heterogeneity, these assumptions cannot explain the 
correlation between the level of initial labor income or initial 
wealth and the propensity to start businesses documented in table 
\ref{table:8389transitions}.  As anyone who has read the fine novel 
{\it A Confederacy of Dunces} knows, there are principal/agent and 
moral hazard problems even for a hot dog stand vendor, so the same 
logic that leads to the conclusion that other entrepreneurial ventures 
should yield a high rate of return should apply in this context as 
well.  Gentry and Hubbard~\cite{gentry&hubbard:wealthysave} address 
this problem by simply assuming that there is a minimum efficient 
scale for business enterprises which is large relative to the 
resources of the median household, but this approah is unsatisfactory 
because the richest households would have wealth vastly greater than 
the minimum efficient scale and therefore would have no need to tie up 
more than a trivial fraction of their total net worth in the business 
enterprise.  Quadrini~\cite{quadrini:entrepreneurship} deals with this 
problem by postulating a complicated `ladder' of business 
opportunities at rising minimum efficient scales.

Even if we were to accept the story that there is a complicated 
ladder of minimum efficient scales of business operation {\it a la}
Quadrini, the capital market imperfections story still has a big 
problem:  once again, it provides no explanation for the failure of
the rich elderly to begin running down their wealth.  Again, if
we accept the standard model's assumption that the only purpose of
saving is to finance one's own future consumption, there is simply
no way to explain the failure of the elderly to eventually spend.

An alternative explanation for the observation that entry rates into 
entrepreneurship are strongly related to {\it ex ante} wealth is the 
possibility that the rich are less risk averse than the rest of us.  
Some evidence on this is presented in Table~\ref{table:riskaver}.  The 
table shows that occupants of the highest `permanent' income and net 
worth brackets are notably more likely to express a great willingness 
to accept above-average risk in exchange for above-average returns.  
Even more dramatic is the difference between the proportion of the 
rich and of the rest who express themselves as `not willing to take 
any financial risks.'  Among the richest 1 percent by wealth, less 
than ten percent express such extreme risk aversion; among the bottom 
80 percent, nearly half express this sentiment.

The simplest interpretation of this table, of course, would be in
terms of the heterogeneous preferences story outlined above.  But,
as we have already noted, that story has several problems of its own.
Note, however, that the results in this table are precisely what
would be expected if the `capitalist spirit' model is correct:
because the rich have sufficient wealth that they plan to leave 
a large proportion as a bequest, their risk aversion is lower 
than that of others, just as indicated in this table.  
We intend in the revised version of this test to use the 83-89
panel data to attempt to provide some direct evidence on the 
direction of the causality between the level of wealth or 
income and risk preferences.  


\section{Conclusions}

[To be completed].



\vfill\eject

\bibliographystyle{Portable:latex:econometrica}
\bibliography{Portable:latex:economics}

\vfill\eject

}{} % End ifthenelse{\boolean{ShowFirstStuff}}

\ifthenelse{\boolean{Tables}}{

\baselineskip 12pt

\begin{table}
\caption{Major Features of the Tax Code Relevant for the Rich}
\label{table:laws}\vspace{.3in}
\centerline{\BoxedEPSF{:Tables:laws.eps}}
\end{table}

\clearpage

\begin{table}
\caption{Ownership Rates of Assets and Liabilities}
\label{table:pctown}\vspace{.3in}
\centerline{\BoxedEPSF{:Tables:pctown.eps}}
\end{table}

\clearpage

\begin{table}
\caption{Composition of Net Worth}
\label{table:compnw}\vspace{.3in}
\centerline{\BoxedEPSF{:Tables:compnw.eps}}
\end{table}

\clearpage

\begin{table}
\caption{Composition of Net Worth by Risk Category}
\label{table:riskysafe}\vspace{.3in}
\centerline{\BoxedEPSF{:Tables:riskysafe.eps}}
\end{table}

\clearpage

\begin{table}
\caption{Degree of Diversification of Portfolio Structure}
\label{table:diversification}\vspace{.3in}
\centerline{\BoxedEPSF{:Tables:diversification.eps}}

\vspace{.15in}
\noindent Note: A description of the asset classifications appears in 
the notes at the end of Table \ref{table:riskysafe}.

\noindent Source: Survey of Financial Characteristics of Consumers and Surveys of Consumer Finances
\end{table}


\clearpage

\begin{table}
\caption{Risk Bearing By Age}
\label{table:riskbyage}\vspace{.3in}
\centerline{\BoxedEPSF{:Tables:riskbyage.eps}}
\vspace{.05in}
\noindent Notes: The definition of risky financial assets corresponds 
to the sum of clearly risky and probably risky assets defined in Table 
\ref{table:riskysafe}.  The definition of broad risky assets 
corresponds to the `risky assets - broad' classification in Table 
\ref{table:riskysafe}.
\medskip

\noindent Source: Survey of Financial Characteristics of Consumers and Surveys of Consumer Finances
\end{table}


\clearpage

\begin{table}
\caption{Exit and Entry Rates for Business Ownership with HH Head Age 30-60}
\label{table:8389transitions}\vspace{.3in}
\centerline{\BoxedEPSF{:Tables:8389transitions.eps}}
\vspace{.05in}
\medskip
\end{table}

\clearpage

\input :Tables:riskaver

\clearpage\clearpage


\centerline{\Large Detailed Definitions and Notes for Tables}

\vspace{.2in}

\noindent {\bf Table~\ref{table:riskysafe}}

Calculations of probably risky and probably safe mutual funds and 
defined contribution pensions are as follows.

\begin{itemize}

\item[1962 SFCC] Due to the lack of information on mutual fund 
investment strategies, all mutual funds are classified as risky and 
all defined contribution pensions are classified as safe.

\item[1983 SCF] Probably safe mutual funds include tax-free mutual 
funds, while probably risky mutual funds include taxable mutual funds.  
The calculation of probably risky and probably safe defined 
contribution pensions uses the institution that held the IRA/Keogh 
accounts as a proxy for investment direction.  If a real estate 
investment company held the accounts, then those defined contribution 
pensions were considered probably risky.  If a commercial bank, 
savings and loan, or credit union held the accounts, then those assets 
were considered probably safe.  The defined contribution pensions were 
split between the two categories if a brokerage, insurance company, 
employer, school/college/university, investment management company, or 
the AARP held the accounts.  In the case that the household had no 
IRA/Keogh accounts, but had a thrift pension account, the assets were 
considered probably safe.

\item[1989-1995 SCF] Probably safe mutual funds include tax-free bond 
mutual funds, plus government bond mutual funds, plus other bond 
mutual funds, plus half of the combination mutual funds.  While 
probably risky mutual funds include stock mutual funds, plus half of 
the combination mutual funds.  The calculation of probably risky and 
probably safe defined contribution pensions used the question about 
the investment direction of IRA/Keogh and thrift-type pension 
accounts.  Defined contribution pensions were considered probably 
risky if IRA/Keogh and thrift accounts were invested in stock, mutual 
funds, commodities, real estate, or limited partnerships.  Defined 
contribution pensions were considered probably safe if IRA/Keogh and 
thrift accounts were invested in certificates of deposit, bank 
accounts, bonds, annuities, or insurance products.  When the 
investment direction consisted of combinations of the above 
categories, half of the values of the IRA/Keogh and thrift accounts 
were allotted to each type of defined contribution pension.

\end{itemize}


}

\clearpage

\begin{figure}[tbp]
	\centerline{\BoxedEPSF{:Figures:Top1pctWProfilePatientvsSCF.eps}}
	\medskip\medskip\medskip\medskip
	\caption{Wealth Profiles for Baseline and More Patient Consumers}
	\label{fig:Top1pctWProfilePatientvsSCF}
\end{figure}

\begin{figure}[p]
	\centerline{\BoxedEPSF{:Figures:OldRichDontDissave.eps}}
	\medskip\medskip\medskip\medskip
	\caption{Age Profile of Log Wealth for the 99th Percentile, SCF Data}
	\label{fig:OldRichDontDissave}
\end{figure}


\end{document}




These studies generally also conclude that the standard Life 
Cycle/Permanent Income Hypothesis model of saving, which assumes that 
the only purpose of saving is to provide for future consumption, 
cannot explain why saving rates of the rich are so much higher than 
those of the rest of the population.  Carroll~\cite{carroll:richsave} 
proposes that the higher saving rate of the rich can be explained if 
the standard assumption about utility is modified to allow the 
treatment of bequests or wealth as `luxury goods'; Gentry and 
Hubbard~\cite{gentry&hubbard:wealthysave} and 
Quadrini~\cite{quadrini:entrepreneurship} propose that the explanation 
is that entrepreneurs can earn a higher rate of return on their 
investments than is available in anonymous capital markets.

\newcommand{\@makecap2}[2]{
  \vspace{10pt}\sbox{\tempbox}{#1: #2}
  \ifthenelse{\lengthtest{\wd\tempbox > \linewidth}%
    { #1: #2\par}%
  {\begin{center}#1:  #2\end{center}}%
}}

\newcommand{\@makecap2}[2]{
  \vspace{10pt}\sbox{\tempbox}{#1: #2}
  \ifthenelse{\lengthtest{\wd\tempbox > \linewidth}%
    { #1: #2\par}%
  {\begin{center}#1:  #2\end{center}}%
}}




\subsection{Portfolios of the Rich, Redux}

Table~\ref{table:compareportfolios} presents information on the 
portfolio allocations between five main categories of assets for 
various subsets of the population.  The five categories are condensed 
from the categories laid out in Table~\ref{table:riskysafe}: Risky
is the sum of clearly and probably risky, Safe is the sum of 
clearly and probably safe, Business is the sum of equity in 
privately held businesses and the net value of investment 
real estate, and Miscellaneous absorbs all the other categories
which did not fit into the foregoing scheme (primarily consumer
installment debt).  

\ifthenelse{\boolean{Tables}}{
\begin{figure}
	\centerline{\BoxedEPSF{:Tables:compareportfolios.eps}}
	\medskip\medskip\medskip\medskip
	\label{table:compareportfolios}
\end{figure}
}

As noted above but seen more clearly here, the principal difference 
between the portfolio allocations of the rich and the rest comes in 
allocations between net worth held in businesses (42 percent for the 
rich, 18 percent for the rest) and that held in home equity (only 6 
percent for the rich, almost 40 percent for the rest).  If we lump 
business assets with the `risky' category to form a `total risky 
assets' category, the contrast between the rich and the rest is stark.  
For the rich, the `total risky' proportion of the portfolio is 72 
percent; for the rest, that proportion is 43 percent.  Furthermore, 
these figures almost certainly represent an {\it underestimate} of the 
extra risk-bearing by the rich, because they ignore the fact that much 
of the `labor' income of the rich is directly tied to the success of 
their business enterprise.  These patterns are remarkably steady 
across decades and surveys, and this stability suggests suggests that 
it is not hopeless to search for general theoretical principles to 
explain them.

\ifthenelse{\boolean{FlorenceVersion}}{

Within the population of the rich, it is useful to compare those who 
are business owners to those who are not.  Under the assumption 
(henceforth maintained) that private business equity is riskier than 
any other form of asset, theory would suggest that if people do not 
differ in their intrinsic levels of risk aversion, among people with 
comparable levels of overall net worth the proportion of the 
remainder of their portfolio that is invested in risky assets would be 
lower.  This proposition receives little support: business owners hold 
18 percent of their portfolios in safe and 24 percent in risky assets, 
for a ratio of 18/24 = .75.  For the rich as a whole, the ratio is 
22/30 = .73.

A natural hypothesis is that the business owners are more risk 
tolerant than non-business owners.  Since, as we noted above, most 
of the non-business owners who are rich have received inheritances, 
the natural hypothesis is that the inheritors are more risk-averse 
than the non-inheritors.  This can be checked by comparing the next 
four panels of the table.  Among business owners who received an 
inheritance, the proportion of safe to risky is 17/28 = 61 percent.  
Among the non-inheritors, that ratio is only 11/24 = 46 percent.  
Among the non-business owners, the contrast is even more dramatic: for 
the inheritors, the ratio of safe to risky is 39/47 = 83 percent, 
while for the non-inheritors that ratio is 24/53=47 percent.
}


\begin{table}[tbp]
\center
\begin{tabular}[c]{lccc}
          & Spending  & Spending  \\ 
  	      &  Usually   & Exceeded  \\
   	      &  Exceeds   & Income    \\
   	      &  Income    & this year \\ \hline
          &  .05    & .23       \\ 
\end{tabular}
	\caption{Saving By the Wealthy Elderly With and Without Children}
	\protect\label{table:SavOfRichOld}
\end{table}



\subsubsection{How Did They Get So Rich?}

Conceptually, there are two possible sources for the wealth that any 
household commands at any point in time: it is either the result of 
their own past saving accumulated at some rate of return, or the 
result of an inheritance, also accumulated at some rate of return 
since the time of receipt.  In practice it is very difficult to 
decompose wealth into these two components, because any such 
decomposition requires a complete model of how household 
consumption/saving decisions are made (because the amount of wealth 
`remaining' from an inheritance depends on the proportion of that 
inheritance that was spent in every period between its receipt and the 
present).

Wealth to `permanent' income ratios at the top of the distribution are 
so extremely high that it is tempting to conclude that they must 
largely reflect inheritances - particularly for the younger households 
who seemingly have not had time to accumulate much wealth by saving 
it.  Table~\ref{table:howgotrich} shows the proportion of the richest 
households in each age group who reported ever having received an 
inheritance in the 1989, 1992, and 1995 SCFs (unfortunately, the 
questions about inheritances on earlier SCFs were quite different and 
are more difficult to interpret, so they are not reported).  While it 
is true that the proportion of rich households who have received 
inheritances is more than twice the proportion for the rest of the 
population, in all three surveys substantially less than half of the 
richest households admitted to ever having received an inheritance.  
This holds true even for young rich households.

If they didn't inherit it, they must have made it themselves.  The 
next panel of the table suggests how: there is a dramatic contrast 
between the entrepreneurial activities of the rich and those of the 
general population.  Following Gentry and Hubbard, we define a 
household as a `business owner' if it reported at least \$5,000 of net 
equity (1992 dollars, consistent across surveys) in a business venture 
in which the household had an `active' management role.\footnote{This 
definition is different from the one employed in Table 
\ref{table:pctown}: in that table, the household was counted as owning 
business equity if it had any positive amount of business equity, 
whether the owner was passive or active.} Among the general 
population, only between 10 and 15 percent of households are business 
owners by this definition.  But among the richest households, the 
proportion who are business owners is between 65 and 80 percent in the 
younger age bracket and between 70 and 80 percent in the older 
bracket.

The remainder of the table breaks down the population and the top one 
percent by the joint characteristics of business owner and inheritor.  
The question of greatest interest is whether it is possible to get 
rich without either inheriting or starting a business.  Roughly 
speaking, the answer is no.  For the younger age group, only about 15 
percent of the richest one percent had neither received an inheritance 
nor was actively engaged in running a privately held business 
(compared to around 75 percent of the general population who had done 
neither).  For the older age group, the proportion of rich non-owner 
non-inheritors averages less than ten percent.  Furthermore, these 
figures reflect only entrepreneurial activity at the time of the 
survey, so these figures likely represent an upper bound on the 
proportion who have gotten rich by means other than entrepreneurship
and inheritance.

The foregoing discussion suggested that for those not lucky enough to 
inherit wealth, almost the only road to riches is to start a 
successful business.  The implicit story is essentially that in 
Quadrini~\cite{quadrini:entrepreneurship} and Gentry and 
Hubbard~\cite{gentry&hubbard:wealthysave}: entreprenuership earns 
considerably above-market returns (in exchange for high risk) because 
there are capital market imperfections (such as adverse selection or 
moral hazard) which mandate that entrepreneurial ventures must be 
financed in large part by the entrepreneur's personal wealth.  Call 
this the risk-and-return story.

There is an alternative interpretation that is equally consistent with 
the data presented thus far: once people become rich (never mind how), 
they find the joys of entrepreneurship so enticing that they are 
willing to devote a large portion of their wealth to entrepreneurial 
ventures, even if those ventures do {\it not} earn higher-than-market 
returns; indeed, they could even earn below-market returns if the 
prospect of being one's own boss were sufficiently tantalizing to be 
worth a financial sacrifice.  Call this the business-as-luxury story.

The two hypotheses can only be distinguished by examining the {\it 
dynamics} of entrepreneurship and wealth.  The risk-and-return story 
implies that much of the wealth of the rich was accumulated {\it 
after} they became entrepreneurs, and as a consequence of that 
entrepreneurship; the business-as-luxury story would imply that wealth 
causes (and perhaps Granger-causes) entrepreneurship rather than the 
other way around.

Gentry and Hubbard provide considerable evidence for the 
risk-and-return story using data from the 1983-1989 panel component of 
the SCFs.  Given that there are two possibilities for entrepreneurship 
status in each of the two years, they divide their sample into four 
types of households: those that are entrepreneurs in both years 
(STAYIN); those that are entrepreneurs in 1983 but not 1989 (EXIT); 
those that are not entrepreneurs in 1983 but are in 1989 (ENTER); and 
those that are not entrepreneurs in either survey (STAYOUT).

Gentry and Hubbard show the following:

\begin{enumerate}

\item The level of wealth and the wealth-to-income ratios of both the 
ENTER and the STAYIN groups increase much more than the level of wealth
or the wealth-to-income ratio of the STAYOUT group

\item The EXIT group experiences a substantial {\it decline} in net worth

\item Controlling for demographic characteristics, the ranking of the 
groups by rate of accumulation is $\{$STAYIN,ENTER,STAYOUT,EXIT$\}$.

\item The probability that 1983 nonentrepreneurs would become 
entrepreneurs by 1989 was much greater for those with high 1983 
wealth.

\end{enumerate}

One unfortunate omission in the Gentry and Hubbard paper is any 
information on the proportion of households in each of these groups.  
While the information they report rather compellingly shows that those 
who exit lose a lot of money, there is no information on the {\it 
proportion} of 1983 entrepreneurs who exited by 1989.

Quadrini does present some useful statistics on exit rates.  He finds 
that the probability of exit is almost 50 percent in the first year of 
an entreprenurial venture (corresponding nicely with the anecdotal 
rule-of-thumb that half of all new businesses fail in their first 
year), but the exit rate declines to 30 percent in the second year and 
is only 13 percent for entrepreneurs whose business has survived for 
three years or more.\footnote{Quadrini also presents evidence similar 
to the Gentry/Hubbard evidence on the ranking of wealth and 
wealth/income transitions among the four classes of households with 
respect to entrepreneurial status.} Thus, it would appear that the 
risk of exit is quite large, and remains substantial even for 
relatively established entrepreneurs.

Both papers present evidence that entry rates are substantially higher 
for households with higher initial net worth, and both implicitly or 
explicitly endorse the idea that this suggests that there is some 
minimum efficient scale of business operation that is large relative 
to the net worth or income of the typical household.  It is not 
entirely clear that this assumption is justified.  The capital 
required to operate a hot dog stand is not large even relative to the 
meager net worth of the median houseold.  As anyone who has read the 
fine novel {\it A Confederacy of Dunces} knows, there are 
principal/agent and moral hazard problems even for a hot dog stand 
vendor, so the same logic that leads to the conclusion that other 
entrepreneurial ventures should yield a high rate of return should 
apply in this context as well.




An alternative explanation for the observation that entry rates into 
entrepreneurship are strongly related to {\it ex ante} wealth is the 
possibility that the rich are less risk averse than the rest of us.  
Some evidence on this is presented in Table~\ref{table:riskaver}.


The table shows that occupants of the highest `permanent' income and 
net worth brackets are notably more likely to express a great 
willingness to accept above-average risk in exchange for above-average 
returns.  Even more dramatic is the difference between the proportion 
of the rich and of the rest who express themselves as `not willing to 
take any financial risks.'  Among the richest 1 percent by wealth, 
less than ten percent express such extreme risk aversion; among the 
bottom 80 percent, nearly half express this sentiment.

The table raises an obvious causality problem, however, especially 
with respect to the rankings by level of wealth - maybe most of the 
people in the top net worth brackets got there because they started 
out no richer than anybody else to begin with, but were less risk 
averse and therefore took high risks and reaped great rewards.  
Distinguishing this possibility from the more straightforward 
interpretation (that higher wealth and income cause, rather than are 
caused by, low risk averison) will be a chief mission of the empricial 
work in the next version of this paper.

Unfortunately, neither Gentry and Hubbard nor Quadrini present 
evidence on entry rates into entrepreneurship by initial level of 
income.  The question of whether high-income households are also more 
likely to become entrepreneurs will be addressed in the later version 
of this paper.




\ifthenelse{\boolean{ModelsVersion}}{
\vfill\clearpage
\section{Four Models}

\subsection{Description of the Models}

\subsubsection{The Basic Stochastic Life Cycle Model with Portfolio Choice}

Within the last decade, advances in computer speed and numerical 
methods have finally allowed economists to solve life cycle 
consumption/saving problems with serious uncertainty and realistic 
utility (see, in particular, Hubbard, Skinner, and 
Zeldes~\cite{hsz:importance}; Huggett~\cite{huggett:wealth}; 
Carroll~\cite{carroll:bslcpih}; and the references therein).  I have 
argued elsewhere (Carroll~\cite{carroll:bslcpih}) that the 
implications of these models fit the available evidence on the 
consumption/saving behavior of the typical household reasonably well, 
certainly much better than the old Certainty Equivalent (CEQ) models 
did.

The standard procedure in all of these papers has been to assume that 
household savings earn a constant, riskfree rate of return $R$.  
Recently, a wave of papers (Haliassos and 
Bertaut~\cite{haliassos&bertaut:fewholdstocks}, 
Fratantoni~\cite{fratantoni:equitypremium},Gakidis~\cite{gakidis:stocksforold}; 
Cocco, Gomes, and Maenhout~\cite{cgm:lcportfolio}; and 
Hochgurtel~\cite{hochgurtel:bufferportfolio}) has examined the 
predictions of these kinds of models when consumers facing labor 
income risk are allowed to choose freely between investing in a 
low-return safe asset like the one considered in the earlier models 
and investing in risky assets parameterized to resemble the returns 
yielded by equity investments in the past.

The following model is what I will characterize as the basic 
stochastic life cycle model with portfolio choice.  The consumer's 
goal is to solve the problem
\begin{eqnarray*}
	\max &  & \sum_{s=t}^{T} \beta^{s-t} \mathcal{D}_{t,s} u(C_{t})  \label{eq:maxutil} \\
\end{eqnarray*}
where $u(C)$ is a constant relative risk aversion utility function 
$u(C)=c^{1-\rho}/(1-\rho)$, $\beta$ is the (constant) geometric 
discount factor, and $\mathcal{D}_{t,s} = \prod_{h=t}^{s-1}(1-d_{h})$ 
is the probability that the consumer will not die between periods $t$ 
and $s$ ($\mathcal{D}_{t,t}$ is defined to be 1; $d_{t}$ is the 
probability of death between period $t$ and $t+1$).

The maximization is of course subject to constraints.  In particular, 
if, following Deaton~\cite{deatonLiqConstrs}, we define $X_{t}$ as 
`cash-on-hand' at time $t$, the sum of wealth and current income, then 
the consumer faces a budget constraint of the form
\begin{eqnarray*}
	X_{t+1} & = & R_{t+1}S_{t} + Y_{t+1}
\end{eqnarray*}
where $S_{t} = X_{t}-C_{t}$ is the portion of last period's resources 
the consumer did not spend, $R_{t+1}$ is the gross rate of return 
earned by the consumer's entire portfolio between $t$ and $t+1$, and 
$Y_{t+1}$ is the noncapital income the consumer earns in period $t+1$.

The portfolio-weighted return $R_{t+1}$ will depend on the 
proportion of the portfolio that is allocated to the safe and the 
risky assets, and on the rate of return on the risky asset between $t$ 
and $t+1$.  Call the proportion of the portfolio invested in the risky 
asset (`stocks') $w_{s,t}$ (where $w$ is mnemonic for the portfolio 
`weight'), and $(1-w_{s,t})$ is the portion invested in the safe 
asset.  If the return on stocks between $t$ and $t+1$ is $R_{s,t+1}$, 
the portfolio-weighted return on the consumer's savings will be 
$R(1-w_{s,t})+R_{s,t}w_{s,t}$.  

Assume that the consumer's noncapital income in each period is given 
by their permanent income $P_{t}$ mutiplied by a mean-one transitory 
shock, $E_{t}[\epsilon_{t+1}] = 1$, and assume that permanent income 
grows at rate $G_{t}$ between periods, but is also buffeted by a 
mean-one shock, $P_{t+1} = G_{t+1}P_{t}\eta_{t+1}$ such that $E_{t} 
[\eta_{t+1}]=1$.

Given these assumptions, the consumer's choices are influenced by
only two state variables at a given point in time: the level of 
the consumer's assets $X_{t}$ and the level of permanent income, $P_{t}$.
As usual, the problem can be rewritten in recursive form with a value
function $V_{t}(X_{t},P_{t})$.  Written out fully in this form,
the consumer's problem is

\begin{eqnarray}
	V_{t}(X_{t},P_{t}) & =& \max_{\{C_{t},w_{s,t}\}} 
	u(C_{t}) + \beta \mathcal{D}_{t,t+1} E_{t}\left[V_{t+1}(X_{t+1},P_{t+1}) \right] \nonumber\\ 
	 & \mbox{such that} &   	\label{eq:bellmanstd}  \\
	S_{t}   & = & X_{t}-C_{t} \nonumber \\
	X_{t+1} & = & R_{t+1}S_{t} + Y_{t+1} \nonumber \\
	Y_{t+1} & = & P_{t+1}\epsilon_{t+1} \nonumber \\
	P_{t+1} & = & G_{t} P_{t} \eta_{t+1} \nonumber \\
	R_{t+1} & = & 
	R(1-w_{s,t}) + 
	R_{s,t+1}w_{s,t} \nonumber 
\end{eqnarray}

\subsubsection{Parameterization}
The papers cited earlier that examine the predictions of such models 
universally find that with plausible assumptions about tastes (in 
particular, assuming a coefficient of relative risk aversion of less 
than 5), the models predict that consumers will want to invest a large 
fraction of their portfolios in stocks.  As several authors have 
noted, this is the microeconomic manifestation of the equity premium 
puzzle pointed out by Mehra and Prescott~\cite{mehraPrescottPuzzle}.

In order to obtain at least a rough match between the data and the 
predictions of the baseline model, I will make several 
parameterization assumptions which will discourage consumers from 
holding a risky asset like stocks.  In the end, it will still be 
necessary to assume a high value for the coefficient of relative risk 
aversion, but not so high as would be necessary for more conventional 
parametric assumptions.

\subsubsection{The Labor Income Process}

Perhaps the most easily measured of the parameters of the model is
$G_{t}$, the determininstic component of the pattern of earnings over
the life cycle.  Carroll~\cite{carroll:bslcpih} presents age/earnings
profiles for consumers in various different occupations; I simply 
adopt the middle of the three stylized parameterizations that he 
argues fit different occupations reasonably well.  Details are in the 
appendix.

A number of studies have examined the characteristics of household 
income processes and found that exclusive of periods when household 
income drops very sharply, the income process is reasonably well 
characterized by assuming that both the transitory and permanent 
shocks to income are serially uncorrelated and lognormally 
distributed.  Carroll~\cite{carroll:brookings} presented evidence that 
household incomes occasionally fall very dramatically, usually in 
periods when the household head is either unemployed for an extended 
period or is ill or disabled; I will follow that paper in assuming 
that the probability that household income is zero is given by 
$p_{\epsilon} = Pr(\epsilon = 0) = .005$.  If the 
realization of $\epsilon_{t}$ is nonzero then $\epsilon_{t}$ is 
distributed lognormally with a mean such that $E_{t} 
[\tilde{\epsilon}_{t+1}] = 1$.  This implies that $E_{t} 
[\tilde{\epsilon_{t}} | \epsilon_{t} > 0] = 1/(1-p_{\epsilon}).$ Using 
the fact that for a lognormally distributed variable $z$
\begin{eqnarray*}
	\log E[z] & = & E[\log z] + \frac{1}{2} \mbox{var}[\log z]  
\end{eqnarray*}
we have that
\begin{eqnarray*}
\log E[\epsilon | \epsilon>0] & = & E[\log \epsilon | \epsilon>0] + \frac{1}{2}\mbox{var}_{t}[\log \epsilon | \epsilon > 0]
\end{eqnarray*}
Using the approximation that $\log[1/(1-p_{\epsilon})] \approx 
p_{\epsilon}$ if $p_{\epsilon}$ is small, this gives us that
\begin{eqnarray*}
p_{\epsilon} & \approx & E[\log \epsilon | \epsilon>0] + \mbox{var}_{t}[\log \epsilon | \epsilon > 0] \\
E[\log \epsilon | \epsilon>0] & \approx & p_{\epsilon}-\sigma_{\epsilon}^{2}/2 
\end{eqnarray*}
and thus our assumption is that 
\begin{equation}
\log \epsilon_{t} \sim \mathcal{N}(p_{\epsilon}-\sigma_{\epsilon}^{2}/2,\sigma^{2}_{\epsilon})
\end{equation}
Similarly, but more simply, we assume that $\eta_{t+1}$ is lognormally 
distributed such that $E_{t}[\eta_{t+1}] = 1$ implying that the 
distribution for $\eta$ is $\eta \sim 
\mathcal{N}(-\sigma_{\eta}^{2}/2,\sigma_{\eta}^{2})$.  We use the 
parameter estimates obtained in Carroll~\cite{carroll:brookings} using 
data from the {\it Panel Study of Income Dynamics}: 
$\sigma_{\epsilon}=\sigma_{\eta}=.1$.


\subsubsection{The Capital Income Process}

The 3-month T-bill is the closest real-life proxy to a truly riskless 
asset, because inflation risk at high frequencies is small.  The 
average rate of return on the three-month T-bill in the postwar period 
has been about zero.  However, to match the model to 
data, we will have to make a decision about which assets in real life 
are ``risky'' and which are ``riskless.''  Because the real-life 
assets that we will identify with the model's `riskless' asset 
are at least slightly risky and consequently earn a higher rate
of return than the t-bill, we somewhat arbitrarily fix the real
rate of return on the `riskless' asset at 2 percent.

One might suppose that the appropriate assumption about the 
distribution of shocks to stock returns was well pinned down by the 
available data.  However, Cochrane~\cite{cochrane:uncertainfacts} 
applies a variety of sophisticated tests to historical stock return 
data and concludes that a reasonable confidence interval for the 
average historical equity premium is between 3 percent and 13 percent.  
Because some of the more plausible explanations of the equity premium 
puzzle assert that, whatever the {\it ex post} returns may have been, 
most people did not {\it ex ante} anticipate that stocks would perform 
as well as they have (and because assuming a small equity premium 
helps the model fit the data), my baseline parameterization will 
assume that the equity premium is only three percent.  

The baseline assumption is that with probability $(1-p^{s})$ stock 
returns will be distributed lognormally with a mean $\overline{R}_{s} 
= E_{t} R^{s}_{t+1} = 1.05$ or 3 percent in excess of the return 
available on the safe asset and a standard deviation $\sigma^{s} = .2$ 
which is approximately equal to the annual standard deviation of 
returns on the S\&P 500 over the period 1925-1995 according to 
Gakidis~\cite{gakidis:stocksforold}.  With probability $p^{s}$ there 
is a stock market `crash' in which the gross return is $R^{s}_{t+1}= 
.1$.  This is meant to capture the experience of the Great Depression 
in the U.S., when the value of the S\&P 500 fell by about 90 percent 
from its 1929 peak to its 1933 trough, and more broadly to capture the 
empirical fact that stock returns are not quite lognormally 
distributed but instead exhibit `fat tails.'  In combination with the 
presence of a small chance of unemployment spells, this assumption is 
sufficient to prevent consumers from holding all of their assets in 
stocks, because of the small risk that labor income and the stock 
porfolio could both fall to nearly zero at the same time, driving the 
marginal utility of consumption to infinity.

\subsubsection{Tastes}
Taste parameters are difficult to measure and so the usual procedure
is to make assumptions about the other parameters and then determine
what structure of tastes is consistent with observed behavior.  This
is the procedure that has led to the implausibly high estimates of
the coefficient of relative risk aversion in the past.  I will follow
this procedure in the sense that I will search for a value of the 
coefficient of relative risk aversion in my baseline model such
that, given the other parameteric assumptions, most households wish
to hold only a small proportion of their portfolio, or none at all,
in stocks.  

Even more difficult to measure than risk aversion is the pure rate of 
time preference.  The only credible estimate I know of comes from 
very recent work by Gourinchas and Parker~\cite{gourinchas&parker:lifecycle}, who 

  The usual assumption is that the annual rate of time 
preference is three or four percent; following 
Carroll~\cite{carroll:bslcpih}, I will assume a time preference rate 
of four percent.

Carroll~\cite{carroll:richsave}, Huggett~\cite{huggett:wealth}, 
Quadrini~\cite{quadrini:entrepreneurship}, and others have argued that 
the `standard' version of the model which lacks the option of 
investing in stocks cannot reproduce the high saving rates observed at 
the top of the income spectrum.  I will therefore modify the model
to incorporate the modifications that these authors have suggested
may be necessary to bring the predictions of the model into
accord with the data.

\subsection{The `Capitalist Spirit'}

Carroll~\cite{carroll:richsave} has argued that in order to explain 
the saving behavior of the truly wealthy it is necessary to modify the 
standard model to provide a mechanism by which the ownership of wealth 
yields utility directly.  That paper proposes a model in which 
consumers decide between consumption and wealth:
\begin{eqnarray}
 	\max_{c_{t}} & ~~u(C_{T}) + B(s_{T}) \\ \nonumber
  	           & \mbox{s.t.~} S_{T} = X_{T} - C_{T}.  \nonumber
\end{eqnarray}
where $u(c)$ is the standard CRRA utility function but $B(S)$ takes
a modified Stone-Geary form, 
\begin{eqnarray*}
B(S) &= &\frac{(S + \gamma)^{1-\alpha}}{1-\alpha}.
\end{eqnarray*}

Carroll~\cite{carroll:richsave} shows that if one assumes that 
$\alpha<\rho$ then wealth will be a `luxury good' in the sense that as 
lifetime resources rise, a larger proportion of those resources is 
devoted to $S_{T}$.  In the limit as $X_{T}$ approaches infinity, 
the proportion of resources devoted to $S_{T}$ approaches 1.  The 
other salient feature of the model is that if $\lambda > 0$ there will 
be a range of values of lifetime income for which the optimal choice 
is to consume all resources and set $S_{T}=0$.

The simplest interpretation is that this is a model of bequests as 
luxury goods.  Interpreted in this way, it implies that people with
low levels of lifetime incomes leave no bequests because they always
find the marginal utility of consumption to be higher than the marginal
utility of even the first dollar of a bequest, but that above some
threshold level of lifetime income (permanent income) one devotes 
an increasingly large proportion of lifetime resources to bequests.

Carroll~\cite{carroll:richsave} argues that the model can be 
interpreted more loosely as capturing a more direct effect of wealth 
on utility.  In this paper I will stick with the bequests 
interpretation, though it bears remembering that the implications for 
saving behavior are virtually the same whether one interprets the 
model as one of bequests or as a reduced-form for all of the 
satisfactions that may be conferred by the possession of wealth other 
than those yielded directly by consumption (power, status, grovelling 
by charitable organizations or relatives, etc.).

In this paper the assumption is that one receives utility from the 
contemplation of the potential bequest in proportion to the 
probability that death (and the bequest) will occur.  Thus Bellman's 
equation is modified to:
\begin{eqnarray}
	V_{t}(X_{t},P_{t}) & =& \max_{\{C_{t},w_{s,t}\}} 
	u(C_{t}) + \beta (1-d_{t}) E_{t}\left[V_{t+1}(X_{t+1},P_{t+1}) \right] + d_{t} B(S_{t})\nonumber\\ 
\end{eqnarray}
and the transition equations for the state variables are unchanged.

Parameterization for this model is even more problematic than for the 
usual model.  The baseline assumption will simply be that $\alpha = 
\rho/2$, because that version of the model has some convenient 
analytical features.  The baseline assumption about $\lambda$ will be 
such that in the perfect-certainty case, the bequest motive would 
begin to manifest itself for consumers with a permanent income equal 
to twice the median level of permanent income.  These parametric 
assumptions are arbitrary, and it would be nice to have some 
compelling method for tying them down.  In principle, $\alpha$ and 
$\lambda$ could be estimated using the methods pioneered by Gourinchas 
and Parker~\cite{gourinchas&parker:lifecycle}, although perhaps not
until the Pentium 8 is introduced.

\subsection{Quadrini/Gentry-Hubbard}
Quadrini~\cite{quadrini:entrepreneurship} and Gentry and 
Hubbard~\cite{gentry&hubbard:wealthysave} propose a different 
explanation for the high saving rates of the wealthy: they argue that 
these high saving rates reflect the fact that entrepreneurs can earn 
higher rates of return than are available in the public capital 
markets.  These authors make plausible moral hazard and adverse 
selection arguments to explain why such capital market imperfections 
might exist, drawing upon the extensive literature from the last 
decade that has investigated the possibility that firms may face 
financing constraints.

The models that Quadrini and Gentry and Hubbard propose are too
complex to graft onto the already-complicated structure of the
dynamic stochastic life cycle optimization problem considered 
above.  Fortunately, we can capture the essence of their models
with a simpler structure.  The model of entrepreneurial investment
is as follows.

Some consumers can invest a positive share of their portfolio $0 < 
w_{k,t} \leq 1$ in an entrepreneurial project.  In order to 
capture in the simplest possible way the higher return on 
self-financed projects emphasized by 
Quadrini~\cite{quadrini:entrepreneurship} and Hubbard and 
Gentry~\cite{gentry&hubbard:wealthysave}, the gross return on the 
entreprenurial project $R_{k,t}$ is scaled by a monotonically 
increasing function $\psi(w_{k,t})$ such that $\psi(1)=1$; 
that is, one can only reap the highest possible return on an 
entreprenurial project by investing one's entire net worth in that 
project.  The simplest choice of functional form for $\psi$ would be 
linear, but for technical reasons it is more convenient to assume that 
$\psi$ is quadratic (and concave).\footnote{Because $\psi'(w_{k,t})$ 
appears in the first order condition, assuming that $\psi$ is 
quadratic helps to guarantee that there will be a uniquely optimal 
portfolio share choice for $w_{k,t}$.}

It is difficult to know how to parameterize the rate of return for the 
entrepreneurial project.  Because entrepreneurial investment is 
undoubtedly riskier than stock market investment, it is clear that we 
should assume some `entrepreneurial premium,' 
$E_{t}[R_{k,t+1}]>E_{t}[R_{s,t+1}]$.  Our baseline assumption is that 
$\overline{R}_{k} = E_{t}[R_{k,t+1}]=1.16$ or $0.10$ higher than the 
expected return on stocks, and that the stochastic distribution of 
rates of return around $\overline{R}_{k}$ mimics the distribution of 
the rate of return on stocks except that there is a much larger 
probability of entrepreneurial `failure' (in which $R_{k,t+1}=0$) 
than there is of a stock market crash (in which $R_{s,t+1}=.1$).  
Specifically, our baseline assumption is that the probability of 
`bankruptcy' of the entreprenurial project is $p_{k}=.1$ 
annually.\footnote{Quadrini~\cite{quadrini:entrepreneurship} provides 
evidence that the rate of failure for entrepreneurs who have been 
self-employed for three years or longer is $.1$ annually.  The failure 
rate in the first two years of entrepreneurial activity is much 
greater, but the additional modelling complication and solution time 
required to capture this phenomenon did not seem worth the payoff of 
greater realism.}

To complete the specification of our treatment of entrepreneurial 
investment, we assume that $\lim_{w_{k}\rightarrow 0} 
\psi(w_{k}) = R/ E_{t}[R_{k,t+1}]$; that is, if the 
consumer were to invest an infinitesimal amount in the 
entrepreneurial project, the expected rate of return would be $R$, the 
riskfree return, but the project would still bear the large {\it ex 
ante} risks associated with entrepreneurial ventures.

Not all households are allowed to pursue an entrepreneurial project in 
every period.  Any household that pursued an entrepreneurial project 
in period $t-1$ is eligible to continue that entrepreneurial activity.  
But period $t-1$ nonentrepreneurs are permitted to invest in an 
entrepreneurial project in period $t$ only if they are lucky enough to 
obtain a new  `idea,' an event that occurs with probability 
$\overline{n}$ in each period.  (Formally, $\tilde{n} \sim u[0,1]$ and 
one obtains an idea if $\tilde{n} \leq \overline{n}$).  

If a consumer who is currently engaged in an entrepreneurial activity 
decides to invest none of his portfolio in the project in this period, 
then that consumer cannot resume the project next period; he joins the 
pool of `nonentrepreneurs' and must await the next random draw of an 
entrepreneurial `idea.'  The variable $Q_{t} \in \{0,1\}$ indicates 
whether the consumer is eligible to pursue an entrepreneurial idea.

For technical reasons the best way to specify the portfolio allocation 
problem is to have consumers first decide how much of their total 
wealth to invest in the entrepreneurial project, and then decide how 
to allocate the remaining non-entrepreneurial wealth between the 
riskless asset and stocks.  Mathematically, $w_{k,t}$ of the 
entire portfolio is invested in the entrepreneurial project, leaving 
$(1-w_{k,t})$ to allocate between riskless and risky 
investment, with $w_{s,t}$ designating the portion of this 
remainder allocated to stocks.

To summarize, the rate of return earned by a consumer who saves any
positive amount $S_{t}$ in period $t$ is given by $R_{t+1}$ where:
\begin{eqnarray*}
	R_{t+1} & = & 
	R(1-w_{k,t}-w_{s,t}) + 
	R_{s,t+1}w_{s,t} + R_{k,t+1}Q_{t+1} 
	\psi(w_{k,t})w_{k,t} \nonumber 
\end{eqnarray*}

\ifthenelse{\boolean{ShowTrueEqns}}{
The actual solution method, however, assumes the problem is solved in
two stages: first, make a choice about the entrepreneurial share $w_{k,t}$,
then decide the percentage of the nonentrepreneurial share to invest in
stocks.  In this case the equation for $R_{t+1}$ is:

\begin{eqnarray*}
	R_{t+1} & = & 
	R(1-w_{k,t})(1-w_{s,t}) + 
	R_{s,t+1}(1-w_{k,t})w_{s,t} + R_{k,t+1}Q_{t+1} 
	\psi(w_{k,t})w_{k,t} \nonumber 
\end{eqnarray*}

}

Finally, we assume that running an entrepreneurial project consumes a 
minimum amount of time $\tau>0$ and thus reduces the amount of wage 
income the entrepreneur earns to $Y_{t+1}(1-\tau)$; our baseline 
assumption is $\tau=.1$.  This assumption is necessary to prevent 
consumers who have ever received an entrepreneurial idea from keeping 
the entreprenurial project `alive' by investing an infinitesimal 
amount in the project in every period.  Such a strategy 
would preserve the option value inherent in the entrepreneurial 
project at an arbitrarily small cost; requiring some minimum fixed 
commitment of time imposes a lower bound on the cost of keeping the 
entrepreneurial option alive.

Bellman's equation for this problem is:

\begin{eqnarray}
	V_{t}(X_{t},P_{t},Q_{t}) & =& \max_{\{C_{t},w_{s,t},w_{k,t}\}} 
	u(C_{t}) + \beta E_{t}\left[(1-d_{t}) 
	V_{t+1}(X_{t+1},P_{t+1},Q_{t+1}) \right] \nonumber\\ 
	 & \mbox{such that} &   	\label{eq:bellmanQGH}  \\
	S_{t}   & = & X_{t}-C_{t} \nonumber \\
	X_{t+1} & = & R_{t+1}S_{t} + Y_{t+1}(1-\tau(w_{k,t}>0)) \nonumber \\
	Y_{t+1} & = & P_{t+1}\epsilon_{t+1} \nonumber \\
	P_{t+1} & = & G_{t} P_{t} \eta_{t+1} \nonumber \\
	R_{t+1} & = & 
	R(1-w_{k,t}-w_{s,t}) + 
	R_{s,t+1}w_{s,t} + R_{k,t+1}Q_{t}\psi(w_{k,t})w_{k,t} \nonumber 
\end{eqnarray}
and
\begin{equation}
Q_{t+1} = 
\begin{cases}
   1 & \text{if $w_{k,t} > 0$ and $Q_{t} = 1$}
\\ 1 & \text{if $w_{k,t} = 0$ and $\tilde{n} < \overline{n}$}
\\ 0 & \text{if $w_{k,t} = 0$ and $\tilde{n} \geq \overline{n}$}.
\end{cases} \nonumber 
\end{equation}

\subsection{Everything At Once}

The final model to be considered combines the 
Carroll~\cite{carroll:richsave} `Capitalist Spirit' utility function 
with the entrepreneurial investment structure in the previous section.  
The specification of the problem is identical to that for the previous 
section (\ref{eq:bellmanQGH}) except for the addition of $d_{t} 
B(S_{t})$ to the right hand side of the Bellman equation.

\section{Comparing the Models to the Stylized Facts}

The culmination of the paper is of course a comparison of the
simulation results from the models to the stylized facts developed
in the first section.  In particular, we examine the performance
of the four models with respect to their conformity (or lack thereof)
with the three principal stylized facts from the first section.

}{}


\begin{table}
    \caption{Business Ownership Status and Inheritance Status}\vspace{.3in}
	\centerline{\BoxedEPSF{:Tables:howgotrich.eps}}
	\medskip\medskip\medskip\medskip
	\label{table:howgotrich}
\end{table}

\clearpage
