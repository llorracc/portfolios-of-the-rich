%&custom12pt
%\documentclass[12pt]{article}
%\usepackage[]{Internal:latex:vmargin}

\setmarginsrb{1in}{1in}{1.0in}{1.25in}{0pt}{0pt}{0pt}{.5in}
% {left}{top}{right}{bottom}


\newboolean{ShowFirstStuff}
\setboolean{ShowFirstStuff}{true}

\newboolean{ShowTrueEqns}       % Solution procedure differs from text notation
\setboolean{ShowTrueEqns}{true} % If true, this prints the formulas associated w. true solution method


\begin{document}
\ifthenelse{\boolean{ShowFirstStuff}}{

\subsubsection{Labor Income}


Each household is characterized by a level of permanent income $P$ 
that evolves over time according to the process
\begin{math}
	P_{t} = \eta_{t}P_{t-1}
\end{math}
where $\log \eta_{t} \sim N(-\sigma_{\eta}^{2}/2,\sigma_{\eta}^{2})$ 
which implies that $E_{t-1} [\eta_{t}] = 1$ and following 
Carroll~\cite{carroll:brookings} we assume $\sigma_{\eta} = .1$.

Realized noncapital income $Y_{t}$ is given by permanent income 
multiplied by a transitory shock $\epsilon_{t}$,
\begin{eqnarray*}
	Y_{t} & = & G_{t} P_{t} \epsilon_{t}
\end{eqnarray*}
where $G_{t}$ captures the age profile of earnings, $\epsilon_{t}$ is 
equal to zero (representing a spell of unemployment) with probability 
$p^{y}=.01$ and otherwise is lognormally distributed with standard 
deviation $\sigma_{\epsilon}=.1$ and with a mean such that $E_{t-1}[ 
\epsilon_{t}] = 1$.


\subsubsection{Capital Income and Entrepreneurial Income}

All consumers can invest their savings in a riskless asset which 
yields gross rate of return $R=1.02$ or in `stocks' which yield 
stochastic rate of return $R_{s,t}$.  We assume that with 
probability $(1-p_{s})$ stock returns will be distributed lognormally 
with a mean that yields an equity premium so that $\overline{R}_{s} = 
E_{t} [R_{s,t+1}] = 1.06$ or 4 percent in excess of the return 
available on the safe asset and a standard deviation $\sigma_{s} = .2$ 
which is approximately equal to the annual standard deviation of 
returns on the S\&P 500 in the postwar period.  With probability 
$p_{s}$ there is a stock market `crash' in which the gross return is 
$R_{s,t+1}= .1$.  This is meant to capture the experience of the Great 
Depression in the U.S., when the value of the S\&P 500 fell by about 
90 percent from its 1929 peak to its 1933 trough, and more broadly to 
capture the empirical fact that stock returns are not quite 
lognormally distributed but instead exhibit `fat tails.'  In 
combination with the presence of a small chance of unemployment 
spells, this assumption is sufficient to prevent consumers from 
holding all of their assets in stocks, which they otherwise would do. 
(The fact that for more traditional parameterizations consumers will 
hold 100 percent of their portfolio in stocks is called the `portfolio 
structure puzzle'; it is the microeconomic manifestation of the equity 
premium puzzle.  In contrast, under our combination of assumptions, 
consumers avoid putting 100 percent of their portfolio in stocks 
because of the small risk that labor income and the stock porfolio 
could both fall to nearly zero at the same time, driving the marginal 
utility of consumption to infinity.)


Some consumers can invest a positive share of their portfolio $0 < 
w_{k,t} \leq 1$ in an entrepreneurial project.  In order to 
capture in the simplest possible way the higher return on 
self-financed projects emphasized by 
Quadrini~\cite{quadrini:entrepreneurship} and Hubbard and 
Gentry~\cite{gentry&hubbard:wealthysave}, the gross return on the 
entreprenurial project $R_{k,t}$ is scaled by a monotonically 
increasing function $\psi(w_{k,t})$ such that $\psi(1)=1$; 
that is, one can only reap the highest possible return on an 
entreprenurial project by investing one's entire net worth in that 
project.  The simplest choice of functional form for $\psi$ would be 
linear, but for technical reasons it is more convenient to assume that 
$\psi$ is quadratic (and concave).\footnote{Because $\psi'(w_{k,t})$ 
appears in the first order condition, assuming that $\psi$ is 
quadratic helps to guarantee that there will be a uniquely optimal 
portfolio share choice for $w_{k,t}$.}

It is difficult to know how to parameterize the rate of return for the 
entrepreneurial project.  Because entrepreneurial investment is 
undoubtedly riskier than stock market investment, it is clear that we 
should assume some `entrepreneurial premium,' 
$E_{t}[R_{k,t+1}]>E_{t}[R_{s,t+1}]$.  Our baseline assumption is that 
$\overline{R}_{k} = E_{t}[R_{k,t+1}]=1.16$ or $0.10$ higher than the 
expected return on stocks, and that the stochastic distribution of 
rates of return around $\overline{R}_{k}$ mimics the distribution of 
the rate of return on stocks except that there is a much larger 
probability of entrepreneurial `failure' (in which $R_{k,t+1}=0$) 
than there is of a stock market crash (in which $R_{s,t+1}=.1$).  
Specifically, our baseline assumption is that the probability of 
`bankruptcy' of the entreprenurial project is $p_{k}=.1$ 
annually.\footnote{Quadrini~\cite{quadrini:entrepreneurship} provides 
evidence that the rate of failure for entrepreneurs who have been 
self-employed for three years or longer is $.1$ annually.  The failure 
rate in the first two years of entrepreneurial activity is much 
greater, but the additional modelling complication and solution time 
required to capture this phenomenon did not seem worth the payoff of 
greater realism.}

To complete the specification of our treatment of entrepreneurial 
investment, we assume that $\lim_{w_{k}\rightarrow 0} 
\psi(w_{k}) = R/ E_{t}[R_{k,t+1}]$; that is, if the 
consumer were to invest an infinitesimal amount in the 
entrepreneurial project, the expected rate of return would be $R$, the 
riskfree return, but the project would still bear the large {\it ex 
ante} risks associated with entrepreneurial ventures.

Not all households are allowed to pursue an entrepreneurial project in 
every period.  Any household that pursued an entrepreneurial project 
in period $t-1$ is eligible to continue that entrepreneurial activity.  
But period $t-1$ nonentrepreneurs are permitted to invest in an 
entrepreneurial project in period $t$ only if they are lucky enough to 
obtain a new  `idea,' an event that occurs with probability 
$\overline{n}$ in each period.  (Formally, $\tilde{n} \sim u[0,1]$ and 
one obtains an idea if $\tilde{n} \leq \overline{n}$).  

If a consumer who is currently engaged in an entrepreneurial activity 
decides to invest none of his portfolio in the project in this period, 
then that consumer cannot resume the project next period; he joins the 
pool of `nonentrepreneurs' and must await the next random draw of an 
entrepreneurial `idea.'  The variable $Q_{t} \in \{0,1\}$ indicates 
whether the consumer is eligible to pursue an entrepreneurial idea.

For technical reasons the best way to specify the portfolio allocation 
problem is to have consumers first decide how much of their total 
wealth to invest in the entrepreneurial project, and then decide how 
to allocate the remaining non-entrepreneurial wealth between the 
riskless asset and stocks.  Mathematically, $w_{k,t}$ of the 
entire portfolio is invested in the entrepreneurial project, leaving 
$(1-w_{k,t})$ to allocate between riskless and risky 
investment, with $w_{s,t}$ designating the portion of this 
remainder allocated to stocks.

To summarize, the rate of return earned by a consumer who saves any
positive amount $S_{t}$ in period $t$ is given by $R_{t+1}$ where:
\begin{eqnarray*}
	R_{t+1} & = & 
	R(1-w_{k,t}-w_{s,t}) + 
	R_{s,t+1}w_{s,t} + R_{k,t+1}Q_{t+1} 
	\psi(w_{k,t})w_{k,t} \nonumber 
\end{eqnarray*}

\ifthenelse{\boolean{ShowTrueEqns}}{
The actual solution method, however, assumes the problem is solved in
two stages: first, make a choice about the entrepreneurial share $w_{k,t}$,
then decide the percentage of the nonentrepreneurial share to invest in
stocks.  In this case the equation for $R_{t+1}$ is:

\begin{eqnarray*}
	R_{t+1} & = & 
	R(1-w_{k,t})(1-w_{s,t}) + 
	R_{s,t+1}(1-w_{k,t})w_{s,t} + R_{k,t+1}Q_{t+1} 
	\psi(w_{k,t})w_{k,t} \nonumber 
\end{eqnarray*}

}

Finally, we assume that running an entrepreneurial project consumes a 
minimum amount of time $\tau>0$ and thus reduces the amount of wage 
income the entrepreneur earns to $Y_{t+1}(1-\tau)$; our baseline 
assumption is $\tau=.1$.  This assumption is necessary to prevent 
consumers who have ever received an entrepreneurial idea from keeping 
the entreprenurial project `alive' by investing an infinitesimal 
amount in the project in every period.  Such a strategy 
would preserve the option value inherent in the entrepreneurial 
project at an arbitrarily small cost; requiring some minimum fixed 
commitment of time imposes a lower bound on the cost of keeping the 
entrepreneurial option alive.


\subsubsection{Utility}

The consumer derives utility from two sources: the ongoing utility 
from consumption obtained from the CRRA utility function $u(c) = 
c^{1-\rho}/(1-\rho)$, and the utility derived from the contemplation 
of the future bequest to be left.  The bequest utility function is of 
the form
\begin{eqnarray*}
	B(S) & = & \frac{(S+\lambda)^{1-\alpha}}{1-\alpha} \\
\end{eqnarray*}
where the parameter restriction $\alpha<\rho$ is imposed because it 
implies that bequests are a luxury good, and $\lambda>0$ is imposed 
because it implies that there will be some level of permanent income 
below which the bequest motive will be inoperative (the desired 
bequest would be negative, which is ruled out).

\subsubsection{Bellman's Equation}


Bellman's equation for this problem is:

\begin{eqnarray}
	V_{t}(X_{t},P_{t},Q_{t}) & =& \max_{\{C_{t},w_{s,t},w_{k,t}\}} 
	u(C_{t}) + \beta E_{t}\left[(1-d_{t}) 
	V_{t+1}(X_{t+1},P_{t+1},Q_{t+1})+ d_{t}B(S_{t}) \right] \nonumber\\ 
	 & \mbox{such that} &   	\label{eq:bellmanraw}  \\
	S_{t}   & = & X_{t}-C_{t} \nonumber \\
	X_{t+1} & = & R_{t+1}S_{t} + Y_{t+1}(1-\tau(w_{k,t}>0)) \nonumber \\
	Y_{t+1} & = & P_{t+1}\epsilon_{t+1} \nonumber \\
	P_{t+1} & = & G_{t} P_{t} \eta_{t+1} \nonumber \\
	R_{t+1} & = & 
	R(1-w_{k,t}-w_{s,t}) + 
	R_{s,t+1}w_{s,t} + R_{k,t+1}Q_{t}\psi(w_{k,t})w_{k,t} \nonumber 
\end{eqnarray}
and
\begin{equation}
Q_{t+1} = 
\begin{cases}
   1 & \text{if $w_{k,t} > 0$ and $Q_{t} = 1$}
\\ 1 & \text{if $w_{k,t} = 0$ and $\tilde{n} < \overline{n}$}
\\ 0 & \text{if $w_{k,t} = 0$ and $\tilde{n} \geq \overline{n}$}.
\end{cases} \nonumber 
\end{equation}

\section{Solution Procedure}


Define the function
\begin{equation*}
\Omega_{t}(S_{t},P_{t},Q_{t},w_{k,t},w_{s,t}) = E_{t}\left[(1-d_{t}) V_{t+1}(R_{t+1} 
S_{t}+Y_{t+1},P_{t+1},Q_{t+1})+ d_{t}B(S_{t})\right]
\end{equation*}
which expresses the expected value as of the end of period $t$ of 
having chosen the indicated values of the choice variables 
$S_{t},w_{k,t},w_{s,t}$ given the indicated values of the states 
$P_{t},Q_{t}$.

Now define the value function associated with the optimal 
stock investment share, given the share chosen for the entrepreneurial 
project:

\begin{equation}
{\overline{\Omega}_{t}}(S_{t},P_{t},Q_{t},w_{k,t}) = 
\max_{\{w_{s,t}\}} 
\Omega_{t}(S_{t},P_{t},Q_{t},w_{k,t},w_{s,t}) 
\label{eq:FOCwrtomegak}
\end{equation}

Assuming there is an optimum for $w_{s,t}$ in the range $0 < 
w_{s,t} < 1$, and denoting the derivative of $\Omega_{t}$ with 
respect to $w_{s,t}$ as $\overline{\Omega}^{s}_{t}$, that optimum 
will be characterized by the FOC:

\begin{eqnarray*}
	0 & = & \overline{\Omega}^{s}_{t}(S_{t},P_{t},Q_{t},w_{k,t})  \\
	0 & = & \beta E_{t}\left[ \frac{\partial X_{t+1}}{\partial 
	w_{s,t}} \left( (1-d_{t})
	V_{t+1}^{x}(X_{t+1},P_{t+1},Q_{t+1})\right) \right]\\
	0 & = & \beta E_{t} \left[(R_{s,t+1} - R)(S_{t}) \left((1-d_{t}) 
	V_{t+1}^{x}(X_{t+1},P_{t+1},Q_{t+1})\right)\right].
\end{eqnarray*}
\ifthenelse{\boolean{ShowTrueEqns}}{
or the program version 
\begin{eqnarray*}
	0 & = & \beta E_{t} \left[(R_{s,t+1} - R)(1-w_{k,t})(S_{t}) \left((1-d_{t}) 
	V_{t+1}^{x}(X_{t+1},P_{t+1},Q_{t+1})\right)\right].
\end{eqnarray*}
}


Call the value of $w_{s,t}$ which satisfies the FOC $\hat{w}_{s,t}$. 
If the expected value function as a function of returns is concave and 
single-peaked, the FOC will identify the optimum portfolio share, even 
if that share is outside the permissible range $0 \leq w_{s,t} \leq 
1$.  Furthermore, if $\hat{w}_{s,t}$ is outside the permissible range, 
whichever of $\{0,1\}$ is closest to $\hat{w}_{s,t}$ will be the 
optimal feasible choice of $w_{s,t}$.  Thus we define
\begin{equation}
\check{w}_{s,t} = 
\begin{cases}
1 & \text{if $\hat{w}_{s,t} > 1$,} \\
0 & \text{if $\hat{w}_{s,t} < 0$,} \\
\hat{w}_{s,t} & \text{if $0 < \hat{w}_{s,t} < 1$} 
\end{cases}
\end{equation}
It is possible, however, that the FOC will correspond to a trough 
rather than a peak; to guard against that possibility we define the 
optimal stock investment value function $\overline{\Omega}_{t}$ is 
defined as:
\begin{eqnarray*}
	\overline{\Omega}_{t}(S_{t},P_{t},Q_{t},w_{k,t}) & = & \max 
	[\Omega_{t}(S_{t},P_{t},Q_{t},w_{k,t},\check{w}_{s,t}),
	\Omega_{t}(S_{t},P_{t},Q_{t},w_{k,t},0),
	\Omega_{t}(S_{t},P_{t},Q_{t},w_{k,t},1)]
\end{eqnarray*}
with the optimal stock portfolio share function defined correspondingly.

Similarly, define the value associated with the optimal entrepreneurial share 
as
\begin{eqnarray*}
	\overline{\overline{\Omega}}_{t}(S_{t},P_{t},Q_{t}) & = & \max_{\{w_{k,t}\}} 
	\overline{\Omega}_{t}(S_{t},P_{t},Q_{t},w_{k,t})
\end{eqnarray*}

Assuming there is an optimum for $w_{k,t}$ in the range $0 < 
w_{k,t} < 1$, and denoting the derivative of 
$\overline{\Omega}_{t}$ with 
respect to $w_{k,t}$ as $\overline{\Omega}^{k}_{t}$, that optimum 
will be characterized by the FOC:
}{} % End ifthenelse{ShowFirstStuff}
\begin{equation}
\begin{split}
	0 
	&= \overline{\Omega}^{k}_{t}(S_{t},P_{t},Q_{t},w_{k,t})  
\\  &=\beta E_{t} \biggl\{
	  (R_{k,t+1}Q_{t+1}[ 
	  \psi(w_{k,t})+\psi^{'}(w_{k,t})w_{k,t}]-R()
\\  &\phantom{{=} E_{t} \biggl\{} \quad S_{t} \left((1-d_{t}) 
	V_{t+1}^{x}(X_{t+1},P_{t+1},Q_{t+1})\right)\biggr\}
\end{split}
\end{equation}
\ifthenelse{\boolean{ShowTrueEqns}}{
or the program version
\begin{equation}
\begin{split}
	0 
	&= \overline{\Omega}^{k}_{t}(S_{t},P_{t},Q_{t},w_{k,t})  
\\  &=\beta E_{t} \biggl\{
	  (R_{k,t+1}Q_{t+1}[ 
	  \psi(w_{k,t})+\psi^{'}(w_{k,t})w_{k,t}]-(R(1-w_{s,t})+R_{s,t}w_{w,t}))
\\  &\phantom{{=} E_{t} \biggl\{} \quad S_{t} \left((1-d_{t}) 
	V_{t+1}^{x}(X_{t+1},P_{t+1},Q_{t+1})\right)\biggr\}
\end{split}
\end{equation}
} 

where implicitly we are assuming that the stock share of 
nonentreprenurial wealth $w_{s,t}$ is being chosen optimally in the 
background.

Call the value of $w_{k,t}$ which satisfies the FOC 
$\hat{w}_{k,t}$.  Then define 
\begin{equation}
\check{w}_{k,t} = 
\begin{cases}
1 & \text{if $\hat{w}_{k,t} > 1$,} \\
0 & \text{if $\hat{w}_{k,t} < 0$,} \\
\hat{w}_{k,t} & \text{if $0 < \hat{w}_{k,t} < 1$} 
\end{cases}
\end{equation}

As above, the expected value function given optimal portfolio choices 
$\overline{\overline{\Omega}}_{t}$ is defined as:

\begin{eqnarray*}
	\overline{\overline{\Omega}}_{t}(S_{t},P_{t},Q_{t}) & = & \max 
	[\overline{\Omega}_{t}(S_{t},P_{t},Q_{t},\check{w}_{k,t}),
	\overline{\Omega}_{t}(S_{t},P_{t},Q_{t},0),
	\overline{\Omega}_{t}(S_{t},P_{t},Q_{t},1)]
\end{eqnarray*}

And we can rewrite the Bellman equation (\ref{eq:bellmanraw}) as 

\begin{eqnarray*}
	V_{t}(X_{t},P_{t},Q_{t}) & =& \max_{\{C_{t}\}} 
	u(C_{t}) + \beta \overline{\overline{\Omega}}_{t}(X_{t}-C_{t},P_{t},Q_{t})
\end{eqnarray*}

for which the FOC is:
\begin{eqnarray*}
    0 & = & u'(C_{t}) + \beta 
    \overline{\overline{\Omega}}_{t}^{S}(X_{t}-C_{t},P_{t+1},Q_{t+1}) \\
	0 & = & u'(C_{t}) + \beta E_{t}\left[ \frac{\partial X_{t+1}}{\partial 
	C_{t}} \left( (1-d_{t})
	V_{t+1}^{x}(X_{t+1},P_{t+1},Q_{t+1}) + d_{t}B'(S_{t}) \right) \right]\\
	u'(C_{t}) & = & \beta E_{t} \left[R_{t+1} \left((1-d_{t}) 
	V_{t+1}^{x}(X_{t+1},P_{t+1},Q_{t+1}) + d_{t}B'(S_{t})\right)\right] \\
	C_{t}^{-\rho} & = & \beta E_{t} \left[R_{t+1} \left((1-d_{t}) 
	V_{t+1}^{x}(X_{t+1},P_{t+1},Q_{t+1}) + d_{t}(S_{t}+\lambda)^{-\alpha}\right)\right] \\
	C_{t} & = & \left(\beta E_{t} \left[R_{t+1} \left((1-d_{t}) 
	V_{t+1}^{x}(X_{t+1},P_{t+1},Q_{t+1}) + d_{t}
	(S_{t}+\lambda)^{-\alpha}\right)\right]\right)^{-1/\rho} \\
\end{eqnarray*}
where again the optimal shares are assumed to be chosen for each value 
of $C_{t}$ considered.

Note that, using the Envelope theorem, we have that, if none of the 
implicit constraints is binding, the derivative of the Bellman 
equation with respect to $X_{t}$ is:

\begin{eqnarray*}
	V_{t}^{x} & = & \beta E_{t}\overline{\overline{\Omega}}^{s}_{t}(X_{t}-C_{t},P_{t},Q_{t})\\
\end{eqnarray*}

\section{The Last Period}

In the last period of life, the consumer knows for certain that he 
will not be alive to consume in the next period.  The maximization 
problem then becomes:

\begin{eqnarray*}
	V_{T}(X_{T},P_{T},Q_{t}) & = & \max_{\{C_{t}\}}  u(C_{t}) + B(X_{t}-C_{t})  \\
\end{eqnarray*}

for which the FOC is

\begin{eqnarray*}
	u'(C_{T}) & = & B'(X_{T}-C_{T})  \\
	C_{T}^{-\rho} & = & (X_{T}-C_{T}+\lambda)^{-\alpha}  \\
	C_{T} & = & (X_{T}-C_{T}+\lambda)^{\alpha/\rho}
\end{eqnarray*}

The envelope theorem tells us that, for consumers who plan on leaving 
a positive bequest, 
\begin{eqnarray*}
	\frac{dV_{T}}{dX_{T}} & = & \frac{\partial V_{T}}{\partial X_{T}}
	+\frac{\partial V_{T}}{\partial C_{T}} \frac{\partial C_{T}}{\partial 
	X_{T}}  \\
	 & = & B'(X_{T})  \\
\end{eqnarray*}


\section{Scaling Issues}

\begin{eqnarray*}
	V(x) & = & \frac{x^{1-\rho}}{1-\rho}  \\
	W(x) & = & [(1-\rho)V(x)]^{1/(1-\rho)} \\
	W'(x) &= & \frac{[(1-\rho)V(x)]^{1/(1-\rho) - 1}}{(1-\rho)} V'(x) 
	(1-\rho)\\
	      &= & [(1-\rho)V(x)]^{1/(1-\rho) - (1-\rho)/(1-\rho)}V'(x) \\
	      &= & [(1-\rho)V(x)]^{\rho/(1-\rho)} V'(x) \\
	      &= & [x^{1-\rho}]^{\rho/(1-\rho)} x^{-\rho} \\
	      &= & x^{\rho}x^{-\rho} \\
	      & =& 1
\end{eqnarray*}

\begin{eqnarray*}
	V'(x) & = & x^{-\rho}  \\
    V''(x) & = & (-\rho)x^{-\rho-1}  \\
	U(x) & = & [V'(x)]^{-1/\rho} \\
	U'(x) &= & \frac{[V'(x)]^{1/-\rho - 1}}{-\rho} V''(x) \\
	      &= & \frac{[V'(x)]^{1/-\rho - (-\rho/-\rho)}}{-\rho} V''(x) \\
	      &= & \frac{[V'(x)]^{(1+\rho)/-\rho}}{-\rho} V''(x) \\
	      &= & \frac{[V'(x)]^{-(1+\rho)/\rho}}{-\rho} V''(x) \\
	      &= & \frac{[x^{-\rho}]^{-(1+\rho)/\rho}}{-\rho} (-\rho)x^{-\rho - 1} \\
	      &= & [x^{-\rho}]^{-(1+\rho)/\rho} x^{-(1+\rho)} \\
	      & =& [x^{-1}]^{-(1+\rho)}x^{-(1+\rho)} \\
	      & =& x^{(1+\rho)-(1+\rho)} \\
	      & =& 1
\end{eqnarray*}

\section{Parametric Assumptions}


\begin{tabular}{lcr}
$\beta$ & - & .95 
\end{tabular}


\section{Useful Derivations}
\subsection{Value Function In Second-To-Last Period of Life If Utility Comes Only From Wealth}
Consider a consumer who obtains utility simply from holding wealth,
\begin{eqnarray}
	V_{T-1}(S_{T-1}) & = & \frac{S_{T-1}^{1-\alpha}}{1-\alpha} + \beta E_{T-1}
	[V_{T}(S_{T})] \nonumber \\
     & \text{such that} &  \nonumber \\
	S_{T} & = & R_{T} S_{T-1} \nonumber
\end{eqnarray}
where $R_{T}$ is the (possibly stochastic) return on the 
wealth portfolio.  For problems with this structure the 
infinite-horizon value function takes the form $V(W) = \gamma 
\frac{W^{1-\alpha}}{1-\alpha}$.  We can use the method of undetermined 
coefficients to obtain the formula for $\gamma$:
\begin{eqnarray*}
	\gamma \frac{S_{T-1}^{1-\alpha}}{1-\alpha} & = & 
	\frac{S_{T-1}^{1-\alpha}}{1-\alpha} + \beta E_{T-1}  \frac{(R_{T} 
	S_{T-1})^{1-\alpha}}{1-\alpha}  \\
	 & = & \frac{S_{T-1}^{1-\alpha}}{1-\alpha}\left[1 + \beta E_{T-1} 
	 [R_{T}^{1-\alpha}] \right]  \\
	\gamma & = & 1+\beta E_{t} [R_{t+1}^{1-\alpha}]  \\
\end{eqnarray*}

\subsection{Infinite-Horizon Value Function If Utility Comes Only From Wealth}
Consider a consumer who obtains utility simply from holding wealth,
\begin{eqnarray}
	V_{t}(S_{t}) & = & \frac{S_{t}^{1-\alpha}}{1-\alpha} + \beta E_{t}
	V_{t+1}(S_{t+1}) \label{eq:wealthutil} \\
     & \text{such that} &  \nonumber \\
	S_{t+1} & = & R_{t+1} S_{t} \nonumber
\end{eqnarray}
where $R_{t+1}$ is the (possibly stochastic) return on the 
wealth portfolio.  For problems with this structure the 
infinite-horizon value function takes the form $V(W) = \gamma 
\frac{W^{1-\alpha}}{1-\alpha}$.  We can use the method of undetermined 
coefficients to obtain the formula for $\gamma$:
\begin{eqnarray*}
	\gamma \frac{S_{t}^{1-\alpha}}{1-\alpha} & = & 
	\frac{S_{t}^{1-\alpha}}{1-\alpha} + \beta E_{t} [\gamma \frac{(R_{t+1} 
	S_{t})^{1-\alpha}}{1-\alpha}]  \\
	 & = & \frac{S_{t}^{1-\alpha}}{1-\alpha}\left[1 + \beta \gamma E_{t} 
	 [R_{t+1}^{1-\alpha}] \right]  \\
	\gamma & = & 1+\beta \gamma E_{t} [R_{t+1}^{1-\alpha}]  \\
	\gamma(1-\beta E_{t} [R_{t+1}^{1-\alpha}]) & = & 1  \\
	\gamma & = & \frac{1}{1-\beta E_{t} [R_{t+1}^{1-\alpha}]}
\end{eqnarray*}
\subsection{Portfolio Choice Between a Safe and a Risky Asset}
Now let's give them a meaningful portfolio choice.  Suppose that the 
agent can choose to put a fraction $w$ of her portfolio in a 
risky asset which has stochastic gross return $\tilde{R}_{s,t+1}$.  
Suppose $\log \tilde{R}_{s,t+1} \sim 
N(r_s,\sigma_{r_s}^{2})$.  Recall that for 
lognormally distributed variables we know that
\begin{eqnarray*}
	\log E_{t} [\tilde{R}_{s,t+1}] & = & E_{t}[ \log \tilde{R}_{s,t+1}] + \frac{1}{2} 
	\text{var}_{t}[\log \tilde{R}_{s,t+1}] \\
	 & = & r_s + \frac{1}{2}\sigma_{r_s}^{2}  \\
\end{eqnarray*}
and so the overall return on the portfolio will be
\begin{eqnarray*}
	R_{t+1} & = & (1-w) R + w \tilde{R}_{s,t+1}  \\
	R_{t+1} & = & (1-w) r + w \tilde{r}_{s,t+1} \\
\end{eqnarray*}
which can be approximated by\footnote{To see this, note that $\log 
R^{1-w}\tilde{R}_{s,t+1}^{w} = (1-w)\log R + w \log 
\tilde{R}_{s,t+1}$ and recall that $\log R \approx r$, $\log 
\tilde{R}_{s,t+1} \approx \tilde{r}_{s,t+1}$, and $\log R_{t+1} \approx 
R_{t+1}.$}
\begin{eqnarray*}
	R_{t+1} & \approx & R^{1-w}\tilde{R}^{w}_{t+1}  \\
\end{eqnarray*}

The value-function-maximizing choice of $w$ will be the choice 
that minimizes $\gamma$ (recall that $W^{1-\alpha}/(1-\alpha)<0$) 
which will therefore be the $w$ that minimizes $E_{t}[ 
R_{t+1}^{1-\alpha}]$.
\begin{eqnarray*}
	E_{t}[ R_{t+1}^{1-\alpha}] & = & 
	E_{t}[R^{(1-w)(1-\alpha)}\tilde{R}_{s,t+1}^{w(1-\alpha)}]  \\
	 & = & R^{(1-w)(1-\alpha)}E_{t}[\tilde{R}_{s,t+1}^{w(1-\alpha)}]  \\
\end{eqnarray*}

But because $\tilde{R}_{s,t+1}$ is distributed lognormally 
\begin{eqnarray*}
	\log E_{t}[\tilde{R}_{s,t+1}^{w(1-\alpha)}] & = & E_{t}[\log 
	\tilde{R}_{s,t+1}^{w(1-\alpha)}] + \frac{1}{2}\text{var}_{t}[\log 
	\tilde{R}_{s,t+1}^{w(1-\alpha)}]  \\
	 & = & w(1-\alpha) E_{t}[\log \tilde{R}_{s,t+1}] + 
	 \frac{[w(1-\alpha)]^{2}}{2}\text{var}_{t}[\log 
	\tilde{R}_{s,t+1}]  \\
	 & = & w(1-\alpha) r_s + 
	 \frac{[w(1-\alpha)]^{2}}{2}\sigma_{r_s}^{2}   \\
\end{eqnarray*}

Note that the choice of $w$ that minimizes 
$E_{t}[R_{t+1}^{1-\alpha}]$ will be identical with the choice that 
minimizes its log.  Thus the problem is to find the $w$ that 
minimizes
\begin{eqnarray*}
	\log E_{t}[R_{t+1}^{1-\alpha}]  & = & \log [R^{(1-w)(1-\alpha)} 
	E_{t}\tilde{R}_{s,t+1}^{w(1-\alpha)}] \\
    & = & (1-w)(1-\alpha) \log R + w(1-\alpha) r_s + 
	 \frac{[w(1-\alpha)]^{2}}{2}\sigma_{r_s}^{2}\\
	 & = & (1-\alpha)r + w(1-\alpha) 
	 (r_s-r) + \frac{[w(1-\alpha)]^{2}}{2}\sigma_{r_s}^{2}  \\
\end{eqnarray*}

The FOC is:
\begin{eqnarray*}
	0 & = & (1-\alpha)(r_s-r) + 
	[w(1-\alpha)](1-\alpha)\sigma_{r_s}^{2}  \\
	 & = & (r_s-r) + 
	[w(1-\alpha)]\sigma_{r_s}^{2}  \\
	w & = & 
	\frac{-(r_s-r)}{(1-\alpha)\sigma_{r_s}^{2}}  \\
	w & = & 
	\frac{(r_s-r)}{(\alpha-1)\sigma_{r_s}^{2}}  \\
\end{eqnarray*}

If $\alpha>1$ then this expression has the intuitive implications that 
the share of the portfolio devoted to risky assets increases with the 
equity premium and decreases as consumers get more risk averse 
($\alpha$ rises) or as the riskiness of the risky asset increases 
($\sigma_{r_s}^{2}$ rises).  For moderate values 
of $\alpha$ like 3 and for reasonable values of the variance of 
lognormally distributed returns $\sigma_{r_s}^{2}$ this 
equation implies a value of $w>1$.  That is, consumers borrow at 
the riskless rate in order to invest in the risky asset.  For example, 
suppose that the equity premium $r_s-r = .05$ and 
suppose that the standard deviation of annual stock returns is 
$\sigma_{r_s} = .15$ implying $\sigma_{r_s}^{2} = 
.0225$.  Then for $\alpha=3$ this equation yields $w = 
-.05/(-2*.0225) = 1.111\ldots$; that is, over 100 percent of the 
portfolio is placed in the risky asset.

\subsection{Portfolio Choice Between a Safe and Two Risky Assets}
\begin{eqnarray*}
	R_{t+1} & \approx & R^{(1-w_{s}-w_{k})}\tilde{R}^{w_{s}}_{t+1}\tilde{R}^{w_{k}}_{t+1}  \\
\end{eqnarray*}

The value-function-maximizing choice of $(w_{s},w_{k})$ will 
be the choices that minimize $E_{t}[ R_{t+1}^{1-\alpha}]$.
\begin{eqnarray*}
	E_{t}[ R_{t+1}^{1-\alpha}] & = & 
	E_{t}[R^{(1-w_{s}-w_{k})(1-\alpha)}\tilde{R}_{s,t+1}^{w_{s}(1-\alpha)}\tilde{R}_{k,t+1}^{w_{k}(1-\alpha)}]  \\
	 & = & R^{(1-w_{s}-w_{k})(1-\alpha)}
     E_{t}[\tilde{R}_{s,t+1}^{w_{s}(1-\alpha)}\tilde{R}_{k,t+1}^{w_{k}(1-\alpha)}]  \\
\end{eqnarray*}

If we assume that $\tilde{R}_{s,t+1}$ and $\tilde{R}_{k,t+1}$ are 
distributed lognormally and independently of each other, we have

\begin{eqnarray*}
	\log E_{t}[\tilde{R}_{i,t+1}^{w_{i}(1-\alpha)}] & = & E_{t}[\log 
	\tilde{R}_{i,t+1}^{w_{i}(1-\alpha)}] + \frac{1}{2}\text{var}_{t}[\log 
	\tilde{R}_{i,t+1}^{w_{i}(1-\alpha)}]  \\
	 & = & w_{i}(1-\alpha) E_{t}[\log \tilde{R}_{i,t+1}] + 
	 \frac{[w_{i}(1-\alpha)]^{2}}{2}\text{var}_{t}[\log 
	\tilde{R}_{i,t+1}]  \\
	 & = & w_{i}(1-\alpha) r_s + 
	 \frac{[w_{i}(1-\alpha)]^{2}}{2}\sigma_{r_i}^{2}   \\
\end{eqnarray*}
for $i \in \{s,k\}$.

Note that the combination of $(w_{s},w_{k})$ that minimizes 
$E_{t}[R_{t+1}^{1-\alpha}]$ will be identical with the choice that 
minimizes its log.  Thus the problem is to find the 
$(w_{s},w_{k})$ that minimize
\begin{eqnarray*}
	\log E_{t}[R_{t+1}^{1-\alpha}] & = & \log [R^{(1-w_{s}-w_{k})(1-\alpha)} 
	E_{t}[\tilde{R}_{s,t+1}^{w_{s}(1-\alpha)}\tilde{R}_{k,t+1}^{w_{k}(1-\alpha)}] \\
	 & = & (1-\alpha)r 
	 + w_{s}(1-\alpha) (r_s-r) + \frac{[w_{s}(1-\alpha)]^{2}}{2}\sigma_{r_s}^{2}  
	 + w_{k}(1-\alpha) (r_k-r) + \frac{[w_{k}(1-\alpha)]^{2}}{2}\sigma_{r_k}^{2}  \\
\end{eqnarray*}

The FOC with respect to $w_{i}$ for $i \in \{s,k\}$ is:
\begin{eqnarray*}
	0 & = & (1-\alpha)(r_i-r) + 
	[w_{i}(1-\alpha)](1-\alpha)\sigma_{r_i}^{2}  \\
	 & = & (r_i-r) + [w_{i}(1-\alpha)]\sigma_{r_i}^{2}  \\
	w_{i} & = & \frac{-(r_i-r)}{(1-\alpha)\sigma_{r_i}^{2}}  \\
	w_{i} & = & \frac{(r_i-r)}{(\alpha-1)\sigma_{r_i}^{2}}  \\
\end{eqnarray*}

This formula provides a useful way to parameterize the problem, 
because as $S \rightarrow \infty$ the consumers in our model in the 
limit derive all of their utility from wealth, so we should expect 
their behavior in the limit to resemble that of agents like those 
defined in equation ($\ref{eq:wealthutil}$).  If we observe values 
for $w_{r_{s}}$ and $\sigma_{r_{s}}^{2}$ we can calculate the 
value of $\alpha$ which would be consistent with those observations:
\begin{eqnarray*}
	\alpha & = & 1 + \frac{r_{s}-r}{w_{s}\sigma_{r_{s}}^{2}}
\end{eqnarray*}

Consider first the rich households who are not engaged in 
entrepreneurial activity.  Averaging across our four datasets, 
the observed value of $w_{s}$ for these households is roughly 
0.7.  Our baseline assumption about the process for the stock market
yields a value of $\sigma_{r_{s}}^{2} \approx .07$.  This yields an 
estimate of
\begin{eqnarray*}
	\alpha & = & 1 + \frac{.035}{0.7*.07} \\
	 & = & 1.7
\end{eqnarray*}


If take the richest 1/2 percent of the consumers in our datasets as 
reflecting a close approximation to $S = \infty$, the appropriate 
sample values to match are roughly $w_{k} = .5$, $w_{s} = 
.25$.  This implies that we must make assumptions such that 

\begin{eqnarray*}
	\frac{w_{s}}{w_{k}} & = & .5  \\
	 & = & \frac{(r_{s}-r)/\sigma_{r_{s}}^{2}}{(r_{k}-r)/\sigma_{r_{k}}^{2}} \\
	 & = & \left(\frac{r_{s}-r}{r_{k}-r}\right)\left(\frac{\sigma_{r_{k}}^{2}}{\sigma_{r_{s}}^{2}}\right)
\end{eqnarray*}


\end{document}



Furthermore, casual scrutiny of the {\it Forbes 400} list reveals that 
most of the wealthiest people in the world made their wealth primarily 
or exclusively from one enterprise or a set of closely connected 
enterprises, so it seems reasonable to impose.

Quadrini form:

In each period there is some probability that a given household will 
have the opportunity to enter the business sector.  However, operating 
a business requires the payment of a fixed cost $f$ every period.  

$s_{t}$  savings from last period outside of the business 
$h_{t}$  capital inside the business 


\begin{eqnarray*}
x_{t+1} & = & R_{t+1} s_{t} + (\alpha_{t+1} h_{t} - f) + y_{t+1} \\
h_{t+1} & = & h_{t}+(x_{t}-s_{t}-c_{t})
\end{eqnarray*}

\begin{eqnarray*}
s_{t} & \geq & 0 \\
h_{t} & \geq & 0 
\end{eqnarray*}


\begin{eqnarray*}
	V_{t}(X_{t}) & = & \max_{\{C_{t},K_{t}\}} u(C_{t}) + \beta E_{t} 
	V_{t+1}(X_{t+1})  \\
	 & s.t. &   \\
	X_{t+1} & = & R_{t+1}^{s}S_{t} + R_{k,t+1}Q_{t+1} \psi[\frac{K_{t}}{S_{t}+K_{t}}]K_{t}  \\
\end{eqnarray*}

\begin{eqnarray*}
	V_{t}(X_{t}) & = & \max_{\{C_{t},K_{t}\}} u(C_{t}) + \beta E_{t} 
	V_{t+1}\left(R_{t+1}^{s}(X_{t}-C_{t}-K_{t}) + 
	R_{k,t+1}Q_{t+1} \psi[\frac{X_{t}-C_{t}-K_{t}}{K_{t}}]K_{t}\right)  \\
\end{eqnarray*}


All period-$t$ dated uncertainty is resolved at the beginning of the 
period, before the consumer makes that period's consumption and 
investment choices.  Thus the consumer learns the value of 
$\epsilon_{t}$, the transitory shock to labor income, $\eta_{t}$ the 
permanent shock to labor income, and the values for $R_{s,t}$, the 
gross return on stocks purchased in the previous period, and 
$R_{k,t}$, the gross return on the entrepreneurial project (if any) 
in which the household invested in the previous period.  We assume 
that the net return on the entrepreneurial project is given by 
$R_{k,t}Q_{t+1} \psi(w_{t-1}_{k})$ where $\psi$ is a function that 
scales down the return on the entrepreneurial project to an extent 
determined by the fraction of the household's portfolio that is 
invested in the project.  If the entire portfolio is invested in the 
project ($w_{t-1}_{k}=1$) then $\psi(1)=1$ and the household 
reaps the full return.  If a smaller fraction of the portfolio is 
invested, the return is proportionately lower.

In order to start an entrepreneurial venture, a household who is not 
currently engaged in an entrepreneurial project must have an `idea,' 
an event that occurs randomly with probability $p=.05$ at the 
beginning of each year.  The household then must choose whether to 
implement the idea by investing a positive fraction of its portfolio 
in the entrepreneurial project, or to remain a worker.  We 
the presence or absence of 


The expected value of wealth 
that is not consumed in the present period is given by the discounted 
value if the consumer is alive in the next period (and event with 
probability $(1-d_{t})$ plus the value they attach to the bequest they 
will leave behind in the event of death.  Denote total household 
resources at the beginning of period $t$ by $X_{t}$, household 
noncapital income by $Y_{t}$, the rate of return on stocks as 
$R_{s,t}$ and the share of the household portfolio in stocks is 
designated $w_{s,t}$.  The return on the entrepreneurial 
project depends on the proportion of the household portfolio invested 
in the project.  The raw return is denoted $R_{k,t}$, but this is 
multiplied by a function $\psi()$




get utility from current consumption and from the 
expectation of future consumption (if they live 

At the beginning of the period, there is a small probability $p=.05$ 
that the household will have an entrepreneurial `idea.'  
Implementing the `idea' 

Non-entrepreneurial households solve the problem:


amount of the entrepreneurial project to the value associated with 
choosing $w_{k,t}=0$.


\subsubsection{First Order Conditions}

The FOC wrt $C_{t}$ is:

\begin{eqnarray*}
	0 & = & u'(C_{t}) + \beta E_{t}\left[ \frac{\partial X_{t+1}}{\partial 
	C_{t}} \left( (1-d_{t})
	V_{t+1}^{x}(X_{t+1},P_{t+1},Q_{t+1}) + d_{t}B'(S_{t}) \right) \right]\\
	u'(C_{t}) & = & \beta E_{t} \left[R_{t+1} \left((1-d_{t}) 
	V_{t+1}^{x}(X_{t+1},P_{t+1},Q_{t+1}) + d_{t}B'(S_{t})\right)\right] \\
	C_{t}^{-\rho} & = & \beta E_{t} \left[R_{t+1} \left((1-d_{t}) 
	V_{t+1}^{x}(X_{t+1},P_{t+1},Q_{t+1}) + d_{t}(X_{t+1}+\lambda)^{-\alpha}\right)\right] \\
	C_{t} & = & \left(\beta E_{t} \left[R_{t+1} \left((1-d_{t}) 
	V_{t+1}^{x}(X_{t+1},P_{t+1},Q_{t+1}) + d_{t}
	(X_{t+1}+\lambda)^{-\alpha}\right)\right]\right)^{-1/\rho} \\
\end{eqnarray*}

The FOC wrt $w_{s,t}$ is:

\begin{eqnarray*}
	0 & = & \beta E_{t}\left[ \frac{\partial X_{t+1}}{\partial 
	w_{s,t}} \left( (1-d_{t})
	V_{t+1}^{x}(X_{t+1},P_{t+1},Q_{t+1}) + d_{t}B'(S_{t}) \right) \right]\\
	0 & = & \beta E_{t} \left[(R - R_{s,t+1})(S_{t}) \left((1-d_{t}) 
	V_{t+1}^{x}(X_{t+1},P_{t+1},Q_{t+1}) + d_{t}B'(S_{t})\right)\right]
\end{eqnarray*}

If $Q_{t}=0$ there is no FOC wrt $w_{k,t}$.  If $Q_{t} > 0$ then 
the FOC wrt $w_{k,t}$ is:

\begin{eqnarray*}
	0 & = & \beta E_{t}\left[ \frac{\partial X_{t+1}}{\partial 
	w_{k,t}} \left( (1-d_{t})
	V_{t+1}^{x}(X_{t+1},P_{t+1},Q_{t+1}) + d_{t}B'(S_{t}) \right) \right]\\
      & = & \beta E_{t} \left[(R - 
	R_{k,t+1}(Q_{t+1} \psi(w_{k,t})+Q_{t+1} \psi^{'}(w_{k,t})w_{k,t}))(S_{t}) \left((1-d_{t}) 
	V_{t+1}^{x}(X_{t+1},P_{t+1},Q_{t+1}) + d_{t}B'(S_{t})\right)\right]
\end{eqnarray*}

The consumer decides whether or not to engage in the project by 
comparing the value associated with engaging in the optimal positive 


	0 & = & \Omega_{s,t}(S_{t},P_{t},Q_{t},w_{k,t},w_{s,t})  \\
      & = & \beta E_{t} \left[(R(1-w_{k,t}) - 
	R_{k,t+1}Q_{t+1}[ 
	\psi(w_{s,t})+\psi^{'}(w_{k,t})w_{s,t}])S_{t} \left((1-d_{t}) 
	V_{t+1}^{x}(X_{t+1},P_{t+1},Q_{t+1}) + d_{t}B'(S_{t})\right)\right]



The quality of the `idea' is indicated by the indicator variable 
$Q_{t} \in [0,1,2]$.  $Q_{t} = 0$ indicates a household without an 
entrepreneurial idea, $Q_{t} = 1$ indicates a household with a `new' 
entrepreneurial idea, and $Q_{t} = 2$ indicates a household with a 
`mature' entrepreneurial idea.  

Quadrini~\cite{quadrini:entrepreuneurship} shows that the exit 
rate from entrepreneurship declines dramatically as tenure of the 
entrepreneur in the business increases.  Roughly 40 percent of 
entrepreneurial projects fail in the first year, but the failure rate 
of projects which have survived for three years or longer is only 
about 10 percent.  We capture this phenomenon by making the stochastic 
rate of return on entreprenurial projects $R_{k,t+1}$ a function of 
$Q_{t}$.  For `new' projects there is a 40 percent chance that 
$R_{k,t+1} = 0$ while for `mature' projects the probability of 
failure is only 10 percent.  

