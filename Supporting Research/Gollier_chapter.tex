%\newtheorem{theorem}{Theorem}
%\newtheorem{lemma}{Lemma}
%\newtheorem{proposition}{Proposition}
%\newtheorem{definition}{Definition}
%\newtheorem{corollary}{Corollary}


\documentstyle[titlepage,bezier,12pt]{article}
%%%%%%%%%%%%%%%%%%%%%%%%%%%%%%%%%%%%%%%%%%%%%%%%%%%%%%%%%%%%%%%%%%%%%%%%%%%%%%%%%%%%%%%%%%%%%%%%%%%%%%%%%%%%%%%%%%%%%%%%%%%%
%TCIDATA{OutputFilter=LATEX.DLL}
%TCIDATA{Created=Fri Dec 19 15:24:55 1997}
%TCIDATA{LastRevised=Thu Mar 11 10:46:39 1999}
%TCIDATA{<META NAME="GraphicsSave" CONTENT="32">}
%TCIDATA{Language=American English}
%TCIDATA{CSTFile=LaTeX article (bright).cst}

\newcommand{\ind}{\hbox {$\perp$\hskip-3pt $\perp$}}
\def\tc{\tilde c}
\def\tg{\tilde g}
\def\ty{\tilde y}
\input{tcilatex}

\begin{document}

\title{What Does The Classical Theory Have to Say about Household Portfolios?%
\thanks{%
To appear in ''Household Portfolios'', MIT\ Press, edited by Luigi Guiso,
Michael Haliassos and Tullio Jappelli.\ Comments welcomed.}}
\author{Christian Gollier \\
%EndAName
Universit\'{e} de Toulouse, and Institut Universitaire de France.}
\date{March 1999 \\
(Very preliminary)}
\maketitle

\section{Introduction}

The objective of this chapter is to examine the mechanisms by which
households determine the optimal structure of their portfolio. Because risky
assets generate larger expected returns than the risk free asset,
risk-averse households must determine the best compromise between risk and
expected return. In a static framework, this problem is usually formalized
by introducing a concave utility function on final consumption.\ Under the
standard axioms of decision under uncertainty (von Neumann-Morgenstern
(1944)), households will select the portfolio which maximizes the expected
utility of their final consumption. The degree of concavity of the utility,
as defined by Arrow (1971) and Pratt (1964), \ characterizes their degree of
absolute risk aversion.\ It can be measured by answering to questionnaires
related to risk choices.\ When this information is combined with the
specific distribution of returns of financial assets, one can compute the
optimal portfolio of the household. An increase in the absolute risk
aversion reduces the demand for risky assets. This mechanism is well
understood and is included in most textbooks in market finance.

I am tempted to say that this is also the only mechanism that is well
understood \ in this field. For example, important progresses have been made
in the late sixties by introducing time into the portfolio strategies.
Usually, households have long term objectives as retirement when they invest
in financial markets. How does long horizon affect the optimal structure of
the portfolio?\ Said differently, how does the option to purchase stocks
tomorrow modify the attitude towards portfolio risk today? The pathbreaking
papers of Mossin (1968), Merton (1969) and Samuelson (1969) did answer to
this question in a very simple but intriguing way: the intrinsically dynamic
structure of portfolio management has no effect the solution to the
problem.\ Under their assumptions, the sequence of portfolio structures that
are statically optimal is also dynamically optimal.\ In other words,
households should do as if each period of investment is the last period
before retirement. Myopia is optimal.\ The econometrician should not use age
as an explanatory variable for households' portfolios. But it is often
forgot that this result holds only under very restrictive conditions on the
utility function. Namely, myopia is optimal only if the utility function
exhibits constant relative risk aversion. In this paper, we explain why this
is the case and under which condition households who can invest longer in
risky assets should invest more in them.

Household portfolios do not only serve long term objectives as financing
retirement.\ An important share of savings are invested in liquid funds that
play the role of buffer stocks. \ When they have been lucky on their
previous investments, they can reduce their saving rate to cash immediately
the benefits of the larger-than-expected portfolio returns.\ They can also
increase their saving rate in case of an adverse shock on portfolio returns.
This means that households do not bear the accumulated lifetime portfolio
risk at the time of retirement.\ Rather, they can disseminate this risk over
their lifetime consumption pattern. This allows for some forms of time
diversification through consumption smoothing. This is a strong incentive to
accept riskier portfolios. The implications for the econometrician is a
testable hypothesis: households who can adapt over time their contributions
to their pension funds do select riskier portfolio structures.

Two other factors will have important implications on household portfolios.\
First, the presence of a liquidity constraint on the consumption-saving
strategy should be taken into account to design portfolio strategies.\
Indeed, this liquidity constraint, if it binds, will eliminate the
possibility to transfer capital risks over time through consumption
smoothing. It will induce a more conservative portfolio management for those
households that are more likely to face a binding liquidity constraint in
the future. Second, most risk borne by households are not portfolio risks
but uninsurable risks affecting their human capital.\ The intuition suggests
that households bearing a larger risk on their human capital should invest
less in risky assets.\ 

We examine all these factors influencing the optimal portfolio composition.
We present each factor in isolation in a simplified model which excludes the
other factors. We hope that this will help the reader to understand the
nature of this factor without hiding the ''big picture''. The analysis of
how these factors can be combined is let for the calibrations that will be
presented in various chapters of this book. Section 2 is devoted to the
presentation of the static portfolio problem under uncertainty. It will be
useful for the remaining of the paper to observe that this problem is
symmetric to the dynamic consumption problem under certainty. In section 3,
we derive the main properties of the optimal static portfolio. The effect of
background risk is examined in section 4, whereas the repeated nature of
portfolio management is examined in section 5. The notion of time
diversification is developed in section 6.\ Finally, we examine the effect
of the liquidity constraint in section 7 before providing some concluding
remarks.

\section{The basic model and its applications}

Most of the paper deals with the properties of the following model: 
\begin{equation}
\max_{C_{1},...,C_{N}}\quad \sum_{i=1}^{N}p_{i}u(C_{i})  \label{1}
\end{equation}
\begin{equation}
s.t.\quad \sum_{i=1}^{N}p_{i}\pi _{i}C_{i}=X,  \label{2}
\end{equation}
where $(p_{1},...,p_{N})$ and $(\pi _{1},...,\pi _{N})$ are two vectors of
nonnegative scalars and $u$ is a real-valued, increasing and concave
function. Notice that the only specificity of this model with respect to the
very basic problem of consumer theory \ under certainty is the additivity of
the objective function. Under the concavity of $u$, the necessary and
sufficient condition for program (\ref{1}) under constraint (\ref{2}) is
written as 
\begin{equation}
u^{\prime }(C_{i})=\xi \pi _{i},\quad \quad i=1,...,N,  \label{foc}
\end{equation}
where $\xi $ is the Lagrangian multiplier associated to the constraint.

There are three standard applications of this decision problem in
economics.\ The first one is related to the efficient allocation of risk in
a pool of $N$ risk-averse agents. This topic is not directly related to
portfolio decisions and will not be covered in this paper. The second
application of program (\ref{1}) is the static portfolio problem of a
risk-averse investor in an Arrow-Debreu economy. More specifically, consider
an economy in which investors live for one period.\ At the beginning of the
period, the investor under scrutiny is endowed with a sure wealth $X$. He
does not know the state of the world that will prevail at the end of the
period.\ There are $N$ states of the world indexed by $i$, $i=1,...,N$. The
uncertainty is described by the probability $p_{i}$ that state $i$ occurs,
with $\sum_{i}p_{i}=1$. Consumption takes place only after that the
realization of $i$ is observed. The agent invests his endowment in a
portfolio of assets that will be liquidated at the end of the period to
finance consumption. We assume that financial markets are complete. It
implies that for each state $i$, there exists an associated state price (per
unit of probability) $\pi _{i}\geq 0$. In other words, the agent must pay $%
p_{i}\pi _{i}$ at the beginning of the period to increase his consumption by
one unit in state $i$. Vector $(C_{1},...,C_{N})$ is the state-contingent
consumption plan \ of the agent, and equation (\ref{2}) is the budget
constraint of the investor. This vector can also be seen as a portfolio of
Arrow-Debreu securities. The objective function in (\ref{1}) is the ex ante
expected utility of the investor who selects this portfolio. This
application has been intensively used in the theory of finance during the
last three decades.\ 

In this presentation, we assumed that the agent had no income ex-post.
Introducing state-contingent incomes is not a problem when markets are
complete. Indeed, $X$ can also be seen as the ex ante market value of these
state-contingent incomes, i.e., $X=\sum_{i}p_{i}\pi _{i}Y_{i}$ where $Y_{i}$
is the household's income in state $i$. This is the main feature of markets
completeness, as agents can transfer their individual risks to the market.

The last application of program (\ref{1}) is the lifetime consumption-saving
problem under certainty. Consider a household who lives for $N$ periods,
from period $i=1$ to period $i=N$. There is no uncertainty about the net
present value of his incomes. The household's net discounted wealth at the
beginning of period $i=1$ equals $X$. Vector $(C_{1},...,C_{N})$ represents
the time-dependent consumption plan, with $C_{i}$ measuring consumption in
period $i$. The objective function is to maximize the discounted utility of
consumption over the lifetime of the household.\ Parameter $p_{i}$ in (\ref
{1}) is the discount factor associated to period $i$.\ It is often assumed
that $p_{i}=\beta ^{i}$, to escape problems of time-consistency in decision
making. People can finance their consumption in period $i$ by purchasing in
period $1$ zero-coupon bonds maturing in period $i$. The gross rate of
return of such a bond is denoted $(p_{i}\pi _{i})^{-1}$. Constraint (\ref{2}%
) is the household's lifetime budget constraint. Program (\ref{1}) has been
a cornerstone of the literature on the Permanent Income Hypothesis in
macroeconomics.

In spite of their technical equivalence, these two problems are different in
nature. For example, the concavity of the utility function represents risk
aversion in the portfolio problem, whereas it implies aversion to
consumption fluctuations in the consumption-saving problem. \ It means at
the same time that the agent is willing to perfectly insure risks if
insurance prices are fair ($\pi _{i}=1$ for all $i$), and that he is willing
to smooth consumption over time if the return on bonds equals the rate of
impatience. This aversion to fluctuations of consumption across time or
states is measured by the Arrow-Pratt index of absolute aversion, which is
defined by $A(C)=-u^{\prime \prime }(C)/u^{\prime }(C)$. It is more
convenient in general to use an index of relative aversion, $\rho
(C)=CA(C)=-Cu^{\prime \prime }(C)/u^{\prime }(C).$ In the consumption-saving
problem, $\rho $ is the inverse of the well-known elasticity of
intertemporal substitution.

These dual interpretations of the theoretical model suggest two ways to
estimate $\rho $. Viewing it as a degree of risk aversion, one can estimate
it by answering to the following question: What is the share of one's wealth
that one is ready to pay to escape the risk of gaining or losing a share $%
\alpha $ of it with equal probability? Let $x$ be this (certainty
equivalent) share of wealth. Suppose that the agent has constant relative
risk aversion (CRRA), which implies that 
\begin{equation}
u(C)=\dfrac{C^{1-\rho }}{1-\rho }.  \label{CRRA}
\end{equation}
Normalizing wealth to unity,\footnote{%
With constant relative risk aversion, the certainty \ equivalent share $x$
is independent of initial wealth.} it implies that $x$ is the solution of
the following equation:

\begin{equation}
0.5\frac{(1-\alpha )^{1-\rho }}{1-\rho }+0.5\frac{(1+\alpha )^{1-\rho }}{%
1-\rho }=\frac{(1-x)^{1-\rho }}{1-\rho }.  \label{x}
\end{equation}
Table 1 relates $x$ to $\rho $, when $\alpha =10$\% or $\alpha =30$\%.

\begin{center}
\begin{tabular}{|c|c|c|}
\hline
RRA & $\alpha =10\%$ & $\alpha =30\%$ \\ \hline
$\rho =0.5$ & 0.3\% & 2.3\% \\ \hline
$\rho =1$ & 0.5\% & 4.6\% \\ \hline
$\rho =4$ & 2.0\% & 16.0\% \\ \hline
$\rho =10$ & 4.4\% & 24.4\% \\ \hline
$\rho =40$ & 8.4\% & 28.7\% \\ \hline
\end{tabular}

Table 1: Relative certainty equivalent loss $x$ associated to the risk of
gaining or losing a share $\alpha $ of wealth, with constant relative risk
aversion $\rho $.
\end{center}

If we focus on the risk of gaining or losing $10\%$ of one's wealth, we
would consider an answer $x=0.5\%,...,2\%$ as a sensible answer to the
question. It implies that it is reasonable to believe that relative risk
aversion is somewhere between 1 and 4.\ Saying it differently, a relative
risk aversion superior to 10 seems foolish, as it implies very high relative
risk premiums.\ Notice in particular that a relative risk aversion of 40\
implies that one would be ready to pay as much as 8.4\% to escape the risk
of gaining or losing 10\% of one's wealth!

The other approach is to see $\rho $ as the degree of relative aversion to
consumption fluctuations. \ Consider an agent who consumes income $1-\alpha $
in each even year, and $1+\alpha $ in each odd year. He is offered to smooth
his consumption by paying a premium $x$ on his average income. What is the
critical value of $x$ which makes the agent indifferent between the two
consumption plans? Under CRRA, the relationship between $x$, $\rho $ and $%
\alpha $ is governed by exactly the same formula (\ref{x}).\footnote{%
To be exact, one of the two terms in the left-hand side of equation (\ref{x}%
) should be multiplied by rate of impatience, depending upon whether we
start with an even year or an odd year.} Looking at Table 1, an interval $%
\left[ 1,4\right] $ for \ the degree of relative aversion to consumption
fluctuations over time seems to be reasonable. Barsky, Juster, Kimball and
Shapiro (1997) used experimental data from the Health and Retirement Study
in the U.S. to measure risk aversion and aversion to consumption
fluctuations for people above 50 of age. They reported values of $\rho $
that are slightly larger than those suggested here.

The beauty and the simplicity of the above model come from the additivity of
the objective function with respect to states of the world or time,
depending upon which application we have in mind. All nice properties that
we will report in the remaining of this chapter will be derived from this
additive hypothesis. In the case of risk, we know that this hypothesis can
be derived from a more fundamental axiom, i.e., the independence axiom. A
similar axiom can be built to derive an additive property for preferences
with respect to time. We also know that the independence axiom has a weak
predictive power in some specific risky situations, mainly those involving
low probabilities. A branch of the economics of uncertainty provides more
general (non-additive) decision criteria to deal with these specific cases.
A similar step has been made in the economics of time, with
non-time-additive models. Those involving habit formations seem to be quite
promising, but will not be examined in this paper (see for example
Constantinides (1990)).

Another problem arises when we try to mix risk and time. When future
consumption levels are uncertain, the objective function is usually defined
as the {\it discounted expected} lifetime utility. This implies that the
utility function represents at the same time the attitude towards risk and
the attitude towards time of the decision maker. But imagine an agent that
does not give the same $x$ when answering to the two questions mentioned
above. This agent may not have the discounted expected lifetime utility as
an objective function.\ Kreps and Porteus (1978) and Selden (1978) suggested
a model \ that would disentangle the degree of risk aversion from the degree
of aversion to consumption fluctuations. In this model, the additivity
across states in each period is preserved, together with the additivity
across period in each state.\ But the additivity in the full space (risk,
time) is not preserved, contrary to what we have with the discounted
expected utility model. Again, we will not cover this generalization in this
chapter.\ Applications of Kreps-Porteus preferences in macroeconomics are
explored by Weil (1990).

\section{The standard static portfolio problem\label{standardsec}}

After this discussion about the basic\ decision model (\ref{1}), it is now
time to explore the main features of its solution. We do this with the risk
application in mind.\ We consider an agent who lives for one period and who
has to invest his endowment $X$ in a portfolio that will be liquidated at
the end of the period to finance his final consumption. Of course, this
model is completely unrealistic because it is static, and because we assume
market completeness. This model is considered as a benchmark, and will be
extended to include dynamic strategies and markets incompleteness in
subsequent sections.

First-order condition (\ref{foc}) holding in each possible state, it must
also hold in expectation.\ This condition is written as 
\begin{equation}
Eu^{\prime }(\widetilde{C})=\xi E\widetilde{\pi }=\dfrac{\xi }{R},  \label{3}
\end{equation}
where $E\widetilde{z}=\sum_{i}p_{i}z_{i}$ is the expectation operator and $R=%
\left[ E\widetilde{\pi }\right] ^{-1}$ is the gross risk free rate. Indeed,
an asset guaranteeing one unit of consumption with certainty at the end of
the period costs $\sum_{i}p_{i}\pi _{i}=E\widetilde{\pi }$ at the beginning
of the period. Multiplying both sides of equality (\ref{foc}) by $c_{i}$,
taking the expectation and using conditions (\ref{2}) and (\ref{3}) also
yields 
\begin{equation}
E\widetilde{C}u^{\prime }(\widetilde{C})=RXEu^{\prime }(\widetilde{C})
\label{4}
\end{equation}

Let us assume that a Two-fund Separation result holds here, i.e., that it is
optimal for the agent to limit his choice to the allocation of his endowment
between two specific funds of assets.\ The first fund is risk free with
gross return $R$.\ The second fund is the set of all risky assets, or
''stocks'', of the economy. Its excess return over the risk free rate is
denoted $\widetilde{R}_{s}$. Let $\alpha $ denote the share of the initial
endowment that is invested in the stock fund. The final consumption in this
case equals $\widetilde{C}=X(R+\alpha \widetilde{R}_{s})$. \ The optimal
share $\alpha $ of wealth invested in stocks is then obtained by rewriting
condition (\ref{4}) as 
\begin{equation}
E\widetilde{R}_{s}u^{\prime }(X(R+\alpha \widetilde{R}_{s}))=0.  \label{5}
\end{equation}

What do we know about the properties of the optimal share of wealth invested
\ risky assets as a function of the parameters of the problem? Without
entering into the details of the proofs which can be found for example in
Eeckhoudt and Gollier (1995), we know the following three properties of the
optimal $\alpha :$

\begin{itemize}
\item  If $E\widetilde{R}_{s}>0$, it is optimal to invest in the stock fund (%
$\alpha >0);$

\item  Consider two agents with an identical initial endowment $X$, but
respectively with utility function $u$ and $v$. Individual $u$ invests less
in the risky fund than individual $v$, at all levels of $X$ and for any
distribution of $\widetilde{R}_{s}$, if and only if $u$ is more risk-averse
than $v$ in the sense of Arrow-Pratt, i.e., if $-u^{\prime \prime
}(C)/u^{\prime }(C)$ is larger than $-v^{\prime \prime }(C)/v^{\prime }(C)$
for all $C$.

\item  An increase in initial wealth $X$ always increases the optimal share
invested in the risky fund, for any distribution of $\widetilde{R}_{s}$, if
and only if the index of relative risk aversion $\rho (C)$ is decreasing in $%
C$.
\end{itemize}

These are clear-cut properties.\ The third one in particular is an
hypothesis on preferences that can be tested with data about household
portfolios. The first result is contradicted by the observation that a large
proportion of the population does not hold any risky asset.\ This may be due
to the existence of a participation cost to financial markets. Notice also
that there is an important literature on the effect of a change in the
distribution of excess returns on the demand for stocks. Contrary to the
intuition, it is not true in general that a first-order stochastically
dominated shift in distribution of returns, or also a Rothschild-Stiglitz
increase in risk in returns, does not necessarily reduce the demand for
stocks. This literature did not provide any testable property of optimal
portfolio strategies and will therefore not be covered here.\footnote{%
Gollier (1995) obtained the necessary and sufficient condition on the change
of distribution for a reduction in the demand for stocks by all risk-averse
investors.}

We can also estimate the optimal $\alpha $ by calibrating the model.
Assuming the lognormality of the distribution of $\widetilde{R}_{s}$
together with the constancy of relative risk aversion $\rho $, it can be
shown that \ the solution of equation (\ref{5}) is 
\begin{equation}
\alpha =R\dfrac{E\widetilde{R}_{s}}{\sigma _{s}^{2}}\dfrac{1}{\rho },
\label{6}
\end{equation}
where $\sigma _{s}^{2}$ is the variance of $\widetilde{R}_{s}$. Condition (%
\ref{6}) can also be obtained without lognormality and CRRA, but by assuming
that the portfolio risk is small. In that case, a first-order Taylor
expansion of $u^{\prime }(X(R+\alpha R_{s}))$ around $RX$ in equation (\ref
{5}) directly yields (\ref{6}$)$ as an approximation.

Historically, the equity premium $E\widetilde{R}_{s}$ has been around $6\%$
per year over the century in U.S. markets.\footnote{%
These summary statistics are from Kocherlakota (1996).}\ The standard
deviation of yearly U.S. stock returns over the same period equals $16\%$.
The real risk free rate $R-1$ averaged at $1\%$ per year. Combining this
information with formula (\ref{6}) yields an optimal relative share of
wealth invested in risky assets that equals \ $220\%$ and $55\%$
respectively for a relative risk aversion of $1$ and $4$.\ This very large
shares with respect to observed portfolio compositions by U.S.\ households
is commonly referred to as the Equity Premium Puzzle, as introduced by Mehra
and Prescott (1985). A more standard way to present this puzzle is as
follows: in order to explain actual portfolio composition in the U.S., one
needs to assume a degree of relative risk aversion around 40 for the
representative agent.\ From our discussion about the level of $\rho $, this
is highly implausible. Kocherlakota (1996) provides a survey about the
potential ways to solve the puzzle.

Equations (\ref{5}) and (\ref{6}) have been obtained under the assumption
that all households invest in the same two funds.\ In particular, the
structure of the portfolio of stocks is fixed. Since Wilson (1968), we know
that this is the case at equilibrium only for utility functions exhibiting
harmonic absolute risk aversion (HARA). Absolute risk aversion is harmonic
if its inverse is linear, i.e., if $\left[ A(C)\right] ^{-1}=-u^{\prime
}(C)/u^{\prime \prime }(C)$ is linear in $C$. This is the case for
logarithmic, power and exponential utility functions. When utility functions
are not HARA, some agents will prefer to overinvest in some specific risky
assets that they find underrepresented in the stock fund. For example, some
agents may engage in portfolio insurance schemes by purchasing call options
on stocks. It implies that $C$ would not anymore be linear in the overall
return $\widetilde{R}_{s}$ of the economy, contrary to the specification $%
C(R_{s})=X(R+\alpha R_{s})$.\footnote{%
Leland (1980) determines the characteristics of agents who should purchase
call option on the aggregate risk in the economy.} The simplicity of the
Two-Fund Separation hypothesis is to fully describe the portfolio by a
single variable, $\alpha $. A complete description of optimal portfolios
when the Two-fund separation result does not hold requires us to solve the
system of equation (\ref{2}), (\ref{foc})$.$ The effect of a change in the
risk attitude of the investor is similar to what we stated in the two-fund
case, but this requires to be more cautious about what we mean by a more
risky portfolio.

\section{The optimal static portfolio composition with background risk\label%
{RVsec}}

Up to now, we assumed that the only source of risk faced by the household is
the portfolio risk. This is far from realistic, since most of the observed
volatility of households' earnings comes from variations in labour incomes.
Typically, risks related to human capital cannot be traded on Wall Street.\
And financial intermediaries like insurers are not willing to underwrite
such risks mainly because of moral hazard. Unemployment insurance is
ineffective in most countries.\ This is particularly true for risks of long
term unemployment, which is one of the central concerns of middle class
households in Europe. Thus, it appears to be important to adapt the basic
portfolio problem to include an uninsurable background risk. The question is
how does the presence of a risk on human capital affect the demand for
stocks.

We will assume that the risk on human capital is independent of the
portfolio risk.\ This is clearly unrealistic.\ Most shocks to the economy
affect in the same direction the marginal productivity of labour (wages) and
the marginal productivity of capital (portfolio returns). It implies that
human capital is usually positively correlated with assets value. But
allowing for statistical dependence is not a problem.\ In particular, a
positive correlation makes human capital substitute for stocks. Therefore,
an increase in the correlation will reduce the demand for stocks. It remains
to treat the independence case.

The intuition is strong to suggest that independent risks are substitutes.\
We mean by this that the presence of one risk reduces the demand for other
independent risks. This would mean that background risks have a tempering
effect on the demand for stocks. Households that are subject to a larger
(mean-preserving) uncertainty about their future labour incomes should be
more conservative on their portfolio.

Our theoretical model to treat this question is quite simple.\ Consider an
agent with utility $u$ on his final consumption. He is endowed with capital $%
X$ at the beginning of the period, which can be invested in Arrow-Debreu
securities as in the previous section. What is new here is that the final
wealth of the agent equals the sum of the value of his portfolio and his
labour income.\ Because the risk free part of this income has been included,
discounted, in $X$, we assume that this added has a zero mean.\ This
background risk is denoted $\widetilde{\varepsilon }$ and is independent of
the state of the world $i$. The portfolio problem is \ now written as 
\begin{equation}
\max_{C_{1},...,C_{N}}\quad \sum_{i=1}^{N}p_{i}Eu(C_{i}+\widetilde{%
\varepsilon }),
\end{equation}
subject to the unchanged budget constraint (\ref{2}).

Let us define indirect utility function $v$, with $v(Z)=Eu(Z+\widetilde{%
\varepsilon })$ for all $Z$. In consequence, the above problem can be
rewritten as 
\begin{equation}
\max_{C_{1},...,C_{N}}\quad \sum_{i=1}^{N}p_{i}v(C_{i}),
\end{equation}
subject to (\ref{2}). We conclude that the introduction of an independent
background risk is equivalent to the transformation of the original utility
function $u$ into the indirect utility function $v$.\ This change in the
attitude towards portfolio risks can be signed if and only if the degree of
concavity of these two functions can be ranked in the sense of Arrow-Pratt.
More specifically, the intuition that the background risk affects negatively
the demand for stocks would be sustained by the theory if $v$ is more
concave than $u$. Technically, this means that 
\begin{equation}
E\widetilde{\varepsilon }=0\quad \Longrightarrow \quad \dfrac{-Eu^{\prime
\prime }(C+\widetilde{\varepsilon })}{Eu^{\prime }(C+\widetilde{\varepsilon }%
)}\geq \dfrac{-u^{\prime \prime }(C)}{u^{\prime }(C)}  \label{RV}
\end{equation}
for all $C$ and $\widetilde{\varepsilon }$. This property does not hold in
general. All utility functions which satisfy \ property (\ref{RV}) are said
to be ''risk vulnerable'', to follow a terminology introduced by Gollier and
Pratt (1996). Using Taylor approximations, it is easy to prove that this
property holds for small risks if and only if 
\begin{equation}
A^{\prime \prime }(C)\geq 2A^{\prime }(C)A(C),
\end{equation}
where $A(C)=-u^{\prime \prime }(C)/u^{\prime }(C)$ is absolute risk
aversion. A simple sufficient condition for risk vulnerability is that $A$
be decreasing and convex.\ Because the proof of this result is simple, we
reproduce it here.\ The left-hand condition in (\ref{RV}) can be rewritten
as 
\begin{equation}
EA(C+\widetilde{\varepsilon })u^{\prime }(C+\widetilde{\varepsilon })\geq
A(C)Eu^{\prime }(C+\widetilde{\varepsilon })
\end{equation}
But a decreasing and convex function $A$ implies that 
\begin{equation}
EA(C+\widetilde{\varepsilon })u^{\prime }(C+\widetilde{\varepsilon })\geq %
\left[ EA(C+\widetilde{\varepsilon })\right] \left[ Eu^{\prime }(C+%
\widetilde{\varepsilon })\right] \geq A(C)Eu^{\prime }(C+\widetilde{%
\varepsilon }).  \label{7}
\end{equation}
This is what we had to prove.\ The first inequality in (\ref{7}) comes from
the fact that both $A$ and $u^{\prime }$ are decreasing in $\varepsilon $.
The second inequality is an application of Jensen's inequality, together
with $A^{\prime \prime }\geq 0$ and $E\widetilde{\varepsilon }=0$.

Observe that the classically used power utility function (\ref{CRRA}) has a
decreasing and convex absolute risk aversion, since $A(C)=\rho /C$. Our
conclusion is that the introduction of a background risk in the calibration
of household portfolios will imply a rebalancement towards the risk free
asset.

\section{The optimal dynamic portfolio composition with complete markets%
\label{repeated}}

One of the main deficiencies of the standard portfolio problem that we
examined in section \ref{standardsec} comes from the fact that it is static.
Technically, it is descriptive of the situation of an agent who invests his
wealth in prospect of his retirement that will take place in one year. In
this section, we will examine the optimal portfolio composition for people
who have more periods to go before retirement.\ In other words, we will
examine the relationship between portfolio risk and time horizon.

In the formal literature, the horizon-riskiness issue has received the
greatest attention addressing portfolios appropriate to age. Samuelson
(1989) and several others have asked: ``As you grow older and your
investment horizon shortens, should you cut down your exposure to lucrative
but risky equities?''Conventional wisdom answers affirmatively, stating that
long-horizon investors can tolerate more risk because they have more time to
recoup transient losses. This dictum has not received the imprimatur of
science, however. As Samuelson (1963, 1989) in particular points out, this
``time-diversification'' argument relies on a fallacious interpretation of
the Law of Large Numbers: repeating an investment pattern over many periods
does not cause risk to wash out in the long run. In the next section, we
examine an alternative concept of time diversification.

To address this question, \ we consider the problem of an agent who has to
manage a portfolio over time in order to maximize the expected utility of
his consumption at retirement.\ In this section, we abstract ourself from
the consumption-saving problem by assuming that this portfolio is specific
for retirement and that it cannot be used for consumption before retirement.
Also, we normalize the risk free rate to zero.\ This implies that a young
and an old investor with the same discounted wealth today can secure the
same level of consumption at retirement by investing in the riskless asset.\
If the risk free rate would be positive, the younger investor would
implicitly be wealthier.

We can now understand why the age of the investor has an ambiguous effect on
the optimal portfolio composition. Contrary to the old investor, the young
agent has an option to invest in stocks in the subsequent periods. This
option has a positive value, which makes the younger agent implicitly
wealthier, at least on average.\ Under decreasing absolute risk aversion
(DARA), that makes him less risk-averse.\ This wealth effect affects
positively the share of wealth invested in stocks. But taking a risk has not
the same comparative statics effect than getting its expected net payoff for
sure, as stressed in section \ref{RVsec}. The fact that the younger agent
will take a portfolio risk in the future plays the role of a background risk
with respect to his portfolio choice problem when he is young. This risk
effect goes the opposite direction than the wealth effect, under risk
vulnerability. All this is made more complex by the dynamic aspect of the
problem, in the sense that the portfolio risk that the young will take in
the future can be made contingent to the accumulated portfolio value in
previous periods.

It is standard to solve this kind of dynamic problem by using backward
induction. Let $T$ denote the number of periods before retirement. Because
we assumed that the only decision maker's concern is the welfare of the
investor at the time of retirement, the objective is to maximize $Eu(%
\widetilde{C}_{T})$. Assuming complete markets against all risks occurring
during the last period, the problem of the decision maker is exactly the
same as problem (\ref{1}), (\ref{2}). Denoting $X_{T-1}$ for the wealth that
has been accumulated at the beginning of the last period, this problem is
rewritten as 
\begin{equation}
V_{T-1}(X_{T-1})=\max_{C_{1T},...,C_{NT}}\quad \sum_{i=1}^{N}p_{i}u(C_{iT})
\label{VT-1}
\end{equation}
\begin{equation}
s.t.\quad \sum_{i=1}^{N}p_{i}\pi _{i}C_{iT}=X_{T-1}
\end{equation}
where $C_{iT}$ is the demand for the Arrow-Debreu security associated to
state $i$. Because we assume that there is no serial correlation in asset
returns and that the random walk of returns is stationary, we did not
indexed \ state prices $\pi _{i}$ and probabilities $p_{i}$ by $T$. In
short, financial risks are assumed to be the same at each period. Again,
with a negative serial correlation in assets returns, investing in stocks in
the future may serve as a partial insurance for portfolio risks taken
today.\ This would provide an additional incentive for young investors to
raise the share of their wealth invested in stocks. We do not consider this
possibility here.

Equation (\ref{VT-1}) introduced the (Bellman) value function $V_{T-1}$ into
the picture. $V_{T-1}(X_{T-1})$ is the maximum expected utility of
consumption at retirement that can be obtained when the household
accumulated a capital $X_{T-1}$ at the end of period $T-1$. Then, the
portfolio problem at the beginning of period $T-1$ when the accumulated
capital at that date is $X_{T-2}$ can be written as 
\begin{equation}
V_{T-2}(X_{T-2})=\max_{X_{1T-1},...,X_{NT-1}}\quad
\sum_{i=1}^{N}p_{i}V_{T-1}(X_{iT-1})  \label{VT-2}
\end{equation}
\begin{equation}
s.t.\quad \sum_{i=1}^{N}p_{i}\pi _{i}X_{iT-1}=X_{T-2}
\end{equation}
where $X_{iT-1}$ is at the same time the demand at $T-1$ for the
Arrow-Debreu security associated to state $i$ and the accumulated capital at
the end of period $T-1$ if state $i$ occurs. The optimal portfolio strategy
in period $T-1$ is to find a portfolio which maximizes the expected value $%
V_{T-1}$ (which is itself the maximal expected utility of final consumption)
of the accumulated wealth at the end of the period. This will generate a
dynamic portfolio strategy that is optimal, in the sense that it will
maximize the expected utility of final consumption. Pursuing this method by
backward induction, we obtain the full description of the optimal nonmyopic
portfolio strategy, which is given by the set of functions $\left\{
X_{it}(X_{t-1})\mid i=1,...,N;\quad t=1,...,T\right\} ,$ with $%
X_{iT}(X)=C_{iT}(X)$.

The question is to determine the impact of index $t$ on the optimal
portfolio composition $(X_{1t}(X),...,X_{Nt}(X))$ for a given wealth $X$
accumulated at the beginning of the period. To illustrate, let us limit the
analysis to the comparison between the optimal portfolio composition in
periods $T-1$ and $T$, assuming the same $X=X_{T-2}=X_{T-1}$. We see that
the only difference between these two decision problems (\ref{VT-1}) and (%
\ref{VT-2}) comes from the replacement of the original utility function $u$
on retirement consumption by the value function $V_{T-1}$ on the accumulated
wealth \ at the end of the period. We know that this is the case if and only
if the degree of concavity of $V_{T-1}$ is comparable to the degree of
concavity of $u$ in the sense of Arrow-Pratt. Thus, we need to evaluate the
degree of concavity of $V_{T-1}$.\ This is done as follows.\ At the last
period, the optimal portfolio composition is characterized by $C_{iT}(X)$
that is the solution of 
\begin{equation}
u^{\prime }(C_{iT}(X))=\xi (X)\pi _{i}
\end{equation}
subject to the budget constraint $\sum_{i}p_{i}\pi _{i}C_{iT}(X)=X.$\ Fully
differentiating this condition with respect to $X$ and eliminating $\pi _{i}$
yields 
\begin{equation}
C_{iT}^{\prime }(X)=\dfrac{-\xi ^{\prime }(X)}{\xi (X)}\tau (C_{iT}(X)),
\label{8}
\end{equation}
where $\tau (C)=-u^{\prime }(C)/u^{\prime \prime }(C)=\left[ A(C)\right]
^{-1}$ is the degree of absolute risk tolerance of the agent. From the
budget constraint, we have that $\sum_{i}p_{i}\pi _{i}C_{iT}^{\prime }(X)=1$%
.\ Using (\ref{8}), it implies that 
\begin{equation}
\dfrac{-\xi ^{\prime }(X)}{\xi (X)}=\left[ \sum_{i}p_{i}\pi _{i}\tau
(C_{iT}(X))\right] ^{-1}.
\end{equation}
On the other side, we know from the standard Lagrangian method that $%
V_{T-1}^{\prime }(X)=\xi (X).$ It implies that 
\begin{equation}
\dfrac{-V_{T-1}^{\prime }(X)}{V_{T-1}^{\prime \prime }(X)}=\sum_{i}p_{i}\pi
_{i}\tau (C_{iT}(X)).  \label{Wilson}
\end{equation}
The left-hand side of this equation is the weighted expectation of the
absolute risk tolerance evaluated at the random final wealth. It is an
expectation under the assumption that $\sum_{i}p_{i}\pi _{i}=1$, which means
that the risk free rate is zero. Condition (\ref{Wilson}) means that the
relative risk tolerance of the value function used to measure the optimal
attitude toward portfolio risk at $T-1$ is a weighted average of ex post
absolute risk tolerance. This result has first been obtained by Wilson
(1968) in the context of static risk sharing. Suppose that function $\tau $
be convex.\ By the Jensen's inequality, we then obtain that 
\begin{equation}
\dfrac{-V_{T-1}^{\prime }(X)}{V_{T-1}^{\prime \prime }(X)}\geq \tau \left(
\sum_{i}p_{i}\pi _{i}C_{iT}(X)\right) =\tau (X)=\dfrac{-u^{\prime }(X)}{%
u^{\prime \prime }(X)}.
\end{equation}
Because this is true for any $X$, $V_{T-1}$ is less concave than $u$ in the
sense of Arrow-Pratt. We conclude that {\it the convexity of absolute risk
tolerance is necessary and sufficient for younger people to take more
portfolio risk, }ceteris paribus. By symmetry, the concavity of absolute
risk tolerance is necessary and sufficient for younger people to take less
portfolio risk. This result has first been obtained by Gollier and
Zeckhauser (1998) who extended this result to the more difficult case of
incomplete markets.

The limit case is when absolute risk tolerance is linear, which corresponds
to the set of HARA utility functions. For HARA preferences, the age of the
investor has no effect on the optimal composition of her portfolio. Because
this set of functions contains the only one for which a complete analytical
solution for the optimal portfolio strategies can be obtained, it is not a
surprise that most economist specialized in this field recommend a
age-independent portfolio strategy. It must be stressed however that there
is no strong argument in favor of HARA functions, except for their
simplicity. Whether absolute risk tolerance is concave or convex, or neither
of the two, remains an open question for empirical investigation. Guiso,
Jappelli and Terlizesse (1995) observed a bell curve for the relationship
between age and the share of wealth invested in stocks, suggesting that $%
\tau $ is neither concave nor convex.

We did not take into account of an important phenomenon in this analysis.\
Namely, younger people usually face a larger background risk on their human
capital. As we grow older, the uncertainty on one's human capital is
revealed by labour markets. From our discussion in section \ref{RVsec}, it
is likely to imply a tempering effect on the optimal demand for stocks.

\section{Self-insurance and time diversification\label{self}}

Up to now, we assumed that  households have no control on what they put in
and out of the fund of assets at each period. In other words, we assumed
that the capital is accumulated for a single objective, which is consumption
at old age. Obviously, this is not a reasonable assumption.\ Most households
also use savings for a precautionary motive.\ They increase their saving in
case of an unexpected transitory increase in income, and they reduce their
saving effort, or even they become borrowers, in case of an adverse
transitory shock on their incomes. Reciprocally, agents can decide to reduce
their saving effort if their have been lucky on their portfolio. This means
that agents can smooth shocks on their accumulated wealth by increasing or
reducing their consumption over several periods. As we will see, consumption
smoothing plays the role of self-insurance.\ That implies in turn more risk
taking.

The simplest model that we could imagine for self-insurance over time is a
model in which  agents live for $N$ periods $t=1,...,N$. They have a
cash-on-hand $K$ prior to period $1$. At each period, they receive a
non-capital income $W$. They consume $C_{t}$ from it in period $t$. The
remaining is saved in a risk free asset whose gross return is $R$. Agents
are allowed to take risk prior to period $1$, but they are prohibited to do
so afterwards. This simplified assumption is made here because we want to
isolate the self-insurance effect of time. We will come back to this point
later on.

Let $\widetilde{\varepsilon }$ denote the net payoff of the risk taken prior
to period $1$. In period $1$, the discounted wealth $X$ will be the sum of
the cash-on-hand, the discounted value of future incomes and the actual
payoff \ of the lottery: $X=K+$ $\sum_{t}R^{-t}W+\varepsilon $. Given $X,$
agents select the consumption plan that maximizes their discounted lifetime
utility $V(X)$, which takes the following form: 
\begin{equation}
V(X)=\max_{C_{1},...,C_{N}}\quad \sum_{t=1}^{N}\beta ^{t}u(C_{t})
\label{inter}
\end{equation}
\begin{equation}
s.t.\quad \sum_{t=1}^{N}R^{-t}C_{t}=X,
\end{equation}
where $\beta $ is the discount factor on utility. It yields an optimal
consumption plan $C_{t}(X)$ which is a function of the discounted wealth of
the agent. To solve this problem, we can use the fact that it is equivalent
to program (\ref{1}) with $p_{t}=\beta ^{t}$ and $\pi _{t}=(R\beta )^{-t}$.\
>From equation (\ref{Wilson}), we directly infer that 
\begin{equation}
\dfrac{-V^{\prime }(X)}{V^{\prime \prime }(X)}=\sum_{t=1}^{N}R^{-t}\tau
(C_{t}(X)).  \label{w2}
\end{equation}

Condition (\ref{w2}) characterizes the degree of tolerance towards the risk $%
\widetilde{\varepsilon }$ taken prior to period $1$. The simplest case is
obtained when $\beta =R$.\ In that case, we know that it is optimal to
smooth consumption perfectly.\ It yields 
\begin{equation}
\dfrac{-V^{\prime }(X)}{V^{\prime \prime }(X)}=\left[ \sum_{t=1}^{N}R^{-t}%
\right] \tau \left( \dfrac{X}{\sum_{t=1}^{N}R^{-t}}\right) =\left[
\sum_{t=1}^{N}R^{-t}\right] \tau \left( W+\dfrac{K+\varepsilon }{%
\sum_{t=1}^{N}R^{-t}}\right)  \label{w3}
\end{equation}
when the risk free rate is small. Now, compare {\it two agents whose wealth
levels per period (}$X/\sum R^{-t}${\it ) are the same}.\ But one of the two
agents has one period to go ($N=1$), whereas the other agent has $N>1$
periods to go. Equation (\ref{w3}) tells us that the agent with time horizon 
$N$ will be $\sum R^{-t}\approx N$ times more risk tolerant than the agent
with only one period to go. Thus, we conclude that there is a strong time
diversification effect which takes place in this model. The intuition is
quite simple.\ Each dollar of loss or gain will be equally split into $1/N$
of a dollar reduction or increase in the consumption in each period. This is
an efficient risk-sharing scheme of the different future selves representing
the household over time.\ This ability to diversify the single risk over
time induces the agent to be more willing to purchase it.

Our ceteris paribus assumption above was that agents have the same wealth
level per period.\ This means that a reduction in the time horizon does not
affect the feasible consumption level per period. This is the case for
example when the agent has no cash-on-hand in period 1 in such a way that
wealth $X$ comes solely from discounting future incomes.\ We can
alternatively compare two agents with different time horizons, but with
identical discounted wealth $X$. In this case, the agent with a longer time
horizon will consume less at each period. Under DARA, that will induce more
risk aversion, and it is not clear which of the time diversification effect
and this wealth effect will dominate. The limit case is when absolute risk
tolerance is homogenous of degree 1 with respect to consumption.\ This is
the case under CRRA, since we have $\tau (C)=C/\gamma $. Equation (\ref{w3})
is rewritten as $-V^{\prime }(X)/V^{\prime \prime }(X)=X/\gamma $ in that
case. This is independent of the time horizon. Under CRRA, the wealth effect
just compensates the time diversification effect: two CRRA agents with the
same aggregate wealth but facing different time horizons will have the same
attitude towards the single risk $\widetilde{\varepsilon }$. When absolute
risk tolerance is subhomogeneous., the agent will be more tolerant to a
single risk if it is resolved earlier in his life.

Remember that equation (\ref{w3}) holds only under the assumption that a
constant consumption over the lifetime is optimal, i.e., when $R\beta =1$.
When $R$ and $\beta ^{-1}$ are not equal, we must use equation (\ref{w2})
with the optimal consumption plan ( $C_{1},...,C_{N})$.\ If absolute risk
tolerance is convex, Jensen's inequality applied to this equation implies
that 
\begin{equation}
\dfrac{-V^{\prime }(X)}{V^{\prime \prime }(X)}\geq \left[
\sum_{t=1}^{N}R^{-t}\right] \tau \left( \dfrac{\sum_{t=1}^{N}R^{-t}C_{t}}{%
\sum_{t=1}^{N}R^{-t}}\right) =\left[ \sum_{t=1}^{N}R^{-t}\right] \tau \left( 
\dfrac{X}{\sum_{t=1}^{N}R^{-t}}\right) .  \label{w4}
\end{equation}
This is equation (\ref{w3}), except that the equality has been replaced by
an inequality. Thus, under convex absolute risk tolerance, the time
diversification effect is even stronger than explained above. \ Technically,
this result has been obtained by following the same procedure than  the one
for the complementarity of repeated risks in section \ref{repeated}. Here,
the optimal fluctuations of consumption over time, rather than across
states, are complementary to risk taking if $\tau $ is convex.

The model presented in this section is not realistic because of the
assumption that there is a single risk in the lifetime of the investor. We
now combine the different effects of time horizon on the optimal
instantaneous portfolio:

\begin{enumerate}
\item  The {\it complementarity effect} of repeated risks over time: the
option to take risk in the future raises the willingness to take risk today
under convex absolute risk tolerance;

\item  The {\it time diversification effect}: the opportunity to smooth
shocks on capital by small variations of consumption over long horizons
raises the willingness to take risk;

\item  The {\it wealth effect}: for a given discounted wealth, a longer
horizon means less consumption at each period, which reduces the willingness
to take risk under DARA.
\end{enumerate}

We consider the following model.\ Investors can consume, save and take risk
at each period from $t=1$ to $t=T$. At each period, there is some
uncertainty about which state of the world $i=1,...,N$ will prevail at the
end of the period. Within each period $t$, the agent begins with the
selection of a portfolio.\ After observing the state of the world and the
value $X_{t}$ of the portfolio, the agent decides how much to consume $(C_{t}
$) and how much to save for the next period ($S_{t}$).\ We can decompose the
above-mentioned effects by using backward induction again.\ This is done in
the following steps:

\begin{itemize}
\item  In period $T$, the agent selects his optimal portfolio of
Arrow-Debreu securities by solving program (\ref{VT-1}), where $X_{T-1}$ is
replaced by $S_{T-1}$. The transformation from $u$ to $V_{T-1}$ describes
the complementarity (or substituability)  effect of repeated risk.

\item  At the end of period $T-1$, after observing the state of the world
and the associated value $X_{T-1}$ of the portfolio, the agent solves his
consumption-saving problem which is written as 
\begin{equation}
\widehat{V}_{T-1}(X_{T-1})=\max_{C}\quad u(C)+\beta V_{T-1}(R(X_{T-1}-C)).
\label{9}
\end{equation}
This operation describes the time diversification effect and the wealth
effect of time horizon.

\item  The agent determines the optimal composition of his portfolio at the
beginning of period $T-1$ by solving program (\ref{VT-2}), but with function 
$V_{T-1}$ being replaced by function $\widehat{V}_{T-1}$ defined by (\ref{9}%
) to take into account of the possibility to smooth consumption over time.
\end{itemize}

Going back to the original question of how does time horizon affect the
optimal structure of households' portfolios, we must compare the degree of
concavity of $\widehat{V}_{T-1}$ with respect to the degree of concavity of $%
u$. If \ $\widehat{V}_{T-1}$ is less concave than $u$, it would imply that,
ceteris paribus, one selects a riskier portfolio in period $T-1$ than in
period $T$. \ The CRRA function is an instructive benchmark. Because CRRA is
a special case of HARA, we know from section \ref{repeated} that $%
V_{T-1}(.)\equiv hu(.)$: the option to invest in risky assets in period $T$
does not affect the degree of concavity of \ the value function.\ It implies
that program (\ref{9}) is a special case of program (\ref{inter}) with $%
\beta $ being replaced by $\beta h$. Because $u$ is CRRA, we also know that
the time diversification effect is completely offset by the wealth effect,
so that $\widehat{V}_{T-1}(.)\equiv \widehat{h}u(.)$: \ $\widehat{V}_{T-1}$
as the same concavity than $u$. We conclude that myopia is optimal under
CRRA. The optimal dynamic portfolio strategy is obtained by doing as if each
period would be the last one before retirement. There is no effect of time
horizon on the optimal portfolio, ceteris paribus.\ This result was already
in Merton (1969) and Samuelson (1969).\ Mossin (1969) showed that CRRA is
also necessary for optimal myopia.\ 

It is easy to combine our other results.\ For example, a larger time horizon
induces riskier portfolios if absolute risk tolerance is convex and
subhomogeneous. On the contrary, a larger time horizon induces more
conservative portfolios if absolute risk tolerance is concave and
superhomogeneous.

\section{The liquidity constraint}

In the previous sections, we assumed that markets are frictionless.\ There
are no transaction costs to exchange assets and no cap on the risk that can
be taken on financial markets. Moreover, we assumed that the borrowing rate
is equal to the lending rate on the credit market.\ Whereas the effect of
the introduction of market imperfections on the optimal portfolio is easy to
examine in a static model, the analysis becomes complex in a dynamic
framework. In this section, we will focus on the effect of a liquidity
constraint on the optimal portfolio. A liquidity constraint is an extreme
form of inefficiency on the credit market.\ It forces households to maintain
a sufficient level of cash-on-hand in each period. In other words, it
prohibits households to borrow too much.\ A weaker version of the liquidity
constraint is when the borrowing rate is larger than the lending rate.

The effect of \ a liquidity constraint on the optimal portfolio is easy to
understand. Remember that it is optimal for households to smooth adverse
shocks on their portfolio return by reducing consumption over a long period
of time. This is done by reducing saving immediately after the adverse shock
to finance short term consumption. This is the mechanism behind consumption
smoothing.\ If the shock is large enough, it can be the case that some
households become short of cash, i.e. that they become net borrowers for a
short period of time. Under the liquidity constraint, this is not allowed.
Thus, these households will be forced to drastically reduce their short term
consumption. They will not be able to time diversify their portfolio risk
anymore.\ This will raise their degree of aversion towards portfolio risk.

The simplest model to describe this phenomenon is similar to the one
presented in section \ref{self}.\ We assume that households live for $N$
periods, and that they start with a cash-on-hand $K$.\ The only possibility
to take risk is at the beginning of period $t=1$. Households have a
permanent income $W.\;$\ They must maintain a positive cash-on-hand
permanently.\ We also assume that $R\beta =1$.\ We showed above how this
simplest model can be extended by the introduction of repeated risk taking
and $R\beta \neq 1$.

Let $\varepsilon $ denote the payoff of the risk taken at the beginning of
period $1$. Thus, when households determine their optimal consumption plan ,
they are endowed with a cash-on-hand $K+\varepsilon $. Because perfect
consumption smoothing is optimal in the absence of the liquidity constraint,
households would consume 
\begin{equation}
C=W+\dfrac{K+\varepsilon }{\sum_{t=1}^{N}R^{-t}}
\end{equation}
at each period. It yields a degree of risk aversion to risk $\widetilde{%
\varepsilon }$ ex ante that is given by equation (\ref{w3}). How is the
degree of aversion to risk $\widetilde{\varepsilon }$ affected by the
constraint that the cash-on-hand must be positive? In the unconstrained
case, the cash-on-hand at the end of the first period would be 
\begin{equation}
K+\varepsilon +W-C=(K+\varepsilon )\dfrac{-1+\sum_{t=1}^{N}R^{-t}}{%
\sum_{t=1}^{N}R^{-t}}.
\end{equation}
If $\varepsilon $ is larger than $K$, the constraint will not be binding and
the local measure of concavity of the value function $V$ , measured at $%
X=K+\varepsilon +W\sum_{t}R^{-t}$, will be as in equation (\ref{w3}).\ Of
course, the difference is for the local measure of the value function at
lower wealth levels, where the liquidity constraint is binding. Indeed, when 
$\varepsilon <-K$, households would like to finance short term consumption
by borrowing money, but they are not allowed to do so.\ They are forced to
swallow the drop in their wealth in one shot by reducing their immediate
consumption accordingly.\ They \ will consume $W$ afterwards. For those
values of $\varepsilon $, we obtain that 
\begin{equation}
\dfrac{-V^{\prime }(X)}{V^{\prime \prime }(X)}=\tau (W+K+\varepsilon ).
\end{equation}
Under DARA, we have that 
\begin{equation}
\tau (W+K+\varepsilon )\leq \tau (W+\dfrac{K+\varepsilon }{%
\sum_{t=1}^{N}R^{-t}})\leq \left[ \sum_{t=1}^{N}R^{-t}\right] \tau (W+\dfrac{%
K+\varepsilon }{\sum_{t=1}^{N}R^{-t}}).
\end{equation}
This means that the local degree of concavity of the value function is
smaller with a binding liquidity constraint than without it.\ Moreover, it
is reduced by a factor at least equal to $\sum_{t=1}^{N}R^{-t}\approx N$ .
There is no more time diversification.

The implication of this analysis is that households that are more likely to
be liquidity constrained will adopt a more conservative portfolio strategy.

\section{Conclusion}

The theory of dynamic portfolio management remains a fascinating field of
research.\ The similarity between the static portfolio problem under
uncertainty and the dynamic consumption problem under certainty is helpful
to understand how risk and time interact with each other to generate a
dynamically optimal portfolio strategy. We have shown in particular that the
concavity or convexity of the absolute risk tolerance with respect to
consumption is important to characterize the effect of time on risk taking.\
This is because the absolute tolerance to portfolio risk in the static
portfolio problem is a weighted average of ex post absolute risk tolerances
on consumption.\ Similarly, the absolute risk tolerance on wealth in the
dynamic consumption problem under certainty is equal to the discounted value
of future absolute risk tolerances on consumption. Thus, absolute risk
tolerances can be summed, contrary to  absolute risk aversions.

This very basic result is at the origin of our understanding of how 
portfolio risks occurring at different points in time interact with each
other.\ We also showed how the familiar notion of time diversification can
be justified on a theoretical ground. We presented several testable
hypotheses using data on household portfolios:

\begin{enumerate}
\item  Wealthier people own more risky assets (under decreasing absolute
risk aversion);

\item  Wealthier people invest a larger share of their wealth in risky
assets (under \ decreasing relative risk aversion);

\item  Households with a riskier human capital invest less in risky assets
(under risk vulnerability);

\item  Households who can invest longer in risky assets will invest more in
them (under convex absolute risk tolerance);

\item  Households who are more likely to be liquidity constraint in the
future will invest less in risky assets (under decreasing absolute risk
aversion).
\end{enumerate}

Empirical evidences related to these hypotheses are presented in various
chapters of this book. 

\newpage 

\begin{center}
{\bf Bibliography}
\end{center}

\vspace{.30cm}

\begin{verse}
Arrow, K. J. (1971). {\it Essays in the Theory of Risk Bearing}. Chicago:
Markham Publishing Co.

Barsky, R.B., F.T. Juster, M.S.\ Kimball and M.D.\ Shapiro, (1997),
Preferences parameters and behavioral heterogeneity: an experimental
approach in the health and retirement study, {\it Quarterly Journal of
Economics, 112}, 537-579.

Constantinides, G.M., (1990) Habit formation: A resolution of the equity
premium puzzle, {\it Journal of Political Economy}, 98, 519-543.

Eeckhoudt, L. and C. Gollier, (1995), {\it Risk: Evaluation, Management and
Sharing}, Hearvester Wheatsheaf, 347 pages.

Gollier, C., (1995), The Comparative Statics of Changes in Risk Revisited, 
{\em Journal of Economic Theory} , 66, 522-536.

Gollier, C. and J.W. Pratt, (1996), Risk vulnerability and the tempering
effect of background risk, {\it Econometrica}, 64, 1109-1124.

Gollier, C. and R.J. Zeckhauser, (1998), Horizon Length and Portfolio Risk,
Discussion Paper, University of Toulouse.

Guiso, L., T. Jappelli and D. Terlizzese, (1996), Income risk, borrowing
constraints, and portfolio choice, {\it American Economic Review}, 86,
158-172.

Jagannathan, R. and N.R.\ Kocherlakota, (1996), Why should older people
invest less in stocks than younger people?, {\it Federal Reserve Bank of
Minneapolis Quarterly Review}, 20, 11-23.

Kocherlakota, N.R., (1996), The Equity Premium: It's Still a Puzzle,{\it \
Journal of Economic Literature}, 34, 42-71.

Kreps, D.M., and E.L. Porteus, (1978), Temporal resolution of uncertainty
and dynamic choice theory,{\it {\rm \ Econometrica, 46, 185-200. }}

Leland, H.E., (1980), Who Should Buy Portfolio Insurance?, {\it The Journal
of Finance}, 35, 581-596.

Merton, R.C., (1969), Lifetime portfolio selection under uncertainty: The
continuous- time case, {\it Review of Economics and Statistics}, 51, pp.
247-257.

Mossin, J., (1968), Optimal multiperiod portfolio policies,{\it \ Journal of
Business}, 215-229.

Pratt, J., (1964), Risk Aversion in the Small and in the Large, {\it %
Econometrica}, 32, 122-136.

Mehra, R. and E. Prescott, (1985), The Equity Premium: A Puzzle, {\it %
Journal of Monetary Economics}, 10, 335-339.

Samuelson, P.A., (1963), Risk and uncertainty: the fallacy of the Law of
Large Numbers, {\it Scientia}, 98, 108-113.

Samuelson, P.A., (1969), Lifetime portfolio selection by dynamic stochastic
programming, {\it Review of Economics and Statistics}, 51, 239-246.

Samuelson, P.A., (1989), The judgement of economic science on rationale
portfolio management: indexing, timing, and long-horizon effects, {\it %
Journal of Portfolio Management}, Fall 1989, 3-12.

Selden, L., (1979), An OCE analysis of the effect of uncertainty on saving
under risk independence, {\it {\rm Review of Economic Studies, }}73-82.{\it 
{\rm \ }}

von Neumann, J.\ and O.\ Morgenstern, (1944), {\it Theory of Games and
Economic Behavior}, Princeton: Princeton University Press, 1944.

Weil, P., (1990), Nonexpected utility in macroeconomics, {\it Quarterly
Journal of Economics},105, 29-42.

Wilson, R. (1968). ``The theory of syndicates'', {\it Econometrica }36,
113-132.
\end{verse}

\end{document}

